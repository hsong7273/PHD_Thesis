\chapter{Backgrounds}
\label{chapter:Backgrounds}
\thispagestyle{myheadings}

% set this to the location of the figures for this chapter. it may
% also want to be ../Figures/2_Body/ or something. make sure that
% it has a trailing directory separator (i.e., '/')!
\graphicspath{{6_Chapter_Backgrounds/Figures/}}

In the search for excited-state double-beta decays in KamLAND-Zen, the signal is extracted through a fit to the reconstructed visible energy spectrum. While Chapter~\ref{chapter:reco_select} described the event selection criteria applied to suppress background contributions in the data, the purpose of this chapter is to describe how the residual backgrounds that pass those selections are modeled and quantified. This chapter presents the expected energy distributions of all relevant background processes, together with the methods used to estimate their rates and associated uncertainties. These background models are implemented in the spectral fit described in Chapter~\ref{chapter:Analysis}, along with any independent constraints—such as delayed-coincidence measurements or external calibration studies—that can be placed on their intensities.

In many KamLAND-Zen analyses, spatial information—particularly the reconstructed radial distribution—is used as an additional discriminator between signal and background. However, due to limitations in the modeling of the radial distribution for this analysis, only the innermost region of the detector is used. As a result, the background discrimination relies primarily on spectral shape rather than spatial separation. Further discussion of the fiducial volume selection adopted for the excited-state analysis is provided in Chapter~\ref{chapter:Analysis}.

\section{\2nbb : Double-Beta Decay}
Two-neutrino double-beta decay (\2nbb) constitutes by far the dominant background in the search for excited-state double-beta decays (\2nbbs). In practice, the excited-state analysis is largely reduced to a search for small distortions in the continuous \2nbb energy spectrum, making an accurate modeling of this background essential. In KamLAND-Zen, two xenon isotopes undergo \2nbb decay: $^{136}$Xe and $^{134}$Xe. Of the xenon dissolved in the xenon-loaded liquid scintillator (XeLS), approximately 90\% is $^{136}$Xe and about 9\% is $^{134}$Xe. The remaining fraction consists of other xenon isotopes at negligible levels.

The \2nbb decay of $^{136}$Xe has been observed and precisely measured by KamLAND-Zen and other experiments. In this analysis, the overall normalization of the $^{136}$Xe \2nbb background is allowed to float freely in the spectral fit. This approach accounts for uncertainties in exposure, detection efficiency, and possible correlations with other background components. In contrast, the \2nbb decay of $^{134}$Xe has not yet been observed. The current world-leading limit on its half-life is $T_{1/2} > 8.2 \times 10^{20}$ years at 90\% confidence level~\cite{klz_xe134}. If this limit is saturated, the corresponding decay rate in KamLAND-Zen XeLS would be approximately $2.7 \times 10^{4}$ events/day/kton.

Although $^{134}$Xe decay could represent a secondary physics goal of KamLAND-Zen, its contribution is effectively unobservable in the present analysis. The expected signal is completely masked by residual low-energy backgrounds, most notably from $^{85}$Kr and $^{210}$Bi contamination in the detector. Furthermore, theoretical calculations predict the $^{134}$Xe \2nbb half-life to be approximately three orders of magnitude longer than that of $^{136}$Xe, implying a correspondingly smaller decay rate. For these reasons, the contribution of $^{134}$Xe \2nbb decays is neglected in this analysis, and only the $^{136}$Xe \2nbb component is included in the background model.


\section{Radioactive Contamination}
Radioactive contamination from naturally occurring radionuclides constitutes an important class of backgrounds in KamLAND-Zen. These backgrounds originate from trace impurities in detector materials, including the liquid scintillator volumes, balloon films, and structural components. This section describes the modeling and treatment of the dominant radioactive contaminants that contribute to the excited-state double-beta decay analysis.

\subsection{$^{238}$U Series}
$^{238}$U is a naturally occurring radionuclide with a half-life of
$T_{1/2} = 4.468 \times 10^{9}$ years and is responsible for approximately 40\% of the radioactive heat generated within the Earth. As an omnipresent contaminant, trace amounts of $^{238}$U are present in essentially all detector materials.

The $^{238}$U decay chain includes a sequence of radioactive daughters such as Th, Pa, Ra, Rn, Po, Pb, Bi, and Tl. In the background model, all decays in the chain that produce visible energy above 0.5~MeV are included. Following standard KamLAND-Zen practice, the decay chain is divided into two sub-series based on the presence of $^{222}$Rn. Decays occurring upstream of $^{222}$Rn are referred to as \emph{Series~1} ($^{238}$U$_{\mathrm{s1}}$), while decays downstream of $^{222}$Rn are referred to as \emph{Series~2} ($^{238}$U$_{\mathrm{s2}}$). The Series~1 component is introduced into the detector primarily through material contamination during detector construction and is therefore assumed to be in secular equilibrium. In contrast, the Series~2 component is introduced into the detector predominantly through $^{222}$Rn ingress during the initial xenon loading into the XeLS. Because the mechanisms and histories of introduction differ, the rates of $^{238}$U$_{\mathrm{s1}}$ and $^{238}$U$_{\mathrm{s2}}$ are treated as independent parameters in the spectral fit.

In this analysis, the $^{238}$U$_{\mathrm{s1}}$ rate is allowed to float freely and is determined directly from the energy spectrum. The $^{238}$U$_{\mathrm{s2}}$ rate is constrained using the measured rate of tagged $^{214}$Bi decays identified via the delayed-coincidence technique described in Chapter~\ref{chapter:reco_select}. This constraint is implemented as a penalty term in the likelihood function.


\subsection{$^{232}$Th Series}
$^{232}$Th is another naturally occurring radioactive contaminant that contributes to the background in KamLAND-Zen. As with the $^{238}$U chain, the $^{232}$Th decay series is divided into two sub-series to account for possible breaks in secular equilibrium. Decays occurring upstream of $^{228}$Th are referred to as \emph{Series~1}, while those downstream of $^{228}$Th are referred to as \emph{Series~2}. All decays with visible energy above 0.5~MeV are included in the background model.

The rates of the two $^{232}$Th sub-series are treated as independent free parameters in the spectral fit. In the higher-energy region between 3 and 5~MeV, the dominant contribution arises from the $\beta$ decay of $^{208}$Tl. As a result, the $^{208}$Tl rate can be well constrained by the spectral fit, and the rates of other decays in the $^{232}$Th chain are determined relative to the fitted $^{208}$Tl activity.



\subsection{$^{40}$K}
$^{40}$K is a common naturally occurring radionuclide that decays via $\beta^-$ decay with an 89.28\% branching ratio and via electron capture with a 10.72\% branching ratio. Its half-life is $T_{1/2} = 1.28 \times 10^{9}$ years. As with other primordial radionuclides, $^{40}$K may be present in detector materials such as the XeLS, balloon films, and KamLS. Due to the high level of purification achieved for the liquid scintillator, the contamination of $^{40}$K within the scintillator volumes is expected to be negligible. Consequently, the dominant contribution of $^{40}$K background is assumed to originate from the inner balloon film. A dedicated study was performed to estimate the magnitude of this film-related $^{40}$K contamination.

The most relevant feature of $^{40}$K decay for the excited-state analysis is the 1.46~MeV $\gamma$ ray emitted following electron capture. In the KamLS region outside the inner balloon, this monoenergetic peak is sufficiently prominent to be resolved above the continuous \2nbb background. 

To estimate the contribution of film-related $^{40}$K events reconstructed within the innermost XeLS volume, the radial distribution of events near the inner balloon boundary was examined. Spherical shells just inside and just outside the inner balloon were defined with equal volumes, and the energy spectrum in each shell was independently fitted using a simplified version of the full KamLAND-Zen spectral model. The purpose of these fits is to measure the relative strength of the $^{40}$K electron-capture peak as a function of radius. The resulting radial dependence of the fitted $^{40}$K rates is compared with the expected distribution obtained from KLG4Sim simulations. A least-squares ``fit of fits'' procedure is then used to extract the overall $^{40}$K activity associated with the inner balloon film.

An initial deviation from the expected radial distribution was observed; however, after vetoing the upper balloon neck region, excellent qualitative agreement between data and simulation was obtained. This result supports the conclusion that the $^{40}$K contamination is predominantly localized on the inner balloon film, with significantly lower contamination in the liquid scintillator volumes. This behavior is consistent with observations for the $^{238}$U and $^{232}$Th series.

\begin{equation}
    Y_{^{40}\mathrm{K},\,\mathrm{film}} = 183 \pm 13~\text{events/day}.
\end{equation}

This independently measured $^{40}$K rate is incorporated into the spectral fit as a penalty term.


\subsection{$^{85}$Kr}
$^{85}$Kr is a radioactive isotope released into the atmosphere primarily through the reprocessing of spent nuclear fuel. It undergoes $\beta^-$ decay with a half-life of $T_{1/2} = 10.76$ years. Despite its relatively short half-life, the atmospheric concentration of $^{85}$Kr has continued to increase due to ongoing nuclear fuel reprocessing. $^{85}$Kr is expected to be introduced into KamLAND-Zen during liquid scintillator purification and handling. Previous KamLAND and KamLAND-Zen 400 analyses observed a non-uniform distribution of $^{85}$Kr along the detector $z$-axis. In the present study, the $^{85}$Kr rate is allowed to float freely within the inner XeLS volume. $^{85}$Kr constitutes a significant background contribution in the low-energy region of the spectrum and is therefore an important component of the overall background model.

\section{Carbon Spallation}
A major class of backgrounds in both the \0nbb and \twonustar analyses arises from radioactive isotopes produced when high-energy cosmic-ray muons spallate carbon nuclei in the detector materials. In this work, spallation products are categorized according to their lifetimes. Isotopes with half-lives ranging from milliseconds to minutes are referred to as \emph{short-lived} spallation products, while isotopes produced primarily via xenon spallation with much longer half-lives are referred to as \emph{long-lived}. This section describes the modeling and estimation of backgrounds from short-lived carbon spallation products, as well as the dedicated treatment of the dominant spallation isotope, $^{11}$C.


\subsection{Short-Lived Spallation Products}
The short-lived carbon spallation products—primarily $^{6}$He, $^{8}$B, $^{8}$Li, $^{10}$C, and $^{12}$B—are effectively suppressed by the triple-coincidence veto techniques described in Chapter~\ref{chapter:reco_select}. Nevertheless, a small residual contribution remains due to imperfect tagging efficiency and detector deadtime. The residual background rate from these isotopes is estimated using a combined energy–$dT$ fit, where $dT$ is the time delay between the parent muon event and the subsequent radioactive decay. The event rate as a function of $dT$ can be expressed as: 
\begin{equation}
	\frac{dN}{dt}
	=
	\sum_i N_i \exp\!\left(-\frac{dT}{\tau_i}\right) + C,
\end{equation}

where $N_i$ is the normalization of isotope $i$, $\tau_i$ is its mean lifetime, and $C$ represents the rate of accidental, non-muon-correlated events. By simultaneously fitting the $dT$ distributions of multiple isotopes together with the accidental component, the production rates of the individual spallation products can be extracted. The fit is performed in the energy range 2–5~MeV, while the resulting full energy spectra are used in the excited-state spectral analysis.

 Figure~\ref{fig:shortspall_fit} shows the result of the combined energy–$dT$ fit. The fit is performed simultaneously over the three energy regions indicated in the figure. The residual background rates that survive the triple-coincidence veto are then estimated using the fitted production rates and the measured shower-veto selection efficiency. This estimation is performed using KamLS events. Any potential discrepancy between spallation production rates in KamLS and XeLS is included as a systematic uncertainty. The resulting short-lived spallation background rates are summarized in Table~\ref{tab:shortspall_tab}.

\begin{figure}[t!]
	\centering
	\includegraphics[scale=0.4]{shortspall_fit.png}
	\caption{The fit to short-lived spallation backgrounds over Energy and $dT$. While the fit for the spallation rates is performed in the energy range 2-5 MeV, the full expected energy distributions are used in the excited state analysis spectral fit. Figure taken from Reference~\cite{miyake_phd}.}
	\label{fig:shortspall_fit}
\end{figure}

\begin{table}[b!]
\centering
\footnotesize
\setlength{\tabcolsep}{11pt}  % was 3.5pt
\renewcommand{\arraystretch}{1.3}
\begin{tabular}{c c c c c c}
\hline
 & $E$ [MeV] & Lifetime $\tau$ & Prod. rate & Bkg. rate & Rej. eff. \\
 & & & [evt/d/kton] & [evt/d/kton] & [\%] \\
\hline

$^6$He   & 3.51 ($\beta^-$) & 1.16 s   & $12.36^{+1.22}_{-1.28}$ ($28 \pm 2$)   & $0.33^{+0.23}_{-0.02}$ & $97.6^{+1.7}_{-1.7}$ \\
$^{10}$C & 3.65 ($\beta^+$) & 27.8 s   & $18.70^{+0.72}_{-0.64}$ ($23 \pm 2$)   & $0.00^{+0.03}_{-0.00}$ & $100^{+0.00}_{-0.66}$ \\
$^8$Li   & 16.0 ($\beta^-$) & 1.21 s   & $25.77^{+0.92}_{-1.04}$ ($47 \pm 3$)   & $0.025^{+0.13}_{-0.14}$ & $99.1^{+0.5}_{-0.5}$ \\
$^{12}$B & 13.4 ($\beta^-$) & 29.1 ms  & $56.14^{+1.29}_{-1.28}$ ($42 \pm 3$)   & $0.015^{+0.002}_{-0.002}$ & $100.0^{+0.0}_{-0.0}$ \\
$^8$B    & 18.0 ($\beta^+$) & 1.11 s   & $0.58^{+0.71}_{-0.44}$ ($11 \pm 1$)     & $0.07^{+0.09}_{-0.07}$ & $88.3^{+14.8}_{-19.8}$ \\
$^{12}$N & 17.3 ($\beta^+$) & 15.9 ms  & $0.218^{+0.20}_{-0.12}$ ($0.74 \pm 0.06$) & $0^{+0}_{-0}$ & $100^{+0}_{-0.0}$ \\
$^9$C    & 16.5 ($\beta^+$) & 182.5 ms & $0.53^{+0.54}_{-0.44}$ ($1.5 \pm 0.1$) & $0^{+0.01}_{-0}$ & $100^{+0.0}_{-1.0}$ \\
$^8$He   & 10.7 ($\beta^-$) & 171.7 ms & $4.89^{+0.93}_{-0.86}$ ($0.55 \pm 0.04$) & $0^{+0.02}_{-0}$ & $100^{+0}_{-0.3}$ \\
$^9$Li   & 13.6 ($\beta^-$) & 257.2 ms & $0.00^{+1.19}_{-0}$ ($4.9 \pm 0.4$) & $0^{+0.01}_{-0}$ & - \\
$^{11}$Be& 11.5 ($\beta^-$) & 19.9 s   & $1.06^{+0.69}_{-0.21}$ ($1.1 \pm 0.1$)  & $0.00^{+0.26}_{-0}$ & $100^{+0.0}_{-22.1}$ \\
\hline
\end{tabular}
\caption{Production and background rate of carbon spallation products.}
\label{tab:shortspall_tab}
\end{table}



\subsection{$^{11}C$ Spallation Estimation}
Among all spallation products, $^{11}$C is by far the dominant isotope in KamLAND-Zen. FLUKA simulations predict a $^{11}$C production rate of $679 \pm 49$ events/day/kton, making it a clear outlier compared to other short- and long-lived isotopes. In addition, $^{11}$C is the dominant background in the medium-energy range (1–2~MeV) of the KamLAND-Zen spectrum, aside from \2nbb itself. As a result, an accurate determination of the $^{11}$C production rate is essential for constraining the \2nbb background normalization and for achieving sensitivity to excited-state decays.

The $^{11}$C production rate was previously measured in KamLAND using KamLS data and muon-coincidence techniques. In this analysis, the measurement is repeated in KamLAND-Zen 800 using XeLS data and MoGURA-based neutron coincidence tagging.

The $^{11}$C production rate is determined by selecting correlated $\mu$–$^{11}$C decay pairs. The following selection criteria are applied:

\begin{itemize}
	\item Muon delay: $100 < dT < 18{,}000$~s,
	\item Event quality cuts as described in Chapter~\ref{chapter:reco_select}, excluding the vertex badness veto to retain $^{11}$C orthopositronium decays,
	\item Removal of other spallation-related vetoes (long-lived veto, MoGURA neutron veto, and $^{137}$Xe veto),
	\item Spatial correlation: $dR < 80$~cm to the nearest neutron in the muon shower,
	\item Fiducial volume: $0 < r < 160$~cm,
	\item Energy selection: $1.0 < E_{\mathrm{vis}} < 1.6$~MeV.
\end{itemize}

\noindent The lower bound of $dT > 100$~s suppresses contributions from short-lived spallation isotopes. Applying these selections to the full KamLAND-Zen 800 dataset yields the $dT$ distributions shown in Figure~\ref{fig:c11_dt_fit} for both XeLS and KamLS.  The distributions are fitted with an exponential decay corresponding to the known $^{11}$C lifetime, $\tau = 1{,}764$~s, together with a linearly varying background to account for other long-lived spallation products with similar lifetimes.

\begin{figure}[t!]
    \centering
    \begin{subfigure}[b]{0.48\textwidth}
        \centering
        \includegraphics[width=\textwidth]{dT_distribution_allcuts_KamLS.png}
        \caption{KamLS}
        \label{fig:dt_dist_kamls}
    \end{subfigure}
    \hfill
    \begin{subfigure}[b]{0.48\textwidth}
        \centering
        \includegraphics[width=\textwidth]{dT_distribution_allcuts_XeLS.png}
        \caption{XeLS}
        \label{fig:dt_dist_xels}
    \end{subfigure}
    \caption{Distribution of $dT$ (time delay from muon to event) for muon-event pairs passing all selection cuts, shown separately for KamLS and XeLS.}
    \label{fig:c11_dt_fit}
\end{figure}

An independent cross-check of the $^{11}$C selection is performed using the reconstructed energy spectrum. The $\mu$–$^{11}$C candidate events are divided into on-time and off-time samples based on $dT$. The energy spectrum of the off-time events is subtracted from that of the on-time events to remove accidental background contributions. The resulting background-subtracted energy spectra are shown in Figure~\ref{fig:c11_energy}. In KamLS, the agreement between the data and the simulated $^{11}$C energy spectrum is clear. In XeLS, larger statistical fluctuations and residual background degrade the comparison. Nevertheless, the broad $^{11}$C spectral shape is clearly resolved. This cross-check confirms that the selected sample is dominated by genuine $^{11}$C decays.

\begin{figure}[t!]
    \centering
    \begin{subfigure}[b]{0.48\textwidth}
        \centering
        \includegraphics[width=\textwidth]{energy_distribution_difference_vs_MC_KamLS.png}
        \caption{KamLS}
        \label{fig:E_dist_kamls}
    \end{subfigure}
    \hfill
    \begin{subfigure}[b]{0.48\textwidth}
        \centering
        \includegraphics[width=\textwidth]{energy_distribution_difference_vs_MC_XeLS.png}
        \caption{XeLS}
        \label{fig:E_dist_xels}
    \end{subfigure}
    \caption{Subtracted energy distributions compared to MC $^{11}$C in XeLS and KamLS.}
    \label{fig:c11_energy}
\end{figure}

There are many spallation isotopes produced in Xenon-spallation with lifetimes of similar orders of magnitude $10^3-10^4$ seconds as $^{11}C$, $\tau=1,764$ seconds. Instead of estimating the relative production of each of these isotopes, the effect of the cumulative background is modeled with an uncosntrained linear slope to the background. 

\subsection*{$^{11}$C Rate Calculation}
The $^{11}$C production rate in XeLS is derived from the exponential component of the $dT$ fit. The expected number of detected, selected, and correlated $\mu$–$^{11}$C pairs is modeled as:
\begin{equation}
	I_{C11}
	=
	Y_{C11}
	\times E_{\mathrm{FBE}}
	\times (1 - dt_{\mathrm{MoG}})
	\times \epsilon_{dR}
	\times \epsilon_{dT}
	\times \epsilon_{\mathrm{FV}}
	\times \epsilon_{E},
\end{equation}
\noindent where the individual quantities are defined as follows:

\begin{itemize}
	\item $I_{C11} = 10{,}028$: integral of the exponential component of the fit (observed $\mu$–$^{11}$C pairs),
	\item $Y_{C11}$: $^{11}$C production rate (final result),
	\item $E_{\mathrm{FBE}} = 22.71$: XeLS exposure in kton$\cdot$days,
	\item $dt_{\mathrm{MoG}} = 1.88\%$: MoGURA deadtime fraction,
	\item $\epsilon_{dR} = 57\%$: spatial correlation efficiency ($dR < 80$~cm), obtained from FLUKA tuned with $^{11}$C data,
	\item $\epsilon_{dT} = 94.5\%$: efficiency of the $dT > 100$~s requirement,
	\item $\epsilon_{\mathrm{FV}} \times \epsilon_{E} = 79.7\%$: combined fiducial volume and energy selection efficiency from KLG4Sim.
\end{itemize}

\noindent Using these inputs, the $^{11}$C production rate in XeLS is calculated to be:
\begin{equation}
	Y_{C11} = 1{,}050 \pm 110~\text{events/(kton$\cdot$day)}.
\end{equation}

\noindent The uncertainty includes contributions from statistical error, exposure uncertainty, the choice of fit model, and neutron production modeling. The individual uncertainty components are summarized in Table~\ref{tab:c11_uncertainty}.


\begin{table}[b!]
    \centering
    \caption{Sources of uncertainty in the $^{11}$C production rate calculation.}
    \label{tab:c11_uncertainty}
    \begin{tabular}{lc}
        \hline
        Source & Uncertainty (\%) \\
        \hline
        Statistical & 5.4 \\
        Exposure uncertainty & 4.0 \\
        Exponential fit model choice & 7.0 \\
        Neutron production & 7.8 \\
        \hline
        Total uncertainty & 10.7 \\
        \hline
    \end{tabular}
\end{table}

\subsection{$^{137}$Xe}
Neutrons produced by cosmic-ray muon spallation can be captured on $^{136}$Xe nuclei, producing the radioactive isotope $^{137}$Xe. The $\beta^-$ decay of $^{137}$Xe has a $Q$-value of 4.16~MeV and a half-life of $T_{1/2} = 229$~s. Based on the neutron capture cross section, the $^{137}$Xe production rate is estimated to be 3.9 events/day/kton. The tagging efficiency of $^{137}$Xe using neutron coincidence techniques is estimated to be $74 \pm 7\%$. The long-lived spallation veto described in the following section also removes a fraction of $^{137}$Xe events. The efficiency of this veto is estimated from FLUKA simulations to be 42\%. After applying all vetoes, the residual $^{137}$Xe background rate in the singles dataset is estimated to be $0.43 \pm 0.36~\text{events/day/kton}$.


\begin{figure}[t!]
	\centering
	\includegraphics[scale=0.6]{xe137.png}
	\caption{The fit to short-lived spallation backgrounds over Energy and $dT$. While the fit for the spallation rates is performed in the energy range 2-5 MeV, the full expected energy distributions are used in the excited state analysis spectral fit. Figure taken from Reference~\cite{miyake_phd}.}
	\label{fig:xe137}
\end{figure}

\section{Long-Lived Xenon Spallation Products}
Long-lived spallation products originating from $^{136}$Xe constitute one of the most important background classes in KamLAND-Zen, particularly near the \0nbb region of interest (ROI). In the context of the excited-state analysis, these backgrounds are also relevant because they contribute non-negligible spectral structure at energies above the bulk of the \2nbb spectrum. If not modeled accurately, residual long-lived spallation backgrounds can bias the inferred \2nbb spectral shape near its endpoint and can mimic or distort weak signal features.

Compared to carbon spallation products (typically decay on millisecond--minute timescales) xenon spallation products often have half-lives of hours to days. This makes their identification by simple time-coincidence or box-cut methods substantially more difficult: the parent muon can be separated from the subsequent decay by timescales comparable to typical run lengths, and accidental coincidences dominate if one attempts direct muon--decay matching.

In the standard \0nbb analysis, KamLAND-Zen employs a likelihood-based tagging strategy to classify events into a lower-contamination ``Singles'' sample and a spallation-enriched ``Long-Lived'' sample. For the present excited-state analysis, this dataset separation is not applied. The primary motivation is to reduce systematic complexity in the modeling of the much larger $^{11}$C contribution in the excited-state energy region. Consequently, the long-lived xenon-spallation contribution must be estimated and constrained using a simulation-driven approach, as described below.

\subsection{FLUKA Simulation}
To estimate the production yields of xenon spallation products, cosmic-ray muons are simulated in a simplified geometry that captures the relevant target materials and shielding volumes. In the FLUKA environment, concentric spherical regions of XeLS ($0 < r < 192$~cm), KamLS ($192 < r < 650$~cm), and buffer oil (outer radius $\sim 9$~m) are modeled. The muon energy and angular distributions at the Kamioka site are taken from MUSIC simulations of the underground muon flux.

A total of $10^7$ muon events ($\mu^+$ and $\mu^-$) are generated, corresponding to an effective exposure of approximately 37 years of KamLAND observation. The muon charge ratio is set to $\mu^-/\mu^+ = 1.3$. The resulting FLUKA output provides the relative production yields of a large number of nuclides created by muon-induced spallation in XeLS. Figure~\ref{fig:fluka_prod} summarizes the production landscape as a function of $(Z,A)$ and highlights the dominant production regions.

\begin{figure}[t!]
    \centering
    \begin{subfigure}[b]{0.48\textwidth}
        \centering
        \includegraphics[width=\textwidth]{fluka_prod.png}
        \caption{Atomic mass number $A$ vs.\ atomic number $Z$.}
        \label{fig:fluka_prod}
    \end{subfigure}
    \hfill
    \begin{subfigure}[b]{0.48\textwidth}
        \centering
        \includegraphics[width=\textwidth]{fluka_mass.png}
        \caption{Production yield vs.\ atomic mass number $A$.}
        \label{fig:fluka_mass}
    \end{subfigure}

    \caption{Production rates of spallation nuclei in KamLAND-Zen XeLS. Two concentrations are visible: one near the primary heavy isotope ($^{136}$Xe), and another near lighter scintillator components (e.g., $A\sim 10$) such as carbon- and oxygen-related fragments. Figure adapted from Ref.~\cite{takeuchi_phd}.}
\end{figure}


\subsection{ENSDF Database}
FLUKA is used to model the muon-induced spallation production process, but it does not generate the full radioactive decay chains for the produced isotopes within the detector response framework used elsewhere in the analysis. Instead, FLUKA provides the \emph{relative production yields} of spallation products, and the subsequent radioactive decays are simulated within KLG4Sim using GEANT4, consistent with the treatment of other backgrounds and signals.

For each isotope produced in the FLUKA simulation, the half-life, decay modes, branching ratios, and daughter transitions are taken from the Evaluated Nuclear Structure Data File (ENSDF). The full decay chains are then propagated using the GEANT4 \texttt{RadioactiveDecay} process, and the resulting decay vertices and emitted particles are passed through the same detector-response simulation and reconstruction chain as the analysis data. This procedure yields both the expected \emph{visible-energy spectra} and the expected \emph{relative contributions} of long-lived xenon spallation products in the excited-state analysis window.

Table~\ref{tab:spall_bkgd} summarizes the dominant long-lived spallation contributions used in the fit. While FLUKA produces hundreds of isotopes in total, the subset listed in Table~\ref{tab:spall_bkgd} accounts for more than 95\% of the expected long-lived spallation background in the relevant energy range.

\subsection{Spectrum Distortion}
In the absence of a dedicated KamLAND-Zen measurement of xenon spallation yields and decay spectra induced by cosmic-ray muons, validating the absolute accuracy of the FLUKA-based long-lived spallation model is challenging. To quantify a realistic model uncertainty, external beam measurements are used as a cross-check on the FLUKA predictions.

Two relevant beam datasets are considered. First, a 490~GeV $\mu^+$ beam on gaseous xenon measured charged-hadron production yields, which can be reproduced by dedicated FLUKA benchmark simulations~\cite{mubeam_fluka}. Second, heavy-ion measurements of $^{136}$Xe incident on a liquid-hydrogen target were performed at 500~MeV/nucleon and at 1~GeV/nucleon~\cite{xenonbeam_500MeV,xenonbeam_1GeV}. These beam data provide a practical handle on the level of spectral disagreement that may arise from modeling uncertainties in hadronic fragmentation and secondary production.

Figure~\ref{fig:distortion} shows the relative difference between the FLUKA-derived long-lived spallation spectrum and the beam-measurement–inferred expectations. The discrepancy indicates a possible mismodeling of the long-lived spallation energy distribution, particularly in and near the \0nbb ROI. To propagate this uncertainty into the excited-state spectral fit, a distortion (shape-nuisance) parameter is introduced that deforms the nominal long-lived spallation spectrum according to the observed discrepancy. The larger deviation (red curve) is adopted as a conservative shape uncertainty, consistent with standard practice when external validation is limited.

\begin{figure}[t!]
	\centering
	\includegraphics[scale=0.35]{distortion.png}
	\caption{Comparison between FLUKA predictions and xenon beam data constraints. The red curve shows the difference inferred from Reference~\cite{xenonbeam_500MeV}, and the blue curve shows the difference from Reference~\cite{xenonbeam_1GeV}. The red curve is adopted as a conservative distortion uncertainty because it produces the larger deviation near the \0nbb ROI.}
	\label{fig:distortion}
\end{figure}


\section{Other Backgrounds}

\subsection{External Gamma Rays}
Gamma rays emitted by radioactive isotopes outside the active scintillator volumes can penetrate into the detector and deposit visible energy. The dominant contribution of this type arises from the 2.6~MeV gamma ray emitted in the decay of $^{208}$Tl, which is present primarily in the PMT glass and surrounding detector materials. Because these sources are external to the XeLS target, the resulting background exhibits a strong radial dependence, decreasing rapidly toward the detector center due to attenuation and geometric effects.

The energy and spatial distributions of external gamma-ray backgrounds are estimated directly from data rather than simulation. Events are selected in an energy window around the $^{208}$Tl gamma line, and their radial distribution is modeled as the sum of an exponentially falling component, representing external gamma rays entering the detector, and a second-degree polynomial component, representing radially uniform internal backgrounds.

Figure~\ref{fig:ext_gamma_radius} shows the reconstructed radial distribution of events with visible energies between 2.6 and 2.65~MeV. The fitted exponential component is extrapolated inward to estimate the residual external gamma-ray contribution in the fiducial volume. Using this method, the background rate within $r < 300$~cm is estimated to be $0.32 \pm 0.05$~events/day/kton for $z < 0$ and $0.69 \pm 0.12$~events/day/kton for $z > 0$. The corresponding modeled energy and radius distributions are shown in Figure~\ref{fig:ext_gamma_evis_radius}.

\begin{figure}[t!]
    \centering
    \begin{subfigure}[b]{0.47\textwidth}
        \centering
        \includegraphics[width=\textwidth]{ext_gamma_radius.png}
        \label{fig:ext_gamma_radius}
    \end{subfigure}
    \hfill
    \begin{subfigure}[b]{0.50\textwidth}
        \centering
        \includegraphics[width=\textwidth]{ext_gamma_evis_radius.png}
        \label{fig:ext_gamma_evis_radius}
    \end{subfigure}
    \caption{Characterization of external gamma-ray backgrounds in KamLAND-Zen. 
    (Left) Radial distribution of events in the 2.6--2.65~MeV energy window, fitted with the sum of an exponential component (external gamma rays) and a second-degree polynomial component (radially uniform backgrounds). 
    (Right) Modeled energy and radial distributions of external gamma-ray events. The 2.6~MeV $^{208}$Tl gamma peak appears near 2.8~MeV due to the higher light yield of the outer KamLS and the calibration of $E_{\mathrm{vis}}$ to XeLS events. Figures adapted from Reference~\cite{takeuchi_phd}.}
    \label{fig:external_gamma}
\end{figure}


Within the innermost fiducial volume used for the excited-state analysis ($r < 1.33$~m), the extrapolated contribution from external gamma rays is negligible. Consequently, this background component is not included explicitly in the spectral fit for the excited-state decay search.

\subsection{Solar Neutrinos}
Neutrino interactions constitute an irreducible background for searches for rare processes in KamLAND-Zen. Relevant neutrino sources include solar neutrinos, atmospheric neutrinos, geoneutrinos, and reactor antineutrinos. Among these, solar neutrinos dominate the interaction rate at visible energies below approximately 20~MeV and are therefore the most relevant for the present analysis.

Neutrinos interact in the detector primarily via elastic scattering (ES) on electrons and charged-current (CC) interactions on nuclei. In elastic scattering, a neutrino transfers a fraction of its energy to an electron, which subsequently deposits energy in the liquid scintillator. The resulting recoil electron spectrum reflects the incident neutrino energy distribution. Figure~\ref{fig:solar_nu} shows the expected solar neutrino fluxes as a function of energy. Of these, $^{8}$B solar neutrinos provide the dominant contribution near the \2nbb endpoint region. Other solar neutrino components occur at lower energies and are negligible compared to internal radioactive backgrounds; they are therefore neglected in this analysis.

\begin{figure}[t!]
	\centering
	\includegraphics[scale=0.45]{solar_nu.png}
	\caption{Energy spectra of solar neutrino fluxes. Figure adapted from~\cite{2nu_review}.}
	\label{fig:solar_nu}
\end{figure}


Charged-current interactions of solar neutrinos on xenon nuclei also contribute to the background. In particular, electron neutrinos can interact with $^{136}$Xe via:
\begin{equation}
	^{136}\mathrm{Xe} + \nu_e \rightarrow {}^{136}\mathrm{Cs} + e^- 
\end{equation}
The produced $^{136}$Cs subsequently decays to $^{136}$Ba with a half-life of $T_{1/2} = 13.01$~days and a decay $Q$-value of 2.55~MeV. The resulting beta decay produces a continuous visible-energy spectrum extending into the region relevant for excited-state decays.

Figure~\ref{fig:b8_cs136} shows the modeled visible-energy spectra of $^{8}$B solar neutrino elastic scattering and $^{136}$Cs decays. Both components are included explicitly in the spectral fit for the excited-state decay analysis.

\begin{figure}[t!]
	\centering
	\includegraphics[scale=0.5]{b8_cs136.png}
	\caption{Modeled solar neutrino backgrounds included in the excited-state analysis, including $^{8}$B elastic scattering and $^{136}$Cs beta decay. Figure adapted from~\cite{takeuchi_phd}.}
	\label{fig:b8_cs136}
\end{figure}

