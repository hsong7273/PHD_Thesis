\chapter{Conclusion}
\label{chapter:Conclusions}
\thispagestyle{myheadings}


\graphicspath{{8_Chapter_Conclusion/Figures/}}


This dissertation presents a preliminary search for the
$2\nu\beta\beta^\ast$ decay of $^{136}$Xe using data from the
KamLAND-Zen~800 experiment, which employs 745~kg of enriched xenon
dissolved in a liquid scintillator calorimeter. Building upon existing KamLAND-Zen~800 analyses, this work focuses on improving the treatment of previously weakly constrained low-energy background components and assessing their impact on the sensitivity to the
$2\nu\beta\beta^\ast$ signal.

In particular, dedicated studies were performed to constrain the rates of $^{11}$C produced by cosmic muon spallation and intrinsic $^{40}$K contamination in the inner balloon film. These backgrounds contribute in the low-energy region relevant for the excited-state decay search and were previously treated with limited external constraints. By introducing data-driven estimates and incorporating them into the spectral model, this analysis reduces degeneracies among background components and improves the robustness of the subsequent fitting procedure.

A spectral fit to the KamLAND-Zen~800 data was performed in a reduced
fiducial volume, and limits on the $2\nu\beta\beta^\ast$ decay rate were derived using a Feldman--Cousins construction. The resulting 90\% confidence level lower limit on the half-life, $T_{1/2} > 4.25 \times 10^{24}$~yr, represents a preliminary but potentially world-leading constraint. However, this limit is obtained under a simplified treatment of detector response and does not yet incorporate the full set of systematic uncertainties associated with
energy scale, non-linearity, and spatial response, and should therefore be interpreted as an optimistic estimate.

Future analyses of the KamLAND-Zen~800 dataset will enable a more
comprehensive and robust search for $2\nu\beta\beta^\ast$ decay.
Improved detector response modeling is expected to reduce systematic
uncertainties in the energy scale and non-linearity, which directly
affect the spectral shape of the dominant ground-state $2\nu\beta\beta$ background. Additionally, resolving residual
mismodeling of the radial distribution of backgrounds near and beyond
the inner balloon will permit an expansion of the fiducial volume and
enhanced rejection of film-originating background events, further
improving sensitivity.Future analyses of the KamLAND-Zen~800 dataset will enable a more comprehensive and robust search for $2\nu\beta\beta^\ast$ decay. Improved detector response modeling is expected to reduce systematic uncertainties in the energy scale and non-linearity, which directly affect the spectral shape of the dominant ground-state $2\nu\beta\beta$ background. In particular, the energy scale can be further constrained using spectral features from long-lived $^{136}$Xe spallation isotopes produced by cosmic muons in the xenon-loaded scintillator. These isotopes provide internally generated calibration handles that are spatially distributed throughout the XeLS volume and span the full duration of KamLAND-Zen~800 data-taking.

Tagged spallation events are identified using a likelihood-based
selection that exploits correlations between candidate decays and the
parent muon track, including the time since the muon ($\Delta T$),
distance to associated neutron captures ($dR$), and the effective
number of neutrons (ENN). This discriminator enables the isolation of
specific xenon spallation products whose decay spectra exhibit
distinctive peak or edge features. By extracting these features in
data and comparing them to Monte Carlo expectations, the detector
energy scale and its time dependence can be constrained independently
of traditional calibration sources.

Because these spallation-derived calibration points populate the same
energy region relevant for the $2\nu\beta\beta^\ast$ search and are
present continuously throughout the data-taking period, they provide a powerful means of validating and correcting residual mismodeling of the energy response. Incorporating this information into future analyses is expected to significantly reduce systematic distortions in the background model and improve sensitivity to excited-state double beta decay.

In addition to improvements in detector calibration and background
modeling, machine learning techniques provide a powerful and
complementary approach to enhancing sensitivity to
$2\nu\beta\beta^\ast$ decays. The KamNet framework, a
spatiotemporal deep neural network developed specifically for
KamLAND-Zen, has demonstrated strong discrimination power between
single-site energy depositions and closely spaced multivertex events
using low-level PMT hit timing and spatial information~\cite{KamNet}. This distinction is particularly relevant for excited-state double-beta decays, which are characterized by prompt $\gamma$ cascades following the primary $\beta\beta$ transition and therefore produce multiple, spatially separated energy deposits in the liquid scintillator.

KamNet exploits the spherical symmetry of the detector geometry through spherical convolutional layers, while simultaneously modeling the time structure of scintillation light using a convolutional LSTM with an attention mechanism. This architecture enables the network to focus on specific regions of the scintillation time profile that carry the most discriminating information. In the case of excited-state decays, KamNet assigns increased attention to time slices associated with delayed energy depositions from $\gamma$ interactions, effectively separating these events from the dominant ground-state $2\nu\beta\beta$ background, which is largely single-site and isotropic.

Using detailed Monte Carlo simulations, KamNet has been shown to improve the signal-to-background ratio for $^{136}$Xe excited-state decays by approximately 70\% relative to a traditional energy-only analysis, corresponding to a substantial gain in experimental sensitivity. Importantly, this improvement is achieved without introducing additional coincidence requirements or hardware upgrades, and the network's attention mechanism provides a degree of
interpretability by revealing which temporal and spatial features drive the classification. These results demonstrate that machine learning approaches such as KamNet are well suited for excited-state
double-beta decay searches and represent a promising avenue for future analyses of the KamLAND-Zen~800 dataset.

