\chapter{$2\nu\beta\beta^*$ Analysis}
\label{chapter:Analysis}
\thispagestyle{myheadings}

\graphicspath{{7_Chapter_Analysis/Figures/}}


This chapter describes the data analysis framework used to search for \twonustar. First the KamLAND-ZEN 800 full dataset is described, including any vetoed data-taking periods. This description is followed by an overview of the systematic uncertainties. Then, the energy spectral fitting procedure is outlined, including the definition of the chi-square metric. Finally, the statistical results are presented, culminating in a limit on the \twonustar rate.

\section{Xenon Enrichment in KamLAND-ZEN}
The amount of xenon gas dissolved into the XeLS is a normalization factor in the final xenon decay rates. The value is determined by subtracting off the xenon that remains after xenon installation:
\begin{enumerate}
	\item initial Xe mass : $769\pm1$ kg
	\item Xe left in LS in tanks and pipe lines : $21.5\pm2.8$ kg
	\item Xe left in storage bottles : $1\pm 1$ kg
	\item Xe trapped by charcoal filter or used for sampling : $1.5\pm 0.5$ kg
\end{enumerate}
From the above calculation, the installed Xenon gas is determined to be, $745\pm3$ kg. The composition of the enriched xenon is evaluated using a mass spectrometer. The measured values agree well with the values provided by the procurement company. The enrich xenon composition can be seen in Table \ref{tab:xenon_comp}.

\begin{table}[htbp]
    \centering
    \caption{Enriched Xenon Composition}
    \label{tab:xenon_comp}
    \begin{tabular}{lcccc}
        \hline
        & $^{136}$Xe & $^{134}$Xe & Others & Total \\
        \hline
        Provided ratio [\%] & 90.85 & 8.82 & 0.33 & 100.00 \\
        Measured ratio [\%] & $90.77 \pm 0.08$ & $8.96 \pm 0.02$ & -- & -- \\
        Atomic mass [u] & 135.907 & 133.905 & -- & -- \\
        Total mass [kg] & 677.39 & 64.83 & 2.79 & 745.0 \\
        \hline
    \end{tabular}
\end{table}


\section{Full KamLAND-ZEN 800 Dataset}
The dataset used in this analysis was taken between February 5, 2019 and April 30, 2023, run range : 15431-18691. Unlike the \0nbb analysis, the dataset is not divided by Long-lived spallation background likelihood into a singles and long-lived dataset. Instead all the events are combined into a single energy spectrum for joint fitting.


\subsection{Vetoed Data Periods}
In addition to the regular deadtime due to maintenance, run quality, described in Chapter \ref{chapter:reco_select}, there are additional vetoed data periods specifically for this analysis. 
\subsection*{Electric Power Supply Instability}
The run range 16790-16874 are excluded from this analysis because the DAQ was unstable due to AVR (automatic voltage regulator) trouble. The constant restarting of the DAQ leads to short runs which make it difficult to perform the run-by-run calibration described in Chapter \ref{chapter:calibration}. 
\subsection*{MoGURA disorder period}
The MoGURA DAQ system was unstable between September-November 2022 and raw data files were corrupted. While the KamDAQ files are readable, the MoGDAQ data is crucial for constructing events near after cosmic ray muons. These events are a key tag for cosmic related spallation backgrounds. Thus, runs 17768-17905 are excluded from this analysis.

\section{Systematic Uncertainties}
While the \0nbb and \twonustar analyses are statistical uncertainty dominated, this section discusses estimates of several sources of systematic error.

The uncertainty in the total amount of xenon dissolved in the XeLS was estimated in an internal study, the estimated uncertainty is 0.4\%. 

As for the xenon enrichment factor, the 0.1\% difference in the supplier stated enrichment and our measuremed enrichment, described in the previous section, is used as the systematic uncertainty in xenon enrichment.

The detector energy scale varies over time as electronics fail, are repaired, and PMTs degrade. Using the neutron capture gamma peak at 2.2 MeV, a maximum error of 0.9\% was determined.

Finally, the uncertainty in the fiducial volume is the uncertainty in the true volume of events encapsulated by the fiducial volume selection of 157.49 cm. This is determined using the early KLZ-800 data, just after xenon dissolving work. As the xenon is introduced, $^{222}$Rn is incidentally introduced from the atmosphere, $^{222}$Rn has a half-life of 3.8 days, and its decay is soon followed by a $^{214}$Bi-Po sequential decays. Data for a month after xenon is introduced to the XeLS is used and the Bi-Po coincident events are analyzed. The volume ratio between spheres of 157.49cm and 192cm, the inner balloon radius, is 0.5686. While the ratio of Bi-Po events observed in these spherical regions is 0.5454. Since it is expected that the $^{222}$Rn is distributed uniformly throughout the XeLS, a 4.1\% difference is taken as the fiducial volume uncertainty.
\section{Spectral Fit}
The \twonustar decay rate is estimated by fitting background and signal models to the energy distribution of reconstructed KLZ-800 data. Namely, the energy distribution of events within the reduced FV of $(r<1.33m)$ and that pass the event selections discussed in Chapter \ref{chapter:reco_select}.
\subsection{Chi-Square Definition}
In this study, a binned chi-square, maximum likelihood fit is performed. The chi-square has multiple components, a energy-bin term and penalty terms.
\begin{equation}
    \chi^2=(\sum_{energy} \chi^2_{energy}) +\chi^2_{penalty}
\end{equation}
Here, $\sum_{energy}$, denotes a sum over each 0.05 MeV energy bin from the range 0.5-4.8 MeV. In each energy bin, the $\chi^2$ is computed.
\begin{equation}
\chi^2_{\text{energy}} =
\begin{cases}
    2 \sum_{i} \left( \nu_{i} - n_{i} + n_{i} \log \frac{n_{i}}{\nu_{i}} \right) & (n_{i} > 0) \\
    2 \sum_{i} (\nu_{i} - n_{i}) & (n_{i} = 0)
\end{cases}
\end{equation}
Now, $\nu_i$ is the model expected energy spectrum for the given fit parameters, and $n_i$ denotes the observed number of events in the $i$-th bin. The penalty terms constrains certain parameters that have independent constraints. The fit parameter configuration is summarized in Table \ref{tab:parameters}, and described in more detail in the later sections. The penalty $\chi^2$ terms are simply defined as:
\begin{equation}
    \chi^2_{penalty}=\sum_i\left(\frac{O_n-E_n}{\sigma_n}\right)^2
\end{equation}
where $O_n$ is the estimated parameter value, $E_n$ are the central expected values, and $\sigma_n$ are the expected parameter errors. 

\subsection{Minimizer}
The ROOT implementation of the MINUIT package distributed by CERN is the minimization package used for this analysis. 

\subsection{Fit Parameters}
The spectral rate parameters are divided by origin volume. This analysis is only concerned with backgrounds originating in the XeLS and the inner balloon film. Each of the background described in Chapter \ref{chapter:backgrounds} is implemented in the spectral fit. 
\subsection{Penalty Terms}

\begin{table}[htbp]
    \centering
    \caption{Fit parameter configuration for the spectral analysis. The fit condition column indicates whether the parameter is free, fixed, scanned, or constrained in the fit.}
    \label{tab:parameters}
    \begin{tabular}{lll}
        \hline
        Material & Parameter & Fit Condition \\
        \hline
        \multirow{12}{*}{XeLS}
            & $^{136}$Xe 2$\nu\beta\beta^*$ & scan \\
            & $^{136}$Xe 2$\nu\beta\beta$ & free \\
            & $^{238}$U series 2 & constrain \\
            & $^{222}$Rn & constrain \\
            & $^{232}$Th series 2 & constrain \\
            & $^{210}$Bi & free \\
            & $^{85}$Kr & free \\
            & $^{11}$C & constrain \\
            & $^{137}$Xe & constrain \\
            & Xe spallation & free \\
            & solar $\nu$ ES + CC & fix \\
            & $^{136}$Cs & constrain \\
        \hline
        \multirow{7}{*}{Film}
            & $^{238}$U series 1 & fix \\
            & $^{238}$U series 2 & free \\
            & $^{232}$Th series 1 & free \\
            & $^{232}$Th series 2 & free \\
            & $^{40}$K & constrain \\
            & $^{210}$Bi & free \\
            & $^{11}$C & free \\
        \hline
        \multirow{7}{*}{common}
            & Energy scale & constrained \\
            & $k_B, R$ & constrained \\
            & LL-distortion & constrain \\
            & $^6$He & fix \\
            & $^{12}$B & fix \\
            & $^8$Li & fix \\
            & $^8$B & fix \\
        \hline
    \end{tabular}
\end{table}

% \section{\twonustar  Results}
% \subsection{Sensitivity}
% \section{Discussion of Results}

% \section{Future Prospects}
% \subsection{Improved Detector Response Calibration}
% \subsection{Improved Event Selection}
% \subsection{KamLAND2-ZEN}
Time to get philosophical and wordy.\cite{takeuchi_phd}

