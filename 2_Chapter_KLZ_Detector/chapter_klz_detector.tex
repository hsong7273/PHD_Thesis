\chapter{The KamLAND-ZEN Experiment}
\label{chapter:klz-detector}
\thispagestyle{myheadings}
\graphicspath{{2_Chapter_KLZ_Detector/Figures/}}

KamLAND, the \textbf{Kam}ioka \textbf{L}iquid-scintillator \textbf{A}nti Neutrino \textbf{D}etector, is a large liquid scintillator calorimeter detector situated 1km below mt. Ikenoyama in Gifu prefecture, Japan. I will describe the KamLAND detector's and the corresponding KamLAND experimental area's important components and features in this chapter. I will also explain how each component contributes to the KamLAND's scientific goals and the work of this thesis.\\


\section{KamLAND}
\label{sec:KamLAND}
One can think of KamLAND as an onion made up of many spherical layers, each layer serving the ultimate goal of shielding and observing the central core, the xenon-loaded liquid scintillator.


\subsection{Detector Infrastructure}
The KamLAND detector is surrounded by the KamLAND experimental area, situated in an old iron mine, multiple caverns and passageways were excavated and set aside for KamLAND experimental use. \\

The KamLAND site is shown in Figure *. The control room contains networking and monitoring equipment which on-site shifters use to observe real-time detector activity. The first LS purification areas contain liquid-liquid extraction and nitrogen purge purification systems. The second LS purification area contains a distillation purification system. A new Xenon purification
area was built for KamLAND-Zen. The dome area is a class 1,000 clean area atop
the detector and includes a calibration source preparation room and electronics enclosure (electronics hut or e-hut). At the center of the dome area, there is a secondary class 100-1000 clean tent covering the KamLAND chimney. The inner balloon installations took place in August 2016 and May 2018 inside this clean tent.\\

The outer detector (OD) is a cylindrical water tank 20m tall and with 20m diameter and filled with pure water. The OD was refurbished in 2016, and 140 new 20-inch PMTs (R3600) were installed inside the cavity. The inner wall of the outer tank and the outer surface of the inner detector stainless steel spherical tank are covered highly reflective Tyvek sheets (Tyvek 1073B and 1082D) to collect as much of the light generated by crossing cosmic ray muons as possible. The outer detector's role is to tag cosmic ray muons, shield radioactivity and fast neutrons from the outer rock, and to stabilize the temperature of the ID.

\subsection{Inner Detector}
KamLAND's inner detector (ID) is the main spherical liquid scintillator detector, it is shown in Figure *. The ID is contained in a 18m diameter stainless steel sphere tank. 1,879 PMTs are mounted onto the inner wall of the ID, 1,325 17-inch and 554 20-inch PMTs. The PMTs are submerged in non-scintillating buffer oil (BO). An acrylic panel separates the buffer layer into two shells. This panel prevents the convection of radon out-gassed from PMT glasses into the central parts of the detector.\\

Next, is the 13m diameter outer balloon (OB). The OB is suspended in the center of the ID within the buffer oil, it is filled with one kiloton of highly purified organic liquid scintillator.

\subsection{Liquid Scintillator}
Liquid scintillator is the vital medium that sensitizes KamLAND to internal radioactivity. 




Here goes all the important stuff, likely with a lot of graphics like this Figure~\ref{fig:sampling} below.

In all likelihood, you will need to insert tables, like Table~\ref{tbl:parameters} on the next page.
\clearpage

\begin{table}[h]
	\caption{Absolute disparity error per pixel for the test data from
		Fig.~\ref{fig:sampling} and different parameter values. In each experiment one
		parameter is adjusted while other parameters are unchanged.} 
	\centering
	\renewcommand{\arraystretch}{1.2}
	\begin{minipage}[b]{0.30\linewidth}
		\centerline{$\eta=6000$, $\mu=2000$}\smallskip
		\centering
		\begin{tabular}{ccc}
			\hline
			$K$ & $u_1$ & $u_2$\\
			\hline
			3   & 0.52 &0.46\\
			7   & 0.47 &0.43\\
			10  & 0.35 &0.36\\
			12  & 0.37 &0.36\\
			\hline
		\end{tabular}
	\end{minipage}
	%
	\begin{minipage}[b]{0.34\linewidth}
		\centerline{$K=10$, $\mu=2000$}\smallskip
		\centering
		\begin{tabular}{ccc}
			\hline
			$\eta$ & $u_1$ & $u_2$\\
			\hline
			1000&0.54& 0.45\\
			3000&0.43& 0.40\\
			6000&0.35& 0.36\\
			9000&0.37& 0.37\\
			\hline
		\end{tabular}
	\end{minipage}
	%
	\begin{minipage}[b]{0.32\linewidth}
		\centerline{$K=10$, $\eta=6000$}\smallskip
		\centering
		\begin{tabular}{ccc}
			\hline
			$\mu$ & $u_1$ & $u_2$\\
			\hline
			100 &1.00&1.16\\
			1000&0.53&0.47\\
			2000&0.35&0.36\\
			3000&0.44&0.43\\
			\hline
		\end{tabular}
	\end{minipage}
	%
	\label{tbl:parameters}
\end{table}

Of course, there must be a Table of Contents, List of Figures and List of Tables at the beginning of the thesis, but this is all set up automatically.

{\bf Important}: You will also be using a lot of citations. The format in this template follows the so-called APA style and looks as follows in the document body: \cite{lamport1985:latex}, \cite{Debr01}. There are no numbers in the list of references -- the list is sorted alphabetically according to the first author's last name.

Other styles of references are allowed by the library as well, e.g., ``plain'' or ``'ieee'', which use numbers in square brackets both in the document body and in the list of references. In order to use another style of references, e.g., ``plain'', follow the steps below:
%
\begin{enumerate}
  \item In ``thesis.tex'' file:
	\begin{itemize}
	  \item comment out the line ``$\backslash$usepackage\{apalike\}'' at the top of the file,
	  \item replace ``$\backslash$bibliographystyle\{apalike\}'' with ``$\backslash$bibliographystyle\{plain\}'' towards the bottom of the file.
	\end{itemize}
  \item In ``bu\_ece\_thesis.tex'' file, comment out all lines in the BIBLIOGRPAHY section (lines 503-517) and save it!
  \item Recompile ``thesis.tex'' twice
\end{enumerate}

