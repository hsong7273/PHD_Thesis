\chapter{The KamLAND-Zen Experiment}
\label{chapter:klz-detector}
\thispagestyle{myheadings}
\graphicspath{{3_Chapter_KLZ_Detector/Figures/}}

KamLAND, the \textbf{Kam}ioka \textbf{L}iquid-scintillator \textbf{A}nti-\textbf{N}eutrino \textbf{D}etector, is a large liquid scintillator calorimeter located approximately 1\,km underground beneath Mt.\ Ikenoyama in the Kamioka mine, Gifu Prefecture, Japan. In this chapter, the KamLAND detector and the surrounding experimental infrastructure are described, with emphasis on the modifications introduced for the KamLAND-Zen experiment. The roles of the detector subsystems are discussed in the context of the scientific goals of KamLAND-Zen and the analyses presented in this thesis.

\section{KamLAND}
\label{sec:KamLAND}
The KamLAND detector can be viewed as a set of concentric spherical layers, each designed to shield and observe the innermost target volume. At the center of the detector resides the xenon-loaded liquid scintillator used by KamLAND-Zen for double-beta decay searches. The KamLAND detector is housed within a dedicated experimental area excavated in the Kamioka mine. Multiple caverns and passageways were constructed to accommodate the detector, purification systems, calibration facilities, and control electronics.

The KamLAND site is shown in Figure~\ref{fig:kamlandsite}. The control room contains the networking, data acquisition, and monitoring equipment used by on-site shifters to oversee detector operation in real time. The first liquid scintillator (LS) purification area houses liquid--liquid extraction systems and nitrogen purging equipment, while the second purification area contains a large-scale distillation system. A dedicated xenon purification and handling area was constructed for KamLAND-Zen.

\begin{figure}[t!]
	\centering
	\includegraphics[scale=0.4]{KamLAND_site.png}
	\caption{Illustration of the KamLAND site.}
	\label{fig:kamlandsite}
\end{figure}

Above the detector, the dome area is maintained as a class-1000 clean environment and includes calibration source preparation facilities and the electronics hut (E-hut). At the center of the dome area, a secondary clean tent (class 100--1000) encloses the KamLAND chimney. All inner balloon installation and replacement operations were performed within this clean environment, most notably in August 2016 and May 2018.

The outer detector (OD) consists of a cylindrical water tank 20\,m in height and 20\,m in diameter, filled with ultra-pure water. The OD was refurbished in 2016, during which 140 new 20-inch Hamamatsu R3600 photomultiplier tubes (PMTs) were installed. The inner surface of the water tank and the exterior of the inner detector stainless-steel sphere are lined with highly reflective Tyvek sheets (Tyvek 1073B and 1082D) to enhance light collection from cosmic-ray muons.  The OD serves three primary functions: tagging cosmic-ray muons, shielding the inner detector from radioactivity and fast neutrons originating in the surrounding rock, and stabilizing the thermal environment of the inner detector.

The inner detector (ID), shown in Figure~\ref{fig:kamland}, is the primary scintillation detector of KamLAND. It consists of an 18\,m-diameter stainless-steel spherical tank instrumented with 1,879 inward-facing PMTs: 1,325 17-inch PMTs and 554 20-inch PMTs. The PMTs are submerged in non-scintillating buffer oil (BO), which optically isolates the central scintillator volume from PMT-related radioactivity. An acrylic barrier divides the buffer oil into inner and outer regions, suppressing convective transport of radon emanating from PMT glass toward the detector center.

\begin{figure}[t!]
	\centering
	\includegraphics[scale=0.4]{kamland.png}
	\caption{KamLAND-Zen detector}
	\label{fig:kamland}
\end{figure}

Photomultiplier tubes (PMTs) act as the photosensors of KamLAND, detecting scintillation photons produced by charged particles traversing the liquid scintillator. Incident photons striking the photocathode liberate photoelectrons, which are subsequently multiplied through a series of dynodes, producing an output charge of approximately $10^{6}$--$10^{7}$ electrons per photoelectron.  Should multiple photons hit the photocathode simultaneously, the output voltage increases proportionally. The resulting current pulse is transmitted via long coaxial cables to the front-end electronics. A schematic of the 17-inch and 20-inch PMTs is shown in Figure~\ref{fig:pmts}.

\begin{figure}[t!]
	\centering
	\includegraphics[scale=0.35]{pmts.png}
	\caption{Schematics of the 17-inch and 20-inch PMTs.}
	\label{fig:pmts}
\end{figure}

The 17-inch PMTs are Hamamatsu R7250 devices, while the 20-inch PMTs consist of Hamamatsu R1449 and R3600 models originally deployed in the Kamiokande experiment. All PMTs employ bialkali photocathodes sensitive to wavelengths between approximately 300 and 650\,nm, well matched to the emission spectrum of the liquid scintillator. The 17-inch PMTs utilize a ``box-and-line'' dynode structure, whereas the 20-inch PMTs employ a ``venetian-blind'' design. As a result, the 17-inch PMTs exhibit superior transit-time spread, while the 20-inch PMTs provide higher photon collection efficiency.  It should also be noted that the 17-inch and 20-inch PMTs both have the same physical footprint, where the 17-inch PMTs have a 20-inch-diameter-photocathode that is `masked' down to a 17-inch diameter.  This masking improves the overall timing resolution. The total photocathode coverage of the inner detector is approximately 34\%, with 23\% contributed by the 17-inch PMTs.

The performance of large-area PMTs is sensitive to external magnetic fields. To mitigate distortions caused by the Earth's magnetic field, KamLAND is equipped with a system of geomagnetic compensation coils. These coils reduce the residual magnetic field inside the detector volume to below 50\,mG, rendering its effect on PMT performance negligible. An additional key parameter governing PMT performance is the quantum efficiency (QE), defined as the probability that an incident photon produces a photoelectron. The QE depends strongly on photon wavelength. To optimize light collection, the KamLAND liquid scintillator is doped with the wavelength shifter 2,5-diphenyloxazole (PPO), which shifts scintillation photons into the spectral region where PMT QE is maximal. Figure~\ref{fig:qe_emission} compares the PMT QE curve with the PPO emission spectrum.

Another important characteristic of PMTs is their quantum efficiency (QE). The QE quantifies the probability that a photon arriving on the photocathode will produce a photoelectron. A PMT's QE varies over the wavelength of the incoming light. To improve our light collection, KamLAND's LS is doped with PPO to shift the wavelength of the incoming light to where the PMTs are most sensitive. Figure ~\ref{fig:qe_emission} shows the PMT QE curve and the PPO reemission spectrum.

\begin{figure}[t!]
	\centering
	\includegraphics[scale=0.35]{qe_PPO_emission.png}
	\caption{Quantum Efficiency of the KamLAND inner PMTs and PPO emission over wavelength. Figure taken from \cite{mastuda_phd}.}
	\label{fig:qe_emission}
\end{figure}

Finally, the 13\,m-diameter outer balloon (OB) is suspended at the center of the inner detector and immersed in the buffer oil. The OB contains approximately one kiloton of highly purified organic liquid scintillator, which serves as the primary active target volume of the KamLAND experiment. In the KamLAND-Zen configuration, this volume also provides the immediate shielding and detection medium surrounding the xenon-loaded inner balloon.

\subsection{Liquid Scintillator Targets}

Liquid scintillator (LS) serves as the active detection medium of KamLAND and is essential for sensitizing the detector to low-energy radioactive processes. Charged particles traversing the LS excite molecular states that subsequently de-excite through the emission of scintillation photons. These photons are detected by the surrounding photomultiplier tubes, enabling precise reconstruction of event energy, position, and timing.

The KamLAND liquid scintillator (KamLS), which fills the volume between the outer balloon and the inner balloon, is composed of a carefully optimized mixture of organic compounds designed to maximize light yield, optical transparency, chemical stability, and radiopurity. The scintillator consists of 80.2\% dodecane (D12) and 19.8\% pseudocumene (PC), with the wavelength shifter 2,5-diphenyloxazole (PPO) added at a concentration of $1.36 \pm 0.03$\,g/L. Pseudocumene acts as the primary scintillating solvent, while dodecane serves as a diluent that improves chemical stability and suppresses radioactive impurities.

Scintillation light produced in pseudocumene peaks in the ultraviolet region. PPO absorbs this ultraviolet light and re-emits it at longer wavelengths, typically around 350--380\,nm, where the quantum efficiency of the KamLAND PMTs is highest. This wavelength-shifting process significantly enhances photon detection efficiency and improves the detector’s energy resolution.

Achieving extremely low levels of radioactive contamination in the KamLS is critical for the success of KamLAND-Zen, as trace impurities from natural decay chains can mimic or obscure rare double-beta decay signals. Through extensive purification procedures, including water extraction, distillation, and nitrogen purging, KamLAND-Zen has achieved contamination levels of approximately $5\times10^{-18}$\,g/g for $^{238}$U and $1.3\times10^{-17}$\,g/g for $^{232}$Th. These values represent some of the lowest radioactive impurity concentrations ever realized in a large-scale liquid scintillator detector.

The chemical composition and relevant physical properties of the KamLAND liquid scintillator components are summarized in Table~\ref{tbl:kamls}. The densities, boiling points, and flash points of the constituent materials are particularly important for detector safety, thermal stability, and purification system design.

\begin{table}[b!]
	\centering
	\renewcommand{\arraystretch}{1.2}
	\begin{tabular}{c|ccc}
		\hline
		& D12 & PC & PPO \\
		\hline
		Chemical Formula & C$_{12}$H$_{26}$ & C$_9$H$_{12}$ & C$_{15}$H$_{11}$NO \\
		Density [$g/cm^3$] & 0.7526 & 0.8796 & -\\
		Boiling Point [$^\circ$C] & 216 & 169 & 360 \\
		Melting Point [$^\circ$C] & -10 & -44 & 72 \\
		Flash Point [$^\circ$C] & 83 & 54 & - \\ \hline
	\end{tabular}
	\caption{Composition and physical properties of the KamLAND liquid scintillator (KamLS).}
	\label{tbl:kamls}
\end{table}

\subsection{KamLAND-Zen and XeLS}

The KamLAND-Zen experiment modifies the central region of the KamLAND detector by introducing xenon-loaded liquid scintillator (XeLS) contained within a transparent inner balloon (IB) of radius approximately 1.9\,m. This configuration places the double-beta decay isotope $^{136}$Xe at the most radiopure and well-shielded location in the detector, maximizing sensitivity to neutrinoless double-beta decay.

The xenon gas is enriched to approximately 90\% in $^{136}$Xe and is dissolved into a modified version of the KamLAND liquid scintillator. The addition of xenon reduces the scintillation light yield by roughly 10\%. To compensate for this effect, the PPO concentration in the XeLS is increased to approximately 4\,g/L, restoring the overall light output and preserving energy resolution. The XeLS density is carefully adjusted to closely match that of the surrounding KamLS, thereby minimizing buoyant forces and mechanical stress on the inner balloon.

The precise chemical composition of the XeLS has evolved over different phases of the KamLAND-Zen experiment, reflecting changes in xenon loading and optimization of detector performance. Table~\ref{tbl:xels} summarizes the relative concentrations of decane, pseudocumene, PPO, and xenon for KamLAND-Zen 400 Phase-1, Phase-2, and KamLAND-Zen 800. The increased xenon concentration in the KamLAND-Zen 800 phase significantly enhances the available source mass for double-beta decay while maintaining acceptable optical and mechanical properties.

\begin{table}[b!]
	\centering
	\renewcommand{\arraystretch}{1.2}
	\begin{tabular}{c|cccc}
		\hline
		Material & Decane (\%) & PC (\%) & PPO (\%) & Xe (\%)\\ \hline
		Zen 400 Phase-1 & 82.3 & 17.7 & 2.7 & 2.44/2.48\\
		Zen 400 Phase-2 & 80.7 & 19.3 & 2.29$\pm$0.03 & 2.91\\
		Zen 800 & 82.4 & 17.6 & 2.38$\pm$0.02 & 3.13\\ \hline
	\end{tabular}
	\caption{Chemical composition of the xenon-loaded liquid scintillator (XeLS) for the different phases of the KamLAND-Zen experiment.}
	\label{tbl:xels}
\end{table}



\section{Chemical Handling Infrastructure}

Background mitigation is a central requirement for a successful search for neutrinoless double-beta decay (\0nbb). In KamLAND-Zen, maintaining ultra-high chemical purity of all liquid volumes is essential, as trace radioactive contaminants can produce backgrounds in the energy region of interest or generate long-lived daughter isotopes that are difficult to remove through event selection alone. In addition, gaseous impurities such as radon can diffuse into the detector and introduce time-dependent backgrounds.

To address these challenges, KamLAND and KamLAND-Zen employ a comprehensive chemical handling and purification infrastructure designed to remove radioactive contaminants, suppress radon ingress, and preserve the optical properties of the liquid scintillator and xenon-loaded liquid scintillator. In this section, the major purification and handling systems used for the liquid scintillator (LS), buffer oil, and XeLS are described.


\subsection{Water Extraction}

The first stage of purification for both the liquid scintillator and buffer oil is performed using a water extraction system, shown schematically in Figure~\ref{fig:waterex}. Prior to entering the extraction column, the liquids are filtered in two stages using filters with pore sizes of 1\,$\mu$m and 0.1\,$\mu$m to remove particulate contamination introduced during handling, circulation, or detector operations.

\begin{figure}[t!]
	\centering
	\includegraphics[scale=0.5]{3_Chapter_KLZ_Detector/Figures/waterextraction.png}
	\caption{Flow diagram of the water extraction and nitrogen purge system. Figure from Reference~\cite{gando_phd}.}
	\label{fig:waterex}
\end{figure}

Following filtration, the liquids are passed through a counter-flow water extraction tower, where they are brought into contact with ultra-pure water. Many metallic impurities, including uranium, thorium, potassium, and their daughter isotopes, exhibit higher solubility in water than in the organic scintillator and preferentially migrate into the aqueous phase. This process effectively removes a broad class of radioactive contaminants that cannot be eliminated through filtration alone.

After water extraction, the liquids are subjected to nitrogen purging using ultra-high-purity nitrogen gas. This step removes dissolved gases such as radon, oxygen, and other electronegative species. The removal of radon is particularly critical, as radon progeny can plate out onto detector surfaces or remain suspended in the scintillator, producing backgrounds in the $0\nu\beta\beta$ energy region. Oxygen removal also improves scintillation light yield and long-term optical stability.


\subsection{Distillation}

Further purification of the liquid scintillator is achieved using a dedicated distillation system, illustrated in Figure~\ref{fig:distillation}. Distillation provides one of the most powerful methods for removing both radioactive contaminants and optically active impurities, taking advantage of differences in boiling points between scintillator components and contaminants.

\begin{figure}[b!]
	\centering
	\includegraphics[scale=0.5]{3_Chapter_KLZ_Detector/Figures/distillation.png}
	\caption{Flow diagram of the KLZ distillation and circulation system. Figure from Reference~\cite{ozaki_phd}.}
	\label{fig:distillation}
\end{figure}

Liquid scintillator from KamLAND is continuously circulated through the distillation system during purification campaigns. In the distillation column, the scintillator mixture is heated and separated into its primary chemical components, most notably pseudocumene (PC) and PPO. Each component is distilled independently under controlled conditions, allowing heavy metal contaminants, oxidation products, and other non-volatile impurities to be efficiently separated and discarded.

After distillation, the purified components are recombined in a dedicated mixing tank to reproduce the original KamLAND liquid scintillator composition. The relative concentrations of PC, dodecane, and PPO are controlled with high precision, achieving density reproducibility at the level of $10^{-3}$\,g/cm$^3$. Maintaining precise density control is important both for detector stability and for matching the density of adjacent liquid volumes.

As in the water extraction stage, the recombined scintillator is purged with high-purity nitrogen gas prior to reintroduction into the detector. This final purge eliminates residual dissolved gases and minimizes the introduction of airborne radon during transfer. The distillation system has played a crucial role in achieving and maintaining the ultra-low background levels required for KamLAND-Zen.


\subsection{Xenon Handling}

The handling of xenon-loaded liquid scintillator (XeLS) requires a specialized system capable of safely extracting, purifying, storing, and reintroducing xenon while preserving scintillator purity and maintaining precise control over chemical composition. A schematic overview of the XeLS handling system is shown in Figure~\ref{fig:xenonhandling}.

The xenon handling system consists of several interconnected tanks and control components, each designed to perform a specific function in the xenon extraction and loading process:

\begin{itemize}
	\item \textbf{Main Tank (1.1\,m$^3$):} The main tank is directly connected to the KamLAND-Zen inner balloon. During xenon extraction, XeLS is transferred from the inner balloon into this tank, which serves as the initial collection and staging point for subsequent processing.

	\item \textbf{Reservoir Tank (1.1\,m$^3$):} The reservoir tank is connected to the main tank via a vacuum pump and a liquid scintillator trap. The tank is cooled to approximately $-50^\circ$C using liquid nitrogen, causing organic scintillator vapors to condense and be trapped while allowing xenon gas to pass through. This step effectively separates xenon from the scintillator without introducing additional chemical contaminants.

	\item \textbf{Storage Tank (25\,m$^3$):} The degassed liquid scintillator remaining after xenon extraction is transferred to a large storage tank for temporary containment. This tank allows the scintillator to be stored safely while xenon purification or system maintenance is performed.

	\item \textbf{Sub-Tank (1.1\,m$^3$):} The sub-tank serves as the primary mixing volume for re-dissolving purified xenon gas into liquid scintillator. Xenon is introduced into the scintillator under controlled conditions to ensure uniform dissolution. The density and composition of the XeLS are continuously monitored during this process.

	\item \textbf{Control Tank (1.1\,m$^3$):} The control tank is connected to the sub-tank and the second purification area and is used to fine-tune the chemical composition of the XeLS. By adjusting the relative concentration of dodecane, the control tank enables precise control of the XeLS density, ensuring compatibility with the surrounding KamLAND liquid scintillator. The tank is pressurized with high-purity nitrogen gas to prevent air ingress and radon contamination.
\end{itemize}

\noindent After mixing and density adjustment, the XeLS is filtered and gradually returned to the inner balloon. This carefully controlled procedure minimizes mechanical stress on the balloon, suppresses the introduction of impurities, and ensures stable long-term detector operation. The xenon handling system has been essential for multiple KamLAND-Zen phases, enabling xenon extraction, purification, and reloading campaigns while maintaining the radiopurity required for world-leading \0nbb searches.


\begin{figure}[t!]
	\centering
	\includegraphics[scale=0.5]{xenonhandling.png}
	\caption{Flow diagram of the KLZ Xenon system. The purple lines denote the flow of Xe/XeLS, the blue line denotes the flow of decane, the the grey line denotes the flow of LS. Figure from Reference~\cite{ozaki_phd}. }
	\label{fig:xenonhandling}
\end{figure}

\section{Data Acquisition}

\subsection{KamLAND DAQ}

The KamLAND data acquisition (DAQ) system is responsible for reading out, digitizing, triggering, and recording signals from the nearly 2,000 photomultiplier tubes instrumenting the detector. The DAQ must accommodate a wide dynamic range of event energies, from low-energy radioactive decays to high-energy cosmic-ray muons, while maintaining high livetime and stable operation over long data-taking periods.

To meet these requirements, KamLAND employs two DAQ systems operating in parallel. The primary system, KamFEE (KamLAND Front-End Electronics), has been used since the beginning of KamLAND physics data-taking and provides reliable readout for the majority of events. The secondary system, MoGURA (Module for General-Use Rapid Application), was developed to mitigate deadtime and waveform distortions following high-energy cosmic-ray muon events. An overview of the dual DAQ architecture and data flow is shown in Figure~\ref{fig:kamland_daq}. Together, these systems ensure efficient readout across a broad range of event types and energies.

\begin{figure}[t!]
	\centering
	\includegraphics[scale=0.3]{kamdaq_flow.png}
	\caption{Flow diagram of the KamLAND data acquisition system, taken from Reference~\cite{li_phd}.}
	\label{fig:kamland_daq}
\end{figure}


\subsection{KamFEE DAQ}

KamFEE constitutes the primary front-end electronics system for KamLAND and is responsible for reading out and controlling all inner detector PMTs. The KamFEE modules are implemented in VME 9U form-factor boards and are synchronized using a global 40\,MHz system clock to ensure precise timing alignment across channels.  Signals from each PMT are split into two parallel processing paths. The first path is routed to a discriminator, which registers a PMT hit when the signal exceeds a predefined threshold corresponding to approximately one-sixth of a single photoelectron. This low threshold allows KamLAND to maintain high efficiency for low-energy events while suppressing electronic noise.

The second signal path is delayed to allow time for the discriminator decision and is subsequently fed into three parallel amplification stages with gains of $\times$20, $\times$4, and $\times$0.5. These gain stages provide sensitivity across a wide dynamic range, enabling accurate digitization of both small scintillation signals and large pulses from energetic events. The amplified signals are digitized by analog transient waveform digitizers (ATWDs). Each ATWD is a 10-bit digitizer that samples the waveform at 1.5\,ns intervals for 128 consecutive samples, corresponding to a total waveform window of approximately 192\,ns. The digitization of a single waveform requires approximately 128\,$\mu$s, during which time the ATWD cannot accept new signals.

To initiate digitization, each KamFEE board generates a ``hitsum'' signal that encodes the number of PMTs registering hits within a predefined time window. This hitsum is sent to the central KamFEE trigger logic, which applies global trigger conditions based on the total number of hits. If the trigger condition is satisfied, a digitization command is issued back to the KamFEE boards.  Because the ATWDs are inactive during digitization, KamFEE assigns two ATWDs to each channel and alternates between them, thereby reducing deadtime. Nevertheless, following large energy depositions—such as cosmic-ray muon events—the KamFEE system can experience extended deadtime due to waveform saturation and recovery effects. These limitations motivated the development of a complementary DAQ system optimized for post-muon recovery.



\subsection{MoGURA}

MoGURA is a secondary DAQ system designed to address limitations of the KamFEE system in the aftermath of high-energy cosmic-ray muon events. Although KamLAND is located deep underground, the residual cosmic muon rate is approximately 0.3\,Hz. These muons deposit large amounts of energy in the detector, saturating front-end electronics and producing long-lasting after-pulses and waveform overshoots that can obscure low-energy signals occurring shortly afterward. To mitigate these effects, MoGURA incorporates several features beyond those of KamFEE:

\begin{itemize}
	\item \textbf{Baseline Recovery:} Following a large energy deposition, PMT signals may exceed the digitization range, resulting in waveform saturation. As the signal returns to baseline, it often exhibits a pronounced overshoot, during which the baseline voltage drops below its nominal value. MoGURA implements baseline restoration techniques that allow reliable digitization of subsequent signals during this recovery period.

	\item \textbf{Adaptive Trigger Mode:} After the detection of a muon event, MoGURA activates a specialized trigger mode optimized for the identification of post-muon activity. This adaptive trigger relies on differential PMT hit patterns and timing information to maintain sensitivity to low-energy events despite elevated noise and after-pulse rates.
\end{itemize}

\noindent In addition to mitigating deadtime, MoGURA plays a critical role in identifying neutrons produced by muon spallation in the detector. These neutrons are typically captured on protons, emitting a 2.2\,MeV gamma ray with a characteristic delay. MoGURA’s improved post-muon sensitivity enables efficient tagging of these spallation neutrons, which are essential for rejecting cosmogenic backgrounds in KamLAND-Zen analyses.

The baseline recovery and neutron-tagging capabilities of MoGURA will be further enhanced in the planned MoGURA2 trigger system. MoGURA2 is designed as a next-generation replacement for both KamFEE and MoGURA in the KamLAND2-Zen experiment, incorporating modern digitization hardware and real-time processing. KamLAND2-Zen is expected to begin physics data-taking in the late 2020s, with MoGURA2 serving as a key component of its upgraded DAQ architecture.


\section{KamLAND-Zen Phases}

The KamLAND-Zen experiment has undergone multiple operational phases, each corresponding to significant upgrades in detector configuration, xenon mass, and background control. These phases were designed to incrementally improve sensitivity to \0nbb decay by increasing exposure while simultaneously reducing background levels. In this section, the major KamLAND-Zen phases are described, with emphasis on the evolution of detector design and background mitigation strategies.


\subsection{KamLAND-Zen 400}

The first implementation of KamLAND-Zen began in 2011 with the installation of an inner balloon and xenon-loaded liquid scintillator at the center of the KamLAND detector, inaugurating the phase known as KamLAND-Zen 400. This configuration featured a spherical inner balloon with a diameter of approximately 3\,m, filled with liquid scintillator loaded with about 3\% xenon by weight. The xenon gas was enriched to approximately 91\% in the double-beta decay isotope $^{136}$Xe, corresponding to an active isotope mass of roughly 400\,kg.

KamLAND-Zen 400 data-taking was divided into two distinct periods, referred to as Phase-1 and Phase-2, reflecting major differences in background conditions and purification status.  During Phase-1, the data were affected by a significant background contribution from the metastable isotope $^{110\mathrm{m}}$Ag. This contamination was traced to the inner balloon film, with the silver believed to have been introduced during balloon fabrication. The presence of $^{110\mathrm{m}}$Ag was likely associated with radioactive fallout from the Fukushima Daiichi nuclear accident, which occurred during the manufacturing period of the inner balloon and in the same geographic region of Japan. The long half-life of $^{110\mathrm{m}}$Ag and its beta decay energy made it a particularly challenging background for the \0nbb search.  To address this issue, extensive purification of the xenon-loaded liquid scintillator was carried out through multiple distillation campaigns. These efforts reduced the $^{110\mathrm{m}}$Ag background by approximately a factor of 20, enabling the start of Phase-2 data-taking under significantly improved background conditions.

Phase-2 accumulated a total livetime of 534.5 days. The combined analysis of Phase-1 and Phase-2 data resulted in a lower limit on the neutrinoless double-beta decay half-life of $^{136}$Xe of $T_{1/2}^{0\nu} > 1.07\times10^{25}\ \mathrm{years}$ at 90\% confidence level. This corresponds to an upper limit on the effective Majorana neutrino mass in the range $m_{\beta\beta} < 61$--165\,meV, depending on the choice of nuclear matrix element calculation.

 

\subsection{KamLAND-Zen 800}

KamLAND-Zen 800 represents the second major phase of the KamLAND-Zen experiment and was designed to significantly extend sensitivity by increasing the xenon mass and improving radiopurity. Data-taking for KamLAND-Zen 800 began in January 2019 and continued through August 2024. Over this period, the experiment accumulated more than 2\,kiloton$\cdot$years of exposure.

The KamLAND-Zen 800 phase employed a substantially upgraded inner balloon and increased the xenon loading to approximately 745\,kg of enriched xenon. In addition to the larger source mass, major emphasis was placed on improving inner balloon cleanliness and reducing contamination from both intrinsic radioactivity and cosmogenic backgrounds. Following the conclusion of data-taking, KamLAND-Zen 800 was decommissioned in the fall of 2024 and is currently in the process of disassembly.

A key upgrade in KamLAND-Zen 800 was the fabrication and deployment of a larger and significantly cleaner inner balloon. The balloon was manufactured at Tohoku University in a class-1 cleanroom to minimize particulate and radioactive contamination. It was constructed from panels of 25\,$\mu$m-thick nylon-6 film, selected for its mechanical strength, chemical compatibility with liquid scintillator, and low intrinsic radioactivity.  The inner balloon fabrication process consisted of multiple carefully controlled steps, several of which are listed below:

\begin{itemize}
	\item \textbf{Washing:} The nylon film was cleaned twice using an ultrasonic bath and subsequently stored between protective cover films to prevent dust adhesion.
	\item \textbf{Welding:} Cleaned balloon panels were welded using a semi-automatic welding machine. For mechanically delicate regions, such as the balloon neck, a hand-operated welding machine was employed. After welding, the average tensile strength across the balloon surface was approximately 35\,N/cm.
	\item \textbf{Helium Leak Check:} Helium gas was injected into the assembled balloon to identify leaks introduced during fabrication. Prior to this step, the protective cover films were removed. Identified leaks were repaired by patching the nylon film. In total, more than 900 leaks were detected and sealed.
	\item \textbf{Folding:} After leak checking, the inner balloon was folded into a cylindrical configuration and covered with protective sheath films to prevent contamination during transport. Teflon sheets and Vectran strings were used to secure the folded balloon.
	\item \textbf{Shipping:} The balloon and associated tools were transported in sealed silver gas bags to minimize exposure to airborne contaminants.
\end{itemize}

The inner balloon was installed on May 10, 2018. Prior to final installation, a full rehearsal deployment was performed in a swimming pool to validate procedures and minimize operational risk. During the final installation, the balloon was deployed through the 50\,cm access port at the top of the KamLAND detector. After initial filling with KamLAND liquid scintillator, the protective Teflon sheets, sheath films, and Vectran strings were carefully removed. The entire deployment was monitored in real time using cameras and an endoscope.

The top of the inner balloon is connected to a corrugated tube fabricated from polyether ether ketone (PEEK), providing mechanical flexibility and chemical compatibility. The balloon is supported by twelve suspension belts that wrap around its full height. The tension in each belt is continuously monitored to ensure positional stability and to detect any mechanical abnormalities. A schematic of the inner balloon structure and its key dimensions is shown in Figure~\ref{fig:balloon_structure}. 

\begin{figure}[t!]
	\centering
	\includegraphics[scale=0.6]{balloonstructure.png}
	\caption{Inner balloon structure and measurements for KamLAND-Zen 800 configuration, taken from Reference~\cite{ozaki_phd}.}
	\label{fig:balloon_structure}
\end{figure}

Once deployed and immersed in liquid scintillator, the inner balloon becomes extremely difficult to access or clean. Consequently, contamination control during fabrication, installation, and early operation is critical for long-term detector performance.

Following deployment, the inner balloon was initially filled with distilled liquid scintillator, during which the $^{232}$Th contamination level was measured to be approximately $10^{-15}$\,g/g, exceeding the target background concentration. Subsequent investigations identified the PPO distillation tower as a potential source of contamination. Detailed studies using inductively coupled plasma mass spectrometry (ICP-MS) and neutron activation analysis were performed at multiple locations along the distillation system to localize the contamination source. After extensive cleaning, filter replacement, and system refurbishment, additional liquid scintillator purification campaigns were conducted. Two major distillation campaigns resulted in a reduction of $^{238}$U and $^{232}$Th contamination levels by approximately an order of magnitude compared to KamLAND-Zen 400.

Residual contamination levels were monitored using delayed coincidence analyses of the $^{214}$Bi--$^{214}$Po and $^{212}$Bi--$^{212}$Po decay sequences. The time evolution of these coincidence event rates is shown in Fig.~\ref{fig:coincidence_rate}, and the resulting contamination levels for each KamLAND-Zen phase are summarized in Table~\ref{tbl:filmcontamination}.

\begin{figure}[t!]
	\centering
	\includegraphics[scale=0.32]{coincidencerate.png}
	\caption{Coincidence event rate in KamLAND-Zen 800 during the first distillation campaign, second distillation campaign, and Zenon loading phase. The red points denote $^{214}$Bi and the blue points denote $^{212}$Bi. Figure taken from Reference~\cite{li_phd}.}
	\label{fig:coincidence_rate}
\end{figure}

\begin{table}[b!]
	\centering
	\renewcommand{\arraystretch}{1.2}
	\begin{tabular}{c|cc}
		\hline
		 & $^{238}$U ($10^{-17}$ g/g) & $^{232}$Th ($10^{-17}$ g/g)\\ \hline
		Zen 400 Phase-1 & 13$\pm$2 & 190$\pm$20 \\
		Zen 400 Phase-2 & 17$\pm$1 & 5.5$\pm$0.3 \\
		Zen 800 & 1.5$\pm$0.4 & 30$\pm$4 \\ \hline
	\end{tabular}
	\caption{Balloon film contamination levels from the last three phases of KamLAND-Zen. Values taken from Reference~\cite{ozaki_phd}.}
	\label{tbl:filmcontamination}
\end{table}

KamLAND-Zen 800 was decommissioned in 2024 after accumulating more than 2\,kiloton$\cdot$years of exposure. The final analysis yielded a lower limit on the neutrinoless double-beta decay half-life of $T_{1/2}^{0\nu} > 3.8\times10^{26}\ \mathrm{years}$ at 90\% confidence level. This result corresponds to an effective Majorana neutrino mass constraint in the range 28--122\,meV, depending on nuclear matrix element assumptions. As of mid-2025, this represents the world-leading limit on the effective Majorana mass from any double-beta decay isotope and provides the only experimental constraint fully covering the inverted neutrino mass ordering.

\subsection{KamLAND2-Zen}

KamLAND2 is the next-generation upgrade of the KamLAND experiment and will be constructed in the same underground cavern as the original KamLAND detector. Building on more than two decades of operational experience, KamLAND2 is designed to significantly improve detector performance through comprehensive upgrades to the photodetectors, optical coverage, inner balloon technology, and data acquisition system. The KamLAND2-Zen program, which incorporates a xenon-loaded liquid scintillator target, aims to reach a neutrinoless double-beta decay half-life sensitivity of
$T_{1/2}^{0\nu\beta\beta} > 2\times10^{27}\ \mathrm{years}$,
corresponding to sensitivity well into the inverted neutrino mass ordering region.

Achieving this goal requires both increased exposure and substantial reductions in background rates, particularly from the tail of the two-neutrino double-beta decay spectrum and residual radioactive contamination. To this end, most major detector components will be replaced or upgraded in the transition from KamLAND to KamLAND2. The most significant upgrades are summarized below.

\begin{itemize}
	\item \textbf{Inner Detector PMTs:} All 1,879 inner detector PMTs will be replaced with modern photodetectors featuring significantly improved quantum efficiency and reduced transit time spread (TTS). These new PMTs are expected to provide higher photon detection efficiency and superior timing performance, leading to improved energy and vertex reconstruction. The reduction in TTS is particularly important for enhancing position resolution and background discrimination in large-volume liquid scintillator detectors.

	\item \textbf{Light-Collecting Mirrors:} Winston cone light concentrators will be installed on each PMT to dramatically increase the effective photocathode coverage, approaching nearly 100\% optical coverage of the inner detector. By redirecting otherwise lost photons onto the PMT photocathodes, these concentrators substantially increase light yield. Together with the upgraded PMTs, this improvement is expected to achieve a target energy resolution of approximately 2\% at the $^{136}$Xe \0nbb $Q$-value. Such an energy resolution would suppress the contribution from the high-energy tail of the two-neutrino double-beta decay spectrum by roughly two orders of magnitude, greatly reducing one of the dominant irreducible backgrounds.

	\item \textbf{Improved Inner Balloon:} The KamLAND2-Zen inner balloon will be constructed from polyethylene naphthalate (PEN), a material that exhibits intrinsic scintillation. Unlike nylon-based films used in previous phases, PEN allows radioactive decays occurring within the balloon film itself to produce detectable scintillation light. This property enables active tagging and rejection of background events originating from the balloon material, significantly reducing backgrounds associated with surface contamination while maintaining mechanical strength and chemical compatibility with liquid scintillator.

	\item \textbf{MoGURA2 DAQ System:} The existing dual DAQ architecture (KamFEE and MoGURA) will be replaced by MoGURA2, a newly developed, compact, deadtime-free data acquisition system based on RFSoC (Radio Frequency System-on-Chip) technology. MoGURA2 integrates high-speed digitization, real-time signal processing, and flexible trigger logic within a single platform. This system is designed to provide continuous waveform acquisition without deadtime, improved post-muon recovery, and enhanced neutron tagging capability, all of which are critical for high-sensitivity rare-event searches.
\end{itemize}

\noindent With these upgrades, KamLAND2-Zen will combine increased source mass, improved energy resolution, and enhanced background rejection to achieve an unprecedented sensitivity to \0nbb decay. Construction and commissioning of KamLAND2-Zen are ongoing, with the start of physics data-taking currently planned for 2028.



