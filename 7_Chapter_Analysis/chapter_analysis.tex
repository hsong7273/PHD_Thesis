\chapter{$2\nu\beta\beta^\ast$ Analysis}
\label{chapter:Analysis}
\thispagestyle{myheadings}

\graphicspath{{7_Chapter_Analysis/Figures/}}

This chapter describes the analysis framework used to search for the $2\nu\beta\beta^\ast$ decay of $^{136}$Xe in KamLAND-Zen 800. We first summarize the xenon loading and enrichment used to normalize decay rates. We then define the full dataset used in this work, including data-taking periods excluded specifically for the excited-state analysis. Next, we summarize the dominant sources of systematic uncertainty. Finally, we describe the one-dimensional binned spectral fit used to extract (or constrain) the $2\nu\beta\beta^\ast$ rate, and we present both a profile-likelihood (Wilks) limit and a Feldman--Cousins construction to obtain correct 90\% confidence-level coverage.

\section{Xenon Enrichment in KamLAND-Zen}
The $^{136}$Xe exposure is a normalization factor for all xenon decay rates. The total mass of xenon dissolved into the XeLS is estimated by accounting for xenon remaining in the handling system after installation:
\begin{enumerate}
    \item initial xenon mass: $769 \pm 1$~kg,
    \item xenon remaining in LS tanks and pipelines: $21.5 \pm 2.8$~kg,
    \item xenon remaining in storage bottles: $1 \pm 1$~kg,
    \item xenon trapped in the charcoal filter and/or used for sampling: $1.5 \pm 0.5$~kg.
\end{enumerate}
From these terms, the installed Xenon gas is determined to be, $745\pm3$ kg. The isotopic composition of the enriched xenon is evaluated using a mass spectrometer. The measured values agree with the enrichment specification provided by the supplier. The enrichment fractions and the corresponding $^{136}$Xe mass are summarized in Table~\ref{tab:xenon_comp}.

\begin{table}[t!]
    \centering
    \setlength{\tabcolsep}{10pt} % Default value: 6pt
    \renewcommand{\arraystretch}{1.2} % Default value: 1
    \begin{tabular}{lcccc}
        \hline
        & $^{136}$Xe & $^{134}$Xe & Others & Total \\
        \hline
        Provided fraction [\%] & 90.85 & 8.82 & 0.33 & 100.00 \\
        Measured fraction [\%] & $90.77 \pm 0.08$ & $8.96 \pm 0.02$ & -- & -- \\
        Atomic mass [u] & 135.907 & 133.905 & -- & -- \\
        Total mass [kg] & 677.39 & 64.83 & 2.79 & 745.0 \\
        \hline
    \end{tabular}
    \caption{Enriched xenon composition used in this analysis.}
    \label{tab:xenon_comp}
\end{table}


\section{Full KamLAND-Zen 800 Dataset}
The dataset used in this analysis was collected between February~5, 2019 and April~30, 2023 (run range 15431--18691). Unlike the primary $0\nu\beta\beta$ analysis, the excited-state search presented here does not subdivide the data by long-lived spallation likelihood. Instead, all events passing the selections in Chapter~\ref{chapter:reco_select} are combined into a single energy spectrum and fitted simultaneously.

\subsection{Vetoed Data Periods}
In addition to standard run-quality selections and detector deadtime described in Chapter~\ref{chapter:reco_select}, we exclude several data-taking periods that are specific to this analysis.

\subsection*{Electric power supply instability}
Runs 16790--16874 are excluded because the data acquisition system was unstable due to automatic voltage regulator (AVR) issues. Frequent DAQ restarts produced many short runs, which prevents stable run-by-run calibration as described in Chapter~\ref{chapter:calibration}.

\subsection*{MoGURA disorder period}
MoGURA data were unstable between September and November 2022 and some raw files were corrupted. While KamDAQ files remain readable, MoGURA information is required to reconstruct and tag events close in time to cosmic-ray muons, which provides key constraints on muon-correlated spallation backgrounds. Therefore, runs 17768--17905 are excluded from this analysis.

\section{Systematic Uncertainties}
Although the $2\nu\beta\beta^\ast$ search is dominated by statistical uncertainty from the large $2\nu\beta\beta$ background, several systematic effects contribute non-negligibly to the final limit. The uncertainty in the total xenon mass dissolved in the XeLS is taken as 0.4\% from an internal study. The xenon enrichment uncertainty is taken as the difference between the supplier specification and the mass-spectrometer measurement (0.1\%). The detector energy response drifts slowly over time due to electronics changes and PMT performance. Using the 2.2~MeV neutron-capture peak, we assign a conservative 0.9\% uncertainty on the energy scale.

Finally, the fiducial-volume (FV) uncertainty reflects the uncertainty in the true volume enclosed by the FV cut. This is evaluated using uniformly distributed $^{222}$Rn introduced during xenon loading, tagged via $^{214}$Bi--Po delayed coincidences. Comparing the expected volume ratio of the FV to the inner balloon volume (0.5686) with the observed Bi--Po ratio (0.5454), we assign a 4.1\% FV uncertainty.

\section{Spectral Fit}
The $2\nu\beta\beta^\ast$ decay rate is extracted (or constrained) by fitting signal and background models to the reconstructed visible-energy spectrum of data events within the reduced fiducial volume $r<1.33$~m, after applying the event selections described in Chapter~\ref{chapter:reco_select}.

In contrast to analyses that bin the data in time and/or hemisphere, we adopt a single one-dimensional spectral fit to reduce complexity and mitigate systematics associated with time-dependent detector response and imperfect spatial modeling. This choice is additionally motivated by the fact that the excited-state search is dominated by the statistical precision of the underlying $2\nu\beta\beta$ spectrum.

The fit is performed over $E_{\rm vis}=0.5$--4.8~MeV using 0.05~MeV-wide bins. This window includes the low-energy radioactive backgrounds ($^{85}$Kr and $^{210}$Bi), cosmogenic contributions ($^{11}$C), the $^{40}$K peak, the $2\nu\beta\beta$ endpoint region, and the $2\nu\beta\beta^\ast$ region of interest.

\subsection{Likelihood (NLL) Definition}
We perform a binned maximum-likelihood fit using Poisson statistics for each energy bin and Gaussian penalty terms for externally constrained parameters. Following the formulation used in the analysis summary slides, the fit minimizes the negative log-likelihood (NLL), written schematically as :contentReference[oaicite:1]{index=1}
\begin{equation}
    -\ln L(\boldsymbol{\eta},\boldsymbol{\theta})
    = \sum_{i=1}^{N_{\rm bins}}
    \left[
        \nu_i(\boldsymbol{\eta},\boldsymbol{\theta})
        - n_i \ln \nu_i(\boldsymbol{\eta},\boldsymbol{\theta})
    \right]
    + \sum_{j=1}^{N_{\rm constr}}
    \frac{\left(\theta_j-\theta_{j,0}\right)^2}{2\sigma_j^2}
    \;+\;{\rm const.},
\end{equation}
where $n_i$ is the observed number of events in energy bin $i$, and $\nu_i$ is the expected number of events in that bin given the model parameters. The vector $\boldsymbol{\eta}$ denotes free (or scanned) parameters such as component normalizations, while $\boldsymbol{\theta}$ denotes constrained nuisance parameters (e.g., background rates constrained by independent measurements and detector-response parameters). The Gaussian penalty terms implement external constraints with central values $\theta_{j,0}$ and standard deviations $\sigma_j$.

For numerical stability (including bins with $n_i=0$), the equivalent Poisson deviance form is often used:
\begin{equation}
    \chi^2_{\rm Poisson}
    = 2\sum_{i=1}^{N_{\rm bins}}
    \left[
        \nu_i - n_i + n_i\ln\left(\frac{n_i}{\nu_i}\right)
    \right]
\end{equation}
with the convention that the $n_i\ln(n_i/\nu_i)$ term is zero when $n_i=0$. In this work, we report fit quality and profile scans using the test statistic:
\begin{equation}
    \Delta\chi^2(\mu) \equiv 2\left[-\ln L(\mu) + \ln L(\hat{\mu})\right]
\end{equation}
where $\mu$ denotes the fixed (scanned) $2\nu\beta\beta^\ast$ signal rate and $\hat{\mu}$ is its best-fit value.

\subsection{Model Prediction per Energy Bin}
The expected bin counts are computed as a sum over templates:
\begin{equation}
    \nu_i(\boldsymbol{\eta},\boldsymbol{\theta})
    =
    \sum_{k}
    \eta_k \, T_{k,i}(\boldsymbol{\theta})
\end{equation}
where $T_{k,i}$ is the predicted spectral shape (template) for component $k$ in bin $i$, including detector-response effects such as energy scale and nonlinearity. The coefficients $\eta_k$ represent the fitted normalizations (rates) of each background and the signal.

An overall energy-scale parameter $\alpha_E$ is implemented by mapping reconstructed energy to simulated energy via $E_{\rm vis} = \alpha_E E_{\rm sim}$ and using linear interpolation of templates to compute $\nu_i$.

\subsection{Minimization and Fit Configuration}
The NLL is minimized using the MINUIT package~\cite{James:1975dr} in the ROOT framework~\cite{root}. The fit parameters are grouped by source location (XeLS and inner balloon film). The overall configuration is summarized in Table~\ref{tab:parameters}. This analysis uses 29 fit parameters in total, with a mixture of fixed, free, and constrained parameters.

\begin{table}[p]
    \centering
    \setlength{\tabcolsep}{10pt} % Default value: 6pt
    \renewcommand{\arraystretch}{1.2} % Default value: 1
    \begin{tabular}{lll}
        \hline
        Material & Parameter & Fit Condition \\
        \hline
        \multirow{12}{*}{XeLS}
            & $^{136}$Xe $2\nu\beta\beta^\ast$ & scan \\
            & $^{136}$Xe $2\nu\beta\beta$ & free \\
            & $^{238}$U series 2 & constrained \\
            & $^{222}$Rn & constrained \\
            & $^{232}$Th series 2 & constrained \\
            & $^{210}$Bi & free \\
            & $^{85}$Kr & free \\
            & $^{11}$C & constrained \\
            & $^{137}$Xe & constrained \\
            & Xe spallation & free \\
            & solar $\nu$ (ES+CC) & fixed \\
            & $^{136}$Cs & constrained \\
        \hline
        \multirow{6}{*}{Film}
            & $^{238}$U series 1 & fixed \\
            & $^{238}$U series 2 & free \\
            & $^{232}$Th series 1 & free \\
            & $^{232}$Th series 2 & free \\
            & $^{40}$K & constrained \\
            & $^{210}$Bi & free \\
        \hline
        \multirow{7}{*}{All}
            & Energy scale $\alpha_E$ & constrained \\
            & $k_B, R$ & fixed \\
            & LL distortion & constrained \\
            & $^6$He & fixed \\
            & $^{12}$B & fixed \\
            & $^8$Li & fixed \\
            & $^8$B & fixed \\
        \hline
    \end{tabular}
    \caption{Fit parameter configuration for the spectral analysis. The fit condition column indicates whether parameters are free, fixed, scanned, or constrained in the fit.}
    \label{tab:parameters}
\end{table}

The constrained radioactive-chain rates ($^{238}$U series 2, $^{222}$Rn, and $^{232}$Th series 2) are obtained from coincidence-tagged Bi--Po events (Chapter~\ref{chapter:reco_select}). The $^{11}$C and $^{40}$K constraints come from independent measurements described in Chapter~\ref{chapter:Backgrounds}. The $^{137}$Xe constraint is derived from its correlation with muon-induced neutrons tagged by MoGURA (Chapter~\ref{chapter:Backgrounds}).

A penalty term is also included for detector-response parameters, most importantly the global energy scale. Although the short-term energy-scale stability, evaluated using the 2.2 MeV neutron-capture peak, is better than 1\%, the global energy-scale parameter in the spectral fit is constrained more conservatively. Following the approach adopted in previous KamLAND-Zen analyses and theses~\cite{ozaki_phd,miyake_phd}, we apply a Gaussian prior of 
$\alpha_E=1\pm0.016$ to account for residual nonlinearity, spatial dependence, and long-term variations not explicitly modeled in the fit while keeping $k_B$ and the Cherenkov fraction $R$ fixed for fit stability.


\section{$2\nu\beta\beta^\ast$ Results}

\subsection{Best-Fit Result}

The spectral fit to the full KamLAND-Zen 800 dataset yields a best-fit $2\nu\beta\beta^\ast$ signal rate consistent with zero, indicating no statistically significant excess above the background-only hypothesis. Figure~\ref{fig:bestfit} shows the fitted energy spectrum in the reduced fiducial volume ($r<1.33$~m), together with the dominant background components and the expected signal shape.

\begin{figure}[t!]
    \centering
    \includegraphics[scale=0.4]{best_fit_thesis.png}
    \caption{Best-fit energy spectrum for the full KamLAND-Zen 800 dataset in the reduced fiducial volume $r<1.33$~m.}
    \label{fig:bestfit}
\end{figure}

Table~\ref{tab:fit_result} summarizes the best-fit parameter values and their fit conditions. All constrained parameters remain within their assigned prior uncertainties, indicating no significant tension between the data and the background model. A notable feature of the fit is the $^{232}$Th S1 contribution on the mini-balloon film, which is driven to zero. This behavior is expected, as the $^{232}$Th S1 film background is spectrally degenerate with the $^{232}$Th S2P film component in the absence of strong radial discrimination. Since both components originate from the mini-balloon film and exhibit similar spatial and spectral characteristics, the inclusion of both parameters introduces redundancy, with the fit naturally preferring a single effective contribution. Overall, the stability of the fit and the absence of pathological parameter behavior demonstrate that the background model provides an adequate description of the data in the energy region relevant for the $2\nu\beta\beta^\ast$ search.

\begin{table}
    \centering
    \setlength{\tabcolsep}{10pt} % Default value: 6pt
    \renewcommand{\arraystretch}{1.2} % Default value: 1
    \begin{tabular}{c|c|c}
    \hline
        Parameter & Value & Fit Condition \\ \hline
        $^{12}$B Spallation &  0.016 & fix \\
        $^{8}$B Spallation &  0.239 & fix\\
        $^{210}$Bi (XeLS) &  26000 & floated\\
        $^{210}$Bi (film) &  22930 & floated\\
        $^{11}$C (XeLS) &  983 & constrained\\
        $^{136}$Cs (XeLS) &  0.8 & fix\\
        $^{6}$He Spallation &   0.33 & fix\\
        $^{40}$K (film) &  186 & constrained \\
        $^{85}$Kr (XeLS) & 41500 & floated \\
        $^{8}$Li Spallation & 0.525 & fix \\
        Long-lived (XeLS) & 0.68 & floated \\
        Monochromatic & 3.4e+07 & fix \\
        $^{222}$Rn (XeLS) & 8002.15 & constrained \\
        Signal ($2\nu\beta\beta^*$, XeLS) & 0 & floated \\
        Solar $\nu$ (XeLS) & 4.87 & fix \\
        $^{232}$Th S1 (film) & 0 & floated \\
        $^{232}$Th S2P (XeLS) & 114 & constrained \\
        $^{232}$Th S2P (film) & 98.7 & floated \\
        $^{238}$U S1 (film) & 25 & fix \\
        $^{238}$U S2 (XeLS) & 30.2 & constrained \\
        $^{238}$U S2 (film) & 61.3 & floated \\
        $^{136}$Xe $2\nu\beta\beta$ (XeLS) & 107601 & floated \\
        $^{137}$Xe (XeLS) & 0.980653 & constrained \\
        Energy Scale $\alpha$ (Internal) & 0.991 & constrained \\
        LL distortion & 0.118 & constrained \\
        $R$ (Internal) & 0.018 & fix \\
        $k_B$ (Internal) & 0.31 & fix \\ \hline
    \end{tabular}
    \caption{The best fit parameter values and their fit conditions.}
    \label{tab:fit_result}
\end{table}

\subsection{\twonustar\ Half-life Limits}

In the absence of a statistically significant $2\nu\beta\beta^\ast$
signal, an upper limit on the decay rate is derived at the 90\%
confidence level (C.L.). This is achieved by performing a profile
likelihood scan over the $2\nu\beta\beta^\ast$ signal rate. For each fixed signal-rate hypothesis, the remaining background parameters are
re-optimized to minimize the negative log-likelihood (NLL). The
degradation in the agreement between the model and the data is quantified by the test statistic:
\begin{equation}
    \Delta\chi^2 = 2\left[\mathrm{NLL}(\text{rate}) -
    \mathrm{NLL}_{\text{best fit}}\right]
\end{equation}
\noindent where $\mathrm{NLL}_{\text{best fit}}$ corresponds to the global minimum. The result of this scan is shown in Figure~\ref{fig:chi2scan}.

\begin{figure}[t!]
	\centering
	\includegraphics[scale=0.7]{chi2scan.png}
	\caption[Difference in the NLL test statistic as a function of fixed $2\nu\beta\beta^\ast$ signal rate.]{Difference in the NLL test statistic as a function of fixed $2\nu\beta\beta^\ast$ signal rate. The horizontal line indicates the Wilks' theorem threshold corresponding to a one-parameter 90\% C.L.}
	\label{fig:chi2scan}
\end{figure}

Under the assumptions of Wilks' theorem~\cite{Hardin_2024}, the 90\% C.L. upper limit on the signal rate is obtained by identifying the rate at which
$\Delta\chi^2 = 2.71$, corresponding to the 90\% quantile of the $\chi^2$ distribution with one degree of freedom. The corresponding lower limit on the $2\nu\beta\beta^\ast$ half-life is calculated as:
\begin{equation}
    T_{1/2} = \frac{\ln(2)\,N_{^{136}\mathrm{Xe}}}
    {\text{(rate)} \times 365.2},
\end{equation}
\noindent where $N_{^{136}\mathrm{Xe}}$ is the number of $^{136}$Xe atoms in the xenon-loaded liquid scintillator (XeLS). The xenon concentration in the XeLS is given by:
\begin{align}
    C_{\mathrm{Xe}} &= \frac{\text{Xe mass}}{\text{total XeLS mass}} \times 100 \\
    &= \frac{745~\text{kg}}
    {30.5~\text{m}^3 \times 780.13~\text{kg/m}^3} \times 100 \\
    &= 3.13 \pm 0.01~\text{wt}\% 
\end{align}

\noindent Using the $^{136}$Xe enrichment fraction of 90.85\%, the number of
$^{136}$Xe atoms per kiloton of XeLS is calculated as
\begin{align}
N_{^{136}\mathrm{Xe}} &=
\left(1~\text{kton} \times C_{\mathrm{Xe}} \times 0.9085\right)
\frac{N_A}{M_{\mathrm{Xe}}} \notag \\
&= \frac{3.13 \times 0.9085}{135.80}
\times 6.022 \times 10^{23} \notag \\
&= (1.261 \pm 0.004)\times 10^{29}~
(\text{kton-XeLS})^{-1}
\end{align}
\noindent where $N_A$ is Avogadro's number and $M_{\mathrm{Xe}}$ is the average atomic mass of enriched xenon.

\subsection{Validity of Wilks' Theorem}

While the Wilks' theorem-based limit provides a useful reference, it is likely not strictly valid for this analysis. First, the
$2\nu\beta\beta^\ast$ signal rate is bounded from below at zero, violating the regularity conditions required for Wilks' theorem. Second, the observed test statistic profile is non-parabolic near the best-fit point and rises sharply as the signal rate approaches zero. These features indicate that asymptotic $\chi^2$ approximations may not accurately describe the true distribution of the test statistic.

For these reasons, the Wilks' theorem result should be regarded as a
preliminary and potentially aggressive estimate. A more reliable
confidence interval is obtained using the Feldman--Cousins approach,
described in the following section.


\subsection{Feldman--Cousins Calculation}

To obtain a statistically robust upper limit with correct coverage, the Feldman--Cousins (FC) method is employed~\cite{feldman-cousins}. Unlike Wilks' theorem, the FC approach does not rely on asymptotic assumptions and explicitly accounts for physical parameter boundaries. The FC construction is performed by scanning over fixed
$2\nu\beta\beta^\ast$ signal-rate hypotheses in steps of 1~cts/(kt$\cdot$day). At each signal rate, an ensemble of Toy Monte Carlo
datasets is generated using the best-fit background model. Each Toy
dataset is fit in the same manner as the real data, and the
corresponding test statistic is recorded. For each signal-rate
hypothesis, 10,000 Toy MC fits are performed to construct the test
statistic distribution, from which the 90\% inclusion threshold is
determined.

The $\Delta\chi^2$ profile obtained from the real data represents a
single realization drawn from these distributions. The 90\% C.L. upper limit is defined by the signal rate at which the observed test statistic intersects the FC inclusion threshold. The result of this procedure is shown in Figure~\ref{fig:FC}, where the Feldman--Cousins inclusion threshold is observed to be nearly flat as a function of the assumed $2\nu\beta\beta^\ast$ signal rate. This behavior indicates that the test statistic distribution is dominated by background fluctuations rather than by the signal hypothesis itself, consistent with the background-limited sensitivity of the analysis. The intersection occurs at a signal rate of $R = 56.25$~cts/(kt$\cdot$day).

At large signal rates, the Feldman--Cousins inclusion threshold
approaches $\Delta\chi^2 \simeq 2.8$, close to the Wilks' theorem
expectation of 2.71 for a single parameter. However, the pronounced
non-parabolic shape of the data test statistic near the physical
boundary at zero signal rate demonstrates the breakdown of asymptotic
$\chi^2$ assumptions in this regime, motivating the use of the
Feldman--Cousins construction.

\begin{figure}[t!]
    \centering
    \includegraphics[scale=0.7]{FC_Final_Stat.pdf}
    \caption[Feldman--Cousins analysis results.]{Feldman--Cousins analysis results. The black curve shows the test statistic obtained from the data as a function of fixed $2\nu\beta\beta^\ast$ signal rate. The red curve indicates the 90\% Feldman--Cousins inclusion threshold derived from Toy MC ensembles. Their intersection defines the 90\% C.L. upper limit.}
    \label{fig:FC}
\end{figure}

Using this approach, the 90\% C.L. lower limit on the $2\nu\beta\beta^\ast$ half-life of $^{136}$Xe is found to be:
\begin{equation}
    T_{1/2} > 4.25 \times 10^{24}~\text{yr}
\end{equation}
\noindent As expected, this result is slightly more conservative than the corresponding Wilks' theorem estimate.  

\subsection{Discussion}

While preliminary, this analysis establishes a new world-leading limit on the $2\nu\beta\beta^\ast$ decay of $^{136}$Xe to the first excited $0^+_1$ state. The derived half-life limit of $T_{1/2} > 4.25 \times 10^{24}$~yr exceeds the previous best constraint from EXO-200, $T_{1/2} > 1.4 \times 10^{24}$~yr~\cite{exo200}, as well as
the earlier KamLAND-Zen 400 result, $T_{1/2} > 8.3 \times 10^{23}$~yr~\cite{KLZ400}. This improvement is driven primarily by the increased $^{136}$Xe exposure of the KamLAND-Zen~800 dataset and the enhanced control of backgrounds achieved through improved detector modeling and statistical treatment.

Figure~\ref{fig:2nuex_limits} summarizes the current experimental status of the $2\nu\beta\beta^\ast$ search by comparing the KamLAND-Zen~800 result with previous experimental limits and a representative set of theoretical predictions compiled in Reference~\cite{2nuex_predictions}. The experimental results are shown as arrows, where the arrow tip indicates the 90\% C.L. lower limit on the half-life. The arrow length carries no physical meaning and is used solely to visually indicate that these are lower limits rather than measured values.

\begin{figure}[b!]
    \centering
    \includegraphics[scale=0.9]{limits_and_theory.png}
    \caption[A comparison of experimental lower limits on the $2\nu\beta\beta^\ast$ decay half-life to the $0^+_1$ excited state in $^{136}$Xe and selected theoretical predictions.]{A comparison of experimental lower limits on the $2\nu\beta\beta^\ast$ decay half-life to the $0^+_1$ excited state in $^{136}$Xe and selected theoretical predictions taken from Reference~\cite{2nuex_predictions}. The experimental results are shown as arrows, where the arrow tip indicates the 90\% C.L. lower limit on the half-life and the arrow length has no physical meaning.
}
    \label{fig:2nuex_limits}
\end{figure}

The shaded horizontal bands represent predicted half-life ranges from
different nuclear-structure approaches, including QRPA, IBM-2, NSM, and EFT-based calculations. The vertical extent of each band reflects the spread of predicted values arising from model assumptions, treatment of nuclear correlations, and uncertainties in nuclear matrix elements. As such, these bands should be interpreted as indicative ranges rather than precise predictions.

The KamLAND-Zen~800 limit lies above the upper edge of the QRPA
prediction band, thereby excluding the currently available QRPA
calculations for the $2\nu\beta\beta^\ast$ transition in $^{136}$Xe.
However, the result remains below the central values of the IBM-2, NSM, and EFT-based predictions, indicating that further sensitivity
improvements are required to fully probe the favored parameter space of these models. The progression from KLZ-400 to EXO-200 and now to
KLZ-800 illustrates a steady experimental approach toward the predicted half-life region.

It is noteworthy that for several lighter nuclei, measured
$2\nu\beta\beta^\ast$ half-lives have historically been significantly
shorter than early theoretical expectations. In this context, the
exclusion of a substantial portion of the predicted parameter space
represents meaningful progress, even in the absence of a positive
observation. Continued improvements in exposure, background reduction, and energy response modeling in future KamLAND-Zen phases may enable direct tests of the remaining theoretical predictions.
