\chapter{$2\nu\beta\beta^*$ Analysis}
\label{chapter:Analysis}
\thispagestyle{myheadings}

\graphicspath{{7_Chapter_Analysis/Figures/}}


This chapter describes the data analysis framework used to search for \twonustar. First the KamLAND-ZEN 800 full dataset is described, including any vetoed data-taking periods. This description is followed by an overview of the systematic uncertainties. Then, the energy spectral fitting procedure is outlined, including the definition of the chi-square metric. Finally, the statistical results are presented, culminating in a limit on the \twonustar rate.

\section{Xenon Enrichment in KamLAND-ZEN}
The amount of xenon gas dissolved into the XeLS is a normalization factor in the final xenon decay rates. The value is determined by subtracting off the xenon that remains after xenon installation:
\begin{enumerate}
	\item initial Xe mass : $769\pm1$ kg
	\item Xe left in LS in tanks and pipe lines : $21.5\pm2.8$ kg
	\item Xe left in storage bottles : $1\pm 1$ kg
	\item Xe trapped by charcoal filter or used for sampling : $1.5\pm 0.5$ kg
\end{enumerate}
From the above calculation, the installed Xenon gas is determined to be, $745\pm3$ kg. The composition of the enriched xenon is evaluated using a mass spectrometer. The measured values agree well with the values provided by the procurement company. The enrich xenon composition can be seen in Table \ref{tab:xenon_comp}.

\begin{table}[htbp]
    \centering
    \caption{Enriched Xenon Composition}
    \label{tab:xenon_comp}
    \begin{tabular}{lcccc}
        \hline
        & $^{136}$Xe & $^{134}$Xe & Others & Total \\
        \hline
        Provided ratio [\%] & 90.85 & 8.82 & 0.33 & 100.00 \\
        Measured ratio [\%] & $90.77 \pm 0.08$ & $8.96 \pm 0.02$ & -- & -- \\
        Atomic mass [u] & 135.907 & 133.905 & -- & -- \\
        Total mass [kg] & 677.39 & 64.83 & 2.79 & 745.0 \\
        \hline
    \end{tabular}
\end{table}


\section{Full KamLAND-ZEN 800 Dataset}
The dataset used in this analysis was taken between February 5, 2019 and April 30, 2023, run range : 15431-18691. Unlike the \0nbb analysis, the dataset is not divided by Long-lived spallation background likelihood into a singles and long-lived dataset. Instead all the events are combined into a single energy spectrum for joint fitting.


\subsection{Vetoed Data Periods}
In addition to the regular deadtime due to maintenance, run quality, described in Chapter \ref{chapter:reco_select}, there are additional vetoed data periods specifically for this analysis. 
\subsection*{Electric Power Supply Instability}
The run range 16790-16874 are excluded from this analysis because the DAQ was unstable due to AVR (automatic voltage regulator) trouble. The constant restarting of the DAQ leads to short runs which make it difficult to perform the run-by-run calibration described in Chapter \ref{chapter:calibration}. 
\subsection*{MoGURA disorder period}
The MoGURA DAQ system was unstable between September-November 2022 and raw data files were corrupted. While the KamDAQ files are readable, the MoGDAQ data is crucial for constructing events near after cosmic ray muons. These events are a key tag for cosmic related spallation backgrounds. Thus, runs 17768-17905 are excluded from this analysis.

\section{Systematic Uncertainties}
While the \0nbb and \twonustar analyses are statistical uncertainty dominated, this section discusses estimates of several sources of systematic error.

The uncertainty in the total amount of xenon dissolved in the XeLS was estimated in an internal study, the estimated uncertainty is 0.4\%. 

As for the xenon enrichment factor, the 0.1\% difference in the supplier stated enrichment and our measuremed enrichment, described in the previous section, is used as the systematic uncertainty in xenon enrichment.

The detector energy scale varies over time as electronics fail, are repaired, and PMTs degrade. Using the neutron capture gamma peak at 2.2 MeV, a maximum error of 0.9\% was determined.

Finally, the uncertainty in the fiducial volume is the uncertainty in the true volume of events encapsulated by the fiducial volume selection of 157.49 cm. This is determined using the early KLZ-800 data, just after xenon dissolving work. As the xenon is introduced, $^{222}$Rn is incidentally introduced from the atmosphere, $^{222}$Rn has a half-life of 3.8 days, and its decay is soon followed by a $^{214}$Bi-Po sequential decays. Data for a month after xenon is introduced to the XeLS is used and the Bi-Po coincident events are analyzed. The volume ratio between spheres of 157.49cm and 192cm, the inner balloon radius, is 0.5686. While the ratio of Bi-Po events observed in these spherical regions is 0.5454. Since it is expected that the $^{222}$Rn is distributed uniformly throughout the XeLS, a 4.1\% difference is taken as the fiducial volume uncertainty.
\section{Spectral Fit}
The \twonustar  decay rate is estimated by fitting background and signal models to the energy distribution of reconstructed KLZ-800 data. Namely, the energy distribution of events within the reduced FV of $(r<1.33m)$ and that pass the event selections discussed in Chapter \ref{chapter:reco_select}.

Unlike the \0nbb analysis \cite{klz800_arxiv}, the data is not separated into time and hemipsherical volume bins. This is done to reduce systematic uncertainties and simplify the analysis. As this analysis is dominated by the statistics of the \2nbb background, the modeling of time-dependent changes in detector response should be marginal. By reducing the fiducial volume to $(r<1.33m)$ we sacrifice statistics but gain robustness by focusing on the best characterized region of the detector.

Thus, a simple 1-dimensional spectral fit is performed over the energy range (0.5-4.8) MeV wiht 0.05 MeV wide bins. This energy range spans the $^{210}$Bi and $^{85}$Kr peaks, the low-energy portion of the \2nbb spectrum, the $^{11}$C and $^{40}$K decays, the \2nbb endpoint, the \0nbb ROI, the $^{212}$Tl peak, and the $2\nu\beta\beta^\ast$ ROI.
\subsection{Chi-Square Definition}
In this study, a binned chi-square, maximum likelihood fit is performed. The chi-square has multiple components, a energy-bin term and penalty terms.
\begin{equation}
    \chi^2=(\sum_{energy} \chi^2_{energy}) +\chi^2_{penalty}
\end{equation}
Here, $\sum_{energy}$, denotes a sum over each 0.05 MeV energy bin from the range 0.5-4.8 MeV. In each energy bin, the $\chi^2$ is computed.
\begin{equation}
\chi^2_{\text{energy}} =
\begin{cases}
    2 \sum_{i} \left( \nu_{i} - n_{i} + n_{i} \log \frac{n_{i}}{\nu_{i}} \right) & (n_{i} > 0) \\
    2 \sum_{i} (\nu_{i} - n_{i}) & (n_{i} = 0)
\end{cases}
\end{equation}
Now, $\nu_i$ is the model expected energy spectrum for the given fit parameters, and $n_i$ denotes the observed number of events in the $i$-th bin. The penalty terms constrains certain parameters that have independent constraints. The fit parameter configuration is summarized in Table \ref{tab:parameters}, and described in more detail in the later sections. The penalty $\chi^2$ terms are simply defined as:
\begin{equation}
    \chi^2_{penalty}=\sum_i\left(\frac{O_n-E_n}{\sigma_n}\right)^2
\end{equation}
where $O_n$ is the estimated parameter value, $E_n$ are the central expected values, and $\sigma_n$ are the expected parameter errors. 

\subsection{Minimizer}
The ROOT implementation of the MINUIT package distributed by CERN is the minimization package used for this analysis. 

\subsection{Fit Parameters}
The spectral rate parameters are divided by origin volume. This analysis is only concerned with backgrounds originating in the XeLS and the inner balloon film. Each of the backgrounds described in Chapter \ref{chapter:backgrounds} is implemented in the spectral fit. 
\subsection{Penalty Terms}

\begin{table}[htbp]
    \centering
    \caption{Fit parameter configuration for the spectral analysis. The fit condition column indicates whether the parameter is free, fixed, scanned, or constrained in the fit.}
    \label{tab:parameters}
    \begin{tabular}{lll}
        \hline
        Material & Parameter & Fit Condition \\
        \hline
        \multirow{12}{*}{XeLS}
            & $^{136}$Xe 2$\nu\beta\beta^*$ & scan \\
            & $^{136}$Xe 2$\nu\beta\beta$ & free \\
            & $^{238}$U series 2 & constrain \\
            & $^{222}$Rn & constrain \\
            & $^{232}$Th series 2 & constrain \\
            & $^{210}$Bi & free \\
            & $^{85}$Kr & free \\
            & $^{11}$C & constrain \\
            & $^{137}$Xe & constrain \\
            & Xe spallation & free \\
            & solar $\nu$ ES + CC & fix \\
            & $^{136}$Cs & constrain \\
        \hline
        \multirow{7}{*}{Film}
            & $^{238}$U series 1 & fix \\
            & $^{238}$U series 2 & free \\
            & $^{232}$Th series 1 & free \\
            & $^{232}$Th series 2 & free \\
            & $^{40}$K & constrain \\
            & $^{210}$Bi & free \\
        \hline
        \multirow{7}{*}{all}
            & Energy scale & constrained \\
            & $k_B, R$ & fix \\
            & LL-distortion & constrain \\
            & $^6$He & fix \\
            & $^{12}$B & fix \\
            & $^8$Li & fix \\
            & $^8$B & fix \\
        \hline
    \end{tabular}
\end{table}
The first penalty terms are on the rate of the constrained background sources. $^{238}$U series 2, $^{222}$Rn, $^{232}$Th series 2 are determined from the rate of coincidence tagged $^{214}$Bi-Po and $^{212}$Bi-Po events. Next, the independent background constraints determined by this study in \ref{chapter:Backgrounds}, $^{11}$C and $^{40}$K. $^{137}$Xe is determined from coincidence with MoGURA neutron captures, the determination of the rate that passes the coincident cut is also described in \ref{chapter:Backgrounds}. 

A penalty term is also included for the detector response parameters. An overall energy scale, scales all of the background and signal energy spectra via linear interpolation. The energy scale $\alpha_E=\frac{E_{vis}}{E_{sim}}$ is constrained to $1\pm 0.016$. The Birks constant, $k_B$, and Chrenkov Ratio, $R$, are kept fixed for stability in the fit. In later analyses, the constraints on them should either be tightened or their effect quantified as another systematic uncertainty. 

\section{\twonustar  Results}
\subsection{Best Fit Result}
The best fit result for the \twonustar decay rate as given by MINUIT is 0, no significant excess over the background expectation was found. Figure \ref{fig:bestfit} shows the fitted energy spectra. 

Table \ref{tab:fit_result} list the best fit parameter values. Constrained values are within their constraints with no significant biases, a notable deviation from previous analyses is the railing of $^{232}$Th S1 film to 0, this background is also present in the $^{232}$Th S2P film background. Without the radial component to differentiate the two, it is a superfluous parameter in the fit.
\begin{figure}[h]
	\centering
	\includegraphics[scale=0.4]{best_fit_thesis.png}
	\caption{Best fit energy spectrum to full KLZ800 dataset in the reduced FV.}
	\label{fig:bestfit}
\end{figure}

\begin{table}
    \centering
    \begin{tabular}{c|c|c}
        Parameter & Value & Fit Condition \\ \hline
        $^{12}$B Spallation &  0.016 & fix \\
        $^{8}$B Spallation &  0.239 & fix\\
        $^{210}$Bi (XeLS) &  26000 & floated\\
        $^{210}$Bi (film) &  22930 & floated\\
        $^{11}$C (XeLS) &  983 & constrained\\
        $^{136}$Cs (XeLS) &  0.8 & fix\\
        $^{6}$He Spallation &   0.33 & fix\\
        $^{40}$K (film) &  186 & constrained \\
        $^{85}$Kr (XeLS) & 41500 & floated \\
        $^{8}$Li Spallation & 0.525 & fix \\
        Long-lived (XeLS) & 0.68 & floated \\
        Monochromatic & 3.4e+07 & fix \\
        $^{222}$Rn (XeLS) & 8002.15 & constrained \\
        Signal ($2\nu\beta\beta^*$, XeLS) & 0 & floated \\
        Solar $\nu$ (XeLS) & 4.87 & fix \\
        $^{232}$Th S1 (film) & 0 & floated \\
        $^{232}$Th S2P (XeLS) & 114 & constrained \\
        $^{232}$Th S2P (film) & 98.7 & floated \\
        $^{238}$U S1 (film) & 25 & fix \\
        $^{238}$U S2 (XeLS) & 30.2 & constrained \\
        $^{238}$U S2 (film) & 61.3 & floated \\
        $^{136}$Xe $2\nu\beta\beta$ (XeLS) & 107601 & floated \\
        $^{137}$Xe (XeLS) & 0.980653 & constrained \\
        Energy Scale $\alpha$ (Internal) & 0.991 & constrained \\
        LL distortion & 0.118 & constrained \\
        $R$ (Internal) & 0.018 & fix \\
        $k_B$ (Internal) & 0.31 & fix
    \end{tabular}
    \label{tab:fit_result}
    \caption{The best fit parameter values and their fit conditions.}
\end{table}

\subsection{\twonustar \ Half-life Limits}
In lieu of a positive \twonustar signal, the upper limit on the rate can be determined from the 90\% confidence level (C.L.). We perform a scan over the \twonustar signal rate. The signal rate is fixed to various values and at each value, a new fit of the backgrounds is performed to the data. The worsening of the fit-data match is quantified by the $\Delta\chi^2$, the difference between the NLL test statistic from the best fit and the fit with a fixed signal rate. Figure \ref{fig:chi2scan} shows the scan results.

An estimate of the 90\% C.L. upper limit on the rate is determined by searching for the threshold value of $\Delta\chi^2=2.71$. This threshold is given by Wilks' theorem for the 90\% coverage of the test $\chi^2$ distribution for a single parameter.

\begin{figure}[h]
	\centering
	\includegraphics[scale=0.7]{chi2scan.png}
	\caption{Difference in the NLL test statistic over fixed signal rates, the intersection with the Wilks' Theorem 1-parameter 90\% C.L. threshold is shown.}
	\label{fig:chi2scan}
\end{figure}

This upper limit on the rate is simply converted to a lower limit on the half-life by the following formula:
\begin{equation}
    T_{1/2} = \frac{\ln(2)\times N_{Xe136}}{rate \times 365.2}
\end{equation}
Where $N_{Xe136}$ is the number of $^{136}$Xe atoms in the XeLS. This can be calculated from the concentration listed in Table \ref{tab:xenon_comp}. 
\begin{align}
    C_{Xe} &= \frac{\text{mass of Xe in total XeLS}}{\text{mass of total XeLS}} \\
           &= \frac{745~\text{kg}}{30.5~\text{m}^3 \times 780.13~\text{kg/m}^3} \times 100 \\
           &= 3.13 \pm 0.01~(\text{wt}\%)
\end{align}
\begin{align}
N_{^{136}\mathrm{Xe}} &= (\text{1 kton} \times \text{concentration of Xe in XeLS} \times \text{ratio of } ^{136}\mathrm{Xe}) \notag \\
    &\quad \times N_A / (\text{atomic mass number of mixed Xe}) \notag \\
    &= 3.13 \times 0.9085 / 135.80 \times 6.022 \times 10^{23} \notag \\
    &= (1.261 \pm 0.004) \times 10^{29}~(\text{kton-XeLS})^{-1}
\end{align}
Here, $N_A$ is Avogrado's number. From these formulae, the upper limit on \twonustar rate is converted to a lower limit on \twonustar half-life.

However, the 90\% C.L. given by Wilks' theorem is likely invalid for this analysis for two reasons. First, the signal rate parameter has a physical boundary at zero, violating the regularity conditions required by Wilks' theorem. Second, the $\chi^2$ profile is not quadratic and falls sharply as the rate approaches zero, indicating that asymptotic approximations may not hold. For these reasons, the Wilks' theorem result should be considered a preliminary and potentially aggressive limit. A later section describes a more accurate determination using the Feldman-Cousins technique.

\subsection{Feldman-Cousins Calculation}
Given that Wilks' theorem conditions are likely not satisfied for this analysis, the upper limit on the \twonustar rate is determined using the Feldman-Cousins (FC) approach. While Wilks' theorem would allow us to assume -2 log(likelihood ratio) follows a $\chi^2$ distribution asymptotically, the FC method instead constructs confidence intervals by simulating the test statistic distribution under various hypothesized signal rates. This allows us to obtain proper 90\% coverage intervals without relying on asymptotic approximations.

We perform the FC calculation by scanning over \twonustar signal rates. At each signal rate hypothesis, multiple Toy MC datasets are generated with the best fit background levels. For each toy MC dataset, the fit is performed and the $\chi^2$ test statistic value calculated. 

\subsection{Discussion}
The Feldman-Cousins limit reveals that without additional inputs, this analysis is unable to set a world-leading limit on \twonustar. Better characterization of detector response, and modeling of the film and KamLS backgrounds will allow us to use the full XeLS volume and improve the sensitivity of the analysis. 
% Comparison to EXO-200 and KLZ400 Limits
% Comparison to theoretical predictions