\chapter{Backgrounds}
\label{chapter:Backgrounds}
\thispagestyle{myheadings}

% set this to the location of the figures for this chapter. it may
% also want to be ../Figures/2_Body/ or something. make sure that
% it has a trailing directory separator (i.e., '/')!
\graphicspath{{5_Chapter_Backgrounds/Figures/}}

In the analysis for excited state decays in KamLAND-ZEN, the signal is searched for via energy spectrum fit. While chapter \ref{chapter:reco_select} described event selections applied in data to reduce backgrounds, this chapter describes how the background energy distributions are modeled and estimated. This chapter describes expected distributions of background events that pass those event selections. These distributions are implemented in the spectral fit of Chapter \ref{chapter:analysis}, along with any independent constraints that can be placed on their intensities.

While in other KamLAND-ZEN analyses, the spatial distribution of the events is used in distinguishing signals from backgrounds, due to issues in the radial distribution modeling, only the innermost part of the detector is used. Further discussion of the volume selection used for the excited state analysis can be found in Chapter \ref{chapter:analysis}
\section{\2nbb : Double-Beta Decay}
Two neutrino double-beta decay (\2nbb) is the largest background in the search for \2nbb$^*$ by far. The search for excited state decays is largely reduced to a search for a distortion in the \2nbb decay spectrum. 

In KamLAND-ZEN, two isotopes of Xenon undergo \2nbb, $^{134}$Xe and $^{136}$Xe. Of the Xenon dissolved in the XeLS, approximately 90\% is $^{136}$Xe and 9\% is $^{134}$Xe. $^{136}$Xe decays have been studied in many experiments, the rate of this background is allowed to float freely in this analysis. $^{134}$Xe decays have yet to be observed, the world-leading limit on this isotope's half-life is $T_{1/2}>8.2\times 10^{20}$ years at 90\% C.L.. This rate corresponds to an event rate of $2.7\times 10^4$ events/day/kton in KamLAND-ZEN's XeLS. 

While $^{134}$Xe decay could be a secondary physics goal of KamLAND-ZEN, in practice, the signal is completely masked by residual $^{85}$Kr and $^{210}$Bi contamination in the detector. In addition, theoretical predictions place the expected $^{134}$Xe half-life around 3 orders of magnitude higher than $^{136}$Xe. For these reasons, $^{134}$Xe is neglected in this analysis.

\section{Radioactive Contamination}
\subsection{$^{238}$U Series}
$^{238}$U is a naturally occurring radionuclide with half-life $T_{1/2}=4.468\times 10^9$ years. $^{238}$U is responsible for 40\% of the radioactive heat generated in the earth. As an omnipresent radioactive contaminant it is present in all components of the detector to some degree. 

The decay series of Uranium includes U, Th, Pa, Ra, Rn, Po, At, Pb, Bi, and Tl. All of the radioactive decays with energy spectra above 0.5 MeV are included in the background model. A distinction is made between the decays before $^{222}$Rn, henceforth referred to as "Series 1", and after $^{222}$Rn, henceforth referrred to as "Series 2". $^{238}$Us1 enters the detector as $^{238}$U as radioactive contamination from construction. Thus, all the $^{238}$Us1 decays are in specular equilibrium with each other. $^{238}$Us2 is introduced to the detector as $^{222}$Rn during the original introduction of Xenon into the XeLS. In this and most KamLAND-ZEN analyses, the rates of $^{238}$Us1 and $^{238}$Us2 are fitted independently from each other. $^{238}$Us1 is floated freely and is determined from the spectral fit, while $^{238}$Us2 is constrained by the rate of tagged $^{214}$Bi decays via the delayed-coincidence method described in Chapter \ref{chapter:reco_select}. A penalty term is added to the fit likelihood to implement this constraint.
\subsection{$^{232}$Th Series}
$^{232}$Th is another omnipresent radioactive contaminant on earth. Just as in the $^{238}$U model, the $^{232}$Th decay chain is split into 2 series. The decays upstream of $^{228}$Th are henceforth referred to as "Series 1", while the decays downstream of $^{228}$Th are henceforth referred to as "Series 2". Once again the lower energy threshold of 0.5 MeV is used.

The rates of both decay chains are fitted independently in the energy spectral fit. It is worth noting that in the higher-energy window of 3-5 MeV, the dominant contributor is $^{208}$Tl decay. Thus, the rate of $^{208}$Tl can be comparatively well determined by the spectral fit, and the other backgrounds in the decay chain can be determined by the rate of $^{208}$Tl decays in KamLAND-ZEN.
\subsection{$^{40}$K}
$^{40}$K is another omnipresent, common radionuclide on earth. It decays via $beta^-$ decay with 89.28\% B.R. and electron capture with 10.72\% B.R. with a half-life of $T_{1/2}=1.28\times 10^9$ years. Like the other backgrounds it is assumed to be present in all the detector materials including XeLS, balloon film, and KamLS. But due to the high level of liquid scintillator purity, it is expected that the contamination of $^{40}$K in the liquid scintillator volumes to be negligible and that the $^{40}$K background to be concentrated on the inner balloon film. In this analysis a special study was done to estimate the amount of film $^{40}$K is present in KamLAND-ZEN.

The most important feature of $^{40}$K decay with regard to the analysis for excited state decays of $^{136}$Xe is the $^{40}$K electron capture gamma peak, $Q=1.5$ MeV. In the KamLS, outside the inner balloon, this peak is prominent enough to resolve on its own.

In order to estimate the amount of film-related $^{40}$K background that gets reconstructed in the innermost XeLS volumes, the radial distribution outside the balloon is measured. The volumes just inside and just outside the inner balloon were separated into equal volume spherical shells. The data in each of these spherical shells was independently fitted with a simplified version of the full KamLAND-ZEN spectral fit. The purpose of each of these fits is to measure the prominence of the $^{40}$K electron-capture gamma peak over the broad \2nbb background. The trend of the individually-fitted rates of $^{40}$K in each of the spherical "volume bins" is compared to the expected radial distribution from KLG4Sim simulations of $^{40}$K. A simple least squares "fit of fits" is performed to determine the measured rate of $^{40}$K from the inner balloon film. 

While an initial deviation from the expected distribution was found, vetoing the upper balloon neck region yielded an excellent qualitative match between the measured and simulated $^{40}$K radial distributions. This independent measurement of the radial distribution supports the hypothesis that the $^{40}$K contamination is mostly limited to inner balloon film itself and much lower in the liquid scintillator volumes. This behavior matches the relative contamination levels of  $^{238}$U and $^{232}$Th.

This indepedently measured rate of $^{40}$K is implemented in the spectral analysis as a penalty term.

\subsection{$^{85}$Kr}

\subsection{Sequential Th Series Decays}

\section{Carbon Spallation}
\subsection{Short-Lived Spallation Products}
\subsection{$^{11}C$ Spallation Estimation}

\section{Long-Lived Xenon Spallation Products}
second most dominant bacvkground in the \0nbb ROI. Long lived spallation affects the resolution of the \2nbb endpoint which affects the energy scale measurement and the rate of \2nbb overall.
\subsection{FLUKA Simulation}
\subsection{ENSDF Database}
\subsection{Long-Lived Spallation Veto}
\subsection{Efficiency and Error Estimation}
\subsection{Spectrum Distortion}

\section{Other Backgrounds}
\subsection{Solar Neutrinos}





Time to get philosophical and wordy.\cite{blackbox}