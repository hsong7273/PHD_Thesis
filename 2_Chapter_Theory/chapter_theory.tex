\chapter{Theory of Neutrinos and Double Beta Decay}
\label{chapter:theory}
\thispagestyle{myheadings}

\graphicspath{{2_Chapter_Theory/Figures/}}

While this chapter reviews the theoretical foundations of neutrino mass and lepton number violation, particular emphasis is placed on Standard Model two-neutrino double beta decay ($2\nu\beta\beta$). In addition to serving as an irreducible background to neutrinoless double beta decay ($0\nu\beta\beta$) searches, $2\nu\beta\beta$ to excited nuclear states provides a unique experimental probe of nuclear structure that directly informs the interpretation of $0\nu\beta\beta$ results.


\section{Neutrinos in the Standard Model}

Neutrinos remain the least understood component of the Standard Model (SM) of particle physics~\cite{lamport1985:latex}. Their elusive nature and extremely weak interactions make them challenging to study, yet they play a central role in both particle physics and cosmology. The modern understanding of neutrinos began in 1914, when James Chadwick used magnetic spectrometry to measure the energy spectrum of electrons emitted in beta decay. He observed that the spectrum was continuous rather than discrete, implying an apparent violation of energy conservation.

To resolve this puzzle, Pauli postulated in 1930 the existence of a new neutral and very light particle that carried away the missing energy~\cite{Pauli:1930pc}. He introduced this idea in his famous letter addressed to the ``Radioactive Ladies and Gentlemen.'' Enrico Fermi later incorporated Pauli’s proposal into his theory of beta decay and named the particle the neutrino, meaning ``little neutral one.'' The neutrino was experimentally detected in 1956 by Cowan and Reines~\cite{cowan1956}, firmly establishing its existence. Since that time, the Standard Model has been extended to include three flavors of neutrinos, each associated with a corresponding charged lepton.

The Standard Model is a gauge theory based on the symmetry group
$SU(3)_C \times SU(2)_L \times U(1)_Y$~\cite{lamport1985:latex}. Neutrinos participate only in the weak interaction, which is mediated by the charged $W^\pm$ bosons and the neutral $Z^0$ boson, and they carry no electric charge. Their extremely small interaction cross sections make them difficult to detect, but also allow them to propagate over vast distances with little attenuation. This unique property enables neutrinos to serve as powerful messengers from otherwise inaccessible regions of the universe.

\subsection{Neutrino Interactions}

The Standard Model unifies the strong, weak, and electromagnetic interactions within the gauge symmetry
$SU(3)_C \times SU(2)_L \times U(1)_Y$. The $SU(3)_C$ sector governs the strong interaction through quantum chromodynamics, while the $SU(2)_L \times U(1)_Y$ sector describes the electroweak interaction. In this framework, the weak interaction is mediated by the charged $W^\pm$ bosons and the neutral $Z^0$ boson.

Neutrinos appear in the Standard Model as components of left handed lepton doublets, which transform as weak isospin doublets under $SU(2)_L$:
\begin{equation}
    L_\ell =
    \begin{pmatrix}
        \nu_{\ell L} \\
        \ell_L
    \end{pmatrix},
    \quad \ell = e, \mu, \tau .
\end{equation}
Here, $\nu_{\ell L}$ and $\ell_L$ denote the neutrino and charged lepton fields of flavor $\ell$, respectively. Only the left handed components of these fermion fields participate in weak interactions. This chiral structure is implemented through the projection operator:
\begin{equation}
    P_L = \frac{1 - \gamma_5}{2},
\end{equation}
where $\gamma_5 = i\gamma^0\gamma^1\gamma^2\gamma^3$ is constructed from the Dirac matrices.

Electroweak interactions are characterized by two quantum numbers: weak isospin $I$ and weak hypercharge $Y$. The electric charge operator is given by:
\begin{equation}
    Q = I_3 + \frac{Y}{2},
    \label{eq:charge}
\end{equation}
where $I_3$ is the third component of weak isospin. For lepton doublets, the total weak isospin is $I = 1/2$ and the hypercharge is $Y = -1$. These assignments correctly reproduce the observed electric charges, yielding $Q = 0$ for neutrinos and $Q = -1$ for charged leptons.

\renewcommand{\arraystretch}{1.2}
\setlength{\tabcolsep}{8pt}
\setlength{\extrarowheight}{2pt}

\begin{table}[b!]
\centering
\begin{tabular}{lclcccc}
\hline
 & & & $I$ & $I_3$ & $Y$ & $Q$ \\
\hline

\multirow{2}{*}{lepton doublet} &
\multirow{2}{*}{$L_L \equiv$} &
\rule{0pt}{4ex}\multirow{2}{*}{$\left(\begin{array}{c}
\nu_{eL} \\
e_L
\end{array}\right)$}
& $1/2$ & $+1/2$ & $-1$ & $0$ \\
& & & $1/2$ & $-1/2$ & $-1$ & $-1$ \\

lepton singlet & $e_R$ & & $0$ & $0$ & $-2$ & $-1$ \\

\multirow{2}{*}{quark doublet} &
\multirow{2}{*}{$Q_L \equiv$} &
\multirow{2}{*}{$\left(\begin{array}{c}
u_L \\
d_L
\end{array}\right)$}
& $1/2$ & $+1/2$ & $1/3$ & $2/3$ \\
& & & $1/2$ & $-1/2$ & $1/3$ & $-1/3$ \\

quark singlets & $u_R$ & & $0$ & $0$ & $4/3$ & $2/3$ \\
& $d_R$ & & $0$ & $0$ & $-2/3$ & $-1/3$ \\
\hline
\end{tabular}
\caption{Weak isospin $I$, third component of weak isospin $I_3$, hypercharge $Y$, and electric charge $Q = I_3 + Y/2$ for fermion doublets and singlets in the Standard Model.}
\label{tab:weakisospin}
\end{table}





Table~\ref{tab:weakisospin} summarizes the weak isospin, hypercharge, and electric charge assignments for the fermion doublets and singlets in the Standard Model. Right handed charged leptons and quarks are singlets under $SU(2)_L$, with $I = 0$, and therefore do not participate in charged weak interactions. Their hypercharge values are chosen to reproduce the observed electric charges through Eq.~\ref{eq:charge}.

The neutrino components of the lepton doublets are referred to as active neutrinos, reflecting their participation in weak interactions. In contrast, hypothetical sterile neutrinos would be singlets under the full Standard Model gauge group and would not couple to the $W^\pm$ or $Z^0$ bosons. Within the Standard Model, there is exactly one active neutrino associated with each charged lepton flavor: $e$, $\mu$, and $\tau$.

Gauge invariance under $SU(2)_L$ dictates the form of the weak charged current and neutral current interactions involving leptons. These interactions are described by the Lagrangian terms:
\begin{align}
    -\mathcal{L}_{\mathrm{CC}} &=
    \frac{g}{\sqrt{2}}
    \sum_{\ell}
    \bar{\nu}_{\ell L} \gamma^\mu \ell_L W_\mu^+
    + \mathrm{h.c.}, \\
    -\mathcal{L}_{\mathrm{NC}} &=
    \frac{g}{2 \cos \theta_W}
    \sum_{\ell}
    \bar{\nu}_{\ell L} \gamma^\mu \nu_{\ell L} Z_\mu^0 ,
    \label{eq:NC}
\end{align}
where $g$ is the weak coupling constant and $\theta_W$ is the Weinberg angle. The charged current interaction governs processes such as beta decay and double beta decay, while the neutral current interaction allows neutrinos to scatter elastically from matter without changing flavor.

Precision measurements of the invisible decay width of the $Z^0$ boson provide a direct constraint on the number of light, active neutrino species~\cite{Zdecay}. The experimentally measured value,
\begin{equation}
    N_\nu = 2.984 \pm 0.008,
\end{equation}
is consistent with three active neutrino flavors and provides strong experimental support for the Standard Model neutrino sector.

The purely left handed nature of weak interactions in the Standard Model has important consequences for neutrino mass and for processes that violate lepton number. Because only left handed neutrino fields appear in the electroweak Lagrangian, no renormalizable mass term for neutrinos can be constructed using Standard Model fields alone. As a result, neutrinos are massless in the minimal Standard Model. Any mechanism that generates neutrino mass must therefore extend the theory, either by introducing new fields or by allowing higher dimensional operators. This chiral structure also plays a central role in double beta decay. In particular, the connection between left handed weak currents, neutrino mass, and lepton number violation underlies the theoretical interpretation of both $2\nu\beta\beta$ and $0\nu\beta\beta$ decay processes, which are discussed in detail in the following sections.

\section{Neutrino Oscillations}

The discovery of neutrino oscillations represents one of the most significant breakthroughs in particle physics in recent decades. This achievement was recognized with the 2015 Nobel Prize in Physics, awarded to Art McDonald of the SNO collaboration and Takaaki Kajita of the Super-Kamiokande collaboration~\cite{nuoscnobel}. The underlying concept of neutrino flavor oscillations was first proposed by Bruno Pontecorvo in the late 1950s, inspired by the phenomenon of neutral kaon mixing, $K^0 \leftrightarrow \overline{K}^0$~\cite{Pontecorvo:1957cp}. Pontecorvo suggested that neutrinos, like kaons, could change identity as they propagate, provided that the states produced in weak interactions were not identical to the states of definite mass.

Neutrino oscillations arise from the misalignment between flavor eigenstates and mass eigenstates. In a weak interaction, a neutrino is produced in a definite flavor state, associated with a charged lepton of the same flavor. However, the flavor eigenstates $\ket{\nu_\alpha}$, with $\alpha = e, \mu, \tau$, are quantum superpositions of mass eigenstates $\ket{\nu_k}$, where $k = 1, 2, 3$:
\begin{equation}
 \ket{\nu_{\alpha}} = \sum_k U^*_{\alpha k} \ket{\nu_k}.
 \label{flavorstate}
\end{equation}

\noindent This relationship may also be written in matrix form as:
\begin{equation}
	\begin{pmatrix}
		\nu_e \\
		\nu_\mu \\
		\nu_\tau
	\end{pmatrix}
	=
	\begin{pmatrix}
		U_{e1} & U_{e2} & U_{e3} \\
		U_{\mu1} & U_{\mu2} & U_{\mu3} \\
		U_{\tau1} & U_{\tau2} & U_{\tau3}
	\end{pmatrix}
	\begin{pmatrix}
		\nu_1 \\
		\nu_2 \\
		\nu_3
	\end{pmatrix},
	\label{unitary}
\end{equation}
where the coefficients $U_{\alpha k}$ are elements of the Pontecorvo-Maki-Nakagawa-Sakata (PMNS) matrix.

The PMNS matrix is parameterized by three mixing angles, $\theta_{12}$, $\theta_{23}$, and $\theta_{13}$, a Dirac charge parity violating phase $\delta_{CP}$, and two additional phases $\xi_1$ and $\xi_2$ that appear if neutrinos are Majorana particles. A commonly used parameterization of the PMNS matrix is:
\begin{multline}
U =
	\begin{pmatrix}
	1 & 0 & 0 \\
    0 & \cos \theta_{23} & \sin \theta_{23} \\
	0 & -\sin \theta_{23} & \cos \theta_{23}
	\end{pmatrix}
	\begin{pmatrix}
	\cos \theta_{13} & 0 & \sin \theta_{13} e^{-i\delta_{CP}} \\
	0 & 1 & 0 \\
	-\sin \theta_{13} e^{i\delta_{CP}} & 0 & \cos \theta_{13}
	\end{pmatrix} \\
	\times
	\begin{pmatrix}
	\cos \theta_{12} & \sin \theta_{12} & 0 \\
	-\sin \theta_{12} & \cos \theta_{12} & 0 \\
	0 & 0 & 1
	\end{pmatrix}
	\begin{pmatrix}
	1 & 0 & 0 \\
	0 & e^{i\xi_1} & 0 \\
	0 & 0 & e^{i\xi_2}
	\end{pmatrix}.
	\label{pmns}
\end{multline}

\noindent The final diagonal matrix containing the Majorana phases does not affect neutrino oscillation probabilities, as these phases cancel when forming the inner products relevant for flavor transitions. Nevertheless, they play a crucial role in lepton number violating processes such as $0\nu\beta\beta$ decay and therefore remain of central interest in neutrino physics.

To illustrate how neutrino oscillation parameters are extracted experimentally, it is instructive to derive the oscillation probability in vacuum. Unlike quarks, which are confined within hadrons, neutrinos propagate freely over macroscopic distances. The massive neutrino states $\ket{\nu_k}$ can therefore be treated as plane wave solutions to the Schrödinger equation, evolving in time as:
\begin{equation}
    \ket{\nu_k(t)} = e^{-iE_k t} \ket{\nu_k},
    \qquad
    E_k = \sqrt{m_k^2 + \vec{p}^{\,2}}.
    \label{pw}
\end{equation}

\noindent A neutrino produced at time $t = 0$ in a flavor state $\ket{\nu_\alpha}$ evolves as a coherent superposition of mass eigenstates,
\begin{equation}
	\ket{\nu_{\alpha}(t)} = \sum_k U^*_{\alpha k} e^{-iE_k t} \ket{\nu_k}.
	\label{trans}
\end{equation}

\noindent Using the unitarity of the PMNS matrix, $U^\dagger U = \mathbb{1}$, this expression may be rewritten in the flavor basis as:
\begin{equation}
\ket{\nu_{\alpha}(t)} =
\sum_{\beta = e, \mu, \tau}
\left(
\sum_k U^*_{\alpha k} e^{-iE_k t} U_{\beta k}
\right)
\ket{\nu_\beta}.
\label{flavevolve}
\end{equation}

\noindent The probability that a neutrino produced in flavor state $\nu_\alpha$ is later detected as flavor $\nu_\beta$ is then given by:
\begin{equation}
P(\nu_\alpha \rightarrow \nu_\beta)
=
|\braket{\nu_\beta | \nu_\alpha(t)}|^2
=
\sum_{k,j}
U^*_{\alpha k} U_{\beta k}
U_{\alpha j} U^*_{\beta j}
e^{-i(E_k - E_j)t}.
\label{tp}
\end{equation}

\noindent For ultra relativistic neutrinos, where $m_k \ll |\vec{p}|$, the energy may be expanded as:
\begin{equation}
E_k \simeq E + \frac{m_k^2}{2E},
\label{dispapprox}
\end{equation}
leading to
\begin{equation}
E_k - E_j = \frac{\Delta m^2_{kj}}{2E},
\qquad
\Delta m^2_{kj} \equiv m_k^2 - m_j^2.
\label{obs}
\end{equation}

\noindent Since oscillation experiments measure the source detector separation $L$ rather than the propagation time, the approximation $t \simeq L$ may be used. The oscillation probability then becomes:
\begin{equation}
P(\nu_\alpha \rightarrow \nu_\beta)
=
\sum_{k,j}
U^*_{\alpha k} U_{\beta k}
U_{\alpha j} U^*_{\beta j}
e^{-i \Delta m^2_{kj} L / 2E}.
\label{tp2}
\end{equation}

\noindent Separating the real and imaginary components yields the familiar form:
\begin{multline}
P(\nu_\alpha \rightarrow \nu_\beta)
=
\delta_{\alpha\beta}
- 4 \sum_{k>j}
\mathfrak{Re}
\left[
U^*_{\alpha k} U_{\beta k}
U_{\alpha j} U^*_{\beta j}
\right]
\sin^2 \left( \frac{\Delta m^2_{kj} L}{4E} \right) \\
+ 2 \sum_{k>j}
\mathfrak{Im}
\left[
U^*_{\alpha k} U_{\beta k}
U_{\alpha j} U^*_{\beta j}
\right]
\sin \left( \frac{\Delta m^2_{kj} L}{2E} \right).
\label{tp3}
\end{multline}

\noindent The oscillation amplitudes are governed by the elements of the PMNS matrix, while the oscillation frequency is set by the ratio $\Delta m^2_{kj} L / E$. In practical units, this phase may be written as:
\begin{equation}
\frac{\Delta m^2_{kj} L}{2E}
\approx
1.27
\frac{\Delta m^2_{kj} \, [\mathrm{eV}^2] \, L \, [\mathrm{km}]}
{E \, [\mathrm{GeV}]}.
\label{oscfrequency}
\end{equation}

Within the past two decades, the majority of neutrino oscillation parameters have been measured with impressive precision. The three mixing angles $\theta_{12}$, $\theta_{23}$, and $\theta_{13}$, along with the two independent squared mass splittings $\Delta m^2_{21}$ and $|\Delta m^2_{31}|$, are now known to the level of a few percent or better. These measurements have been achieved using a diverse set of experiments that study neutrinos originating from the Sun, the Earth’s atmosphere, nuclear reactors, and particle accelerators.

Despite this progress, two fundamental questions remain unresolved within the oscillation framework. The first concerns the value of the charge parity violating phase $\delta_{CP}$, which governs potential differences between neutrino and antineutrino oscillation probabilities. The second is the ordering of the neutrino mass eigenstates, commonly referred to as the neutrino mass hierarchy.

Because oscillation experiments are sensitive only to differences in squared masses, they cannot determine the absolute neutrino mass scale. As a result, two distinct mass orderings remain consistent with current data. In the normal ordering scenario, the third mass eigenstate is the heaviest, with $m_3 > m_2 > m_1$. In the inverted ordering scenario, the third mass eigenstate is the lightest, with $m_2 > m_1 > m_3$. These two possibilities correspond to opposite signs of the atmospheric mass splitting, $\Delta m^2_{31}$ or $\Delta m^2_{32}$, which current experiments have not yet been able to determine conclusively.

Table~\ref{tab:osc_pars} summarizes the current best fit values and one standard deviation uncertainties for the neutrino oscillation parameters, reproduced from the NuFIT version 6.0 global analysis~\cite{Esteban_2024}. This fit incorporates data from a wide range of experiments, including atmospheric neutrino measurements from Super-Kamiokande, and provides results for both normal and inverted mass orderings. The solar mass splitting $\Delta m^2_{21}$ is common to both orderings, while the atmospheric mass splitting $\Delta m^2_{3k}$ differs in sign depending on the assumed hierarchy.  In Table~\ref{tab:osc_pars}, the notation $\Delta m^2_{\mathrm{sol}}$ refers to $\Delta m^2_{21}$, while $\Delta m^2_{\mathrm{atm}}$ denotes either $\Delta m^2_{31}$ or $\Delta m^2_{32}$, depending on the ordering. This convention reflects the historical sensitivity of solar neutrino experiments to $\Delta m^2_{21}$ and of atmospheric neutrino experiments to $|\Delta m^2_{31}| \approx |\Delta m^2_{32}|$.

\renewcommand{\arraystretch}{1.0}
\setlength{\tabcolsep}{8pt}

\begin{table}[t!]
  \centering 
  %\begin{threeparttable}
    \begin{tabular}{ccc}
    %{m{15mm} m{70mm} m{18mm}}
    \midrule
    Oscillation parameter & Normal & Inverted\\
    \midrule\midrule
    $\Delta m_{21}^2 [10^{-5}\, \mathrm{eV^2}]$  &  $7.49^{+0.19}_{-0.19}$ &    $7.49^{+0.19}_{-0.19}$ \\
     & & \\
    $\Delta m_{3k}^2 [10^{-3}\, \mathrm{eV^2}]$  &  $+2.513^{+0.021}_{-0.019}$ &    $-2.484^{+0.020}_{-0.020}$ \\
     & & \\
    $\sin^2{\theta_{12}}$  &  $0.308^{+0.012}_{-0.011}$ &    $0.308^{+0.012}_{-0.011}$ \\
     & & \\
    $\sin^2{\theta_{23}}$  &  $0.470^{+0.17}_{-0.13}$ &    $0.550^{+0.012}_{-0.015}$ \\
     & & \\
    $\sin^2{\theta_{13}}$  &  $0.02215^{+0.00056}_{-0.00058}$ &    $0.02231^{+0.00056}_{-0.00056}$ \\
     & & \\
    $\delta_{CP}[^{\circ}]$  &  $212^{+26}_{-41}$ &    $274^{+22}_{-25}$ \\
    \midrule
    \end{tabular}
    %\end{threeparttable}
    \caption[Best-fit values $\pm 1\sigma$ from a global analysis of neutrino oscillation parameters.]{Best-fit values $\pm 1\sigma$ from a global analysis of neutrino oscillation parameters reproduced from NuFIT version 6.0 in Reference~\cite{Esteban_2024}.  Note that $\Delta m^2_{3k} \equiv \Delta m^2_{31} >0$ for normal ordering and $\Delta m^2_{3k} \equiv \Delta m^2_{32} <0$ for inverted ordering. }
     \label{tab:osc_pars}
\end{table}

Future experiments are expected to resolve the remaining ambiguities in the oscillation framework. Medium baseline reactor experiments such as JUNO~\cite{Paoloni:2024atc} aim to determine the mass ordering through precision measurements of oscillation interference effects. Long baseline accelerator experiments, including Hyper-Kamiokande~\cite{AliAjmi:2024xus} and DUNE~\cite{Gil-Botella:2024duf}, are designed to probe both the mass ordering and the value of $\delta_{CP}$ through detailed studies of neutrino and antineutrino appearance channels.

Equation~\ref{tp3} shows that neutrino oscillation experiments are sensitive only to differences in the squared neutrino masses, $\Delta m^2_{jk}$, and provide no information on the absolute values of the individual mass eigenstates. As a result, determining the absolute neutrino mass scale requires experimental approaches that are complementary to oscillation measurements.

One such approach is pursued by the KATRIN experiment, which directly probes the kinematics of $\beta$ decay. KATRIN measures the energy spectrum of electrons emitted in the decay of tritium, which has a $Q$-value of 18.6\,keV. If neutrinos are massive, a small but measurable fraction of the available decay energy is carried away by the neutrino. Precise measurements of the endpoint of the electron energy spectrum therefore place a constraint on the effective mass of the electron flavor neutrino, which is a superposition of the neutrino mass eigenstates,
\begin{equation}
    m_{\nu_e} = \sqrt{\sum_{i} |U_{ei}|^2 m_i^2}.
\end{equation}

\noindent Achieving sensitivity to this quantity is experimentally challenging and requires sub electron volt energy resolution near the endpoint of the beta decay spectrum. The most stringent direct limit to date has been set by the KATRIN experiment, which reports $m_{\nu_e} < 0.8$\,eV at 90\% confidence level~\cite{KATRIN:2021uub}. Future experiments, such as Project~8, aim to further improve this sensitivity using Cyclotron Radiation Emission Spectroscopy, a technique that measures the frequency of radiation emitted by beta decay electrons spiraling in a magnetic field~\cite{PhysRevLett.131.102502}.

An alternative, indirect probe of neutrino masses is provided by cosmological observations. Neutrinos influence the formation and evolution of large scale structure in the universe due to their relativistic nature in the early universe and their contribution to the total matter density at later times. Measurements of the Cosmic Microwave Background, Baryon Acoustic Oscillations, and Redshift Space Distortions can therefore be combined to constrain the sum of the neutrino mass eigenstates, $\sum m_i = m_1 + m_2 + m_3$. Currently, the strongest cosmological constraint yields an upper limit of $\sum m_i < 0.09$\,eV at 95\% confidence level~\cite{PhysRevD.104.083504}.

A third and potentially most sensitive approach to determining the absolute neutrino mass scale involves $0\nu\beta\beta$ decay. Observation of this decay would not only provide access to an effective neutrino mass parameter, but would also demonstrate the violation of lepton number and establish the Majorana nature of neutrinos. Before discussing aspects of $2\nu\beta\beta$ and $0\nu\beta\beta$ decay, we will first review the theoretical framework of neutrino mass generation that motivates and underpins these searches.



\section{Neutrino Mass}
The observation of neutrino oscillations establishes that neutrinos possess nonzero masses and that flavor eigenstates are superpositions of mass eigenstates. Despite this, the underlying mechanism responsible for neutrino mass generation remains unknown. Numerous extensions of the Standard Model have been proposed to explain this phenomenon, as discussed below.

Here we follow the derivation presented in Ref.~\cite{giunti2007}. Landau, Lee and Yang, and independently Salam, showed that a massless fermion can be consistently described by a chiral field within a two-component theory of massless neutrinos. We begin with the Dirac equation:
\begin{equation}
    (i\gamma^\mu \partial_\mu - m)\psi = 0,
\end{equation}
for a fermion field $\psi$. Decomposing the Dirac spinor into its left- and right-handed chiral components:
\begin{equation}
    \psi = \psi_L + \psi_R,
\end{equation}
where $\psi_{L,R} = P_{L,R}\psi$ and $P_{L,R} = \frac{1}{2}(1 \mp \gamma^5)$ are the chiral projection operators, the Dirac equation can be written as the coupled system:
\begin{equation}
    i\gamma^\mu \partial_\mu \psi_L = m \psi_R,
    \label{eq:left_weyl}
\end{equation}
\begin{equation}
    i\gamma^\mu \partial_\mu \psi_R = m \psi_L,
    \label{eq:right_weyl}
\end{equation}
where the space-time evolution of the left- and right-handed fields is coupled through the mass term $m$.

In the massless limit, $m = 0$, Eqs.~\eqref{eq:left_weyl} and \eqref{eq:right_weyl} decouple:
\begin{equation}
    i\gamma^\mu \partial_\mu \psi_L = 0,
\end{equation}
\begin{equation}
    i\gamma^\mu \partial_\mu \psi_R = 0.
\end{equation}
In this case, a massless fermion may be fully described by a single chiral field (either left-handed or right-handed) which contains only two independent degrees of freedom. These equations are known as the Weyl equations, and the corresponding spinors $\psi_L$ and $\psi_R$ are referred to as Weyl spinors.

The minimal formulation of the Standard Model adopts this two-component description for neutrinos, treating them as massless fermions. In this framework, the neutrino is described entirely by a left-handed Weyl spinor, $\nu_L$, which participates in the weak interaction, while no right-handed neutrino field, $\nu_R$, is included.


\subsection{Dirac Masses}
% If right-handed neutrino fields $\nu_{R}$ exist, a Dirac mass term, just like the one for the charged leptons, can be written:
% \begin{equation}
%     -\mathcal{L}_{\text{Dirac}} = Y_{ij}^\nu \overline{L}_{Li} \tilde{\Phi} \nu_{Rj} + \text{h.c.}
% \end{equation}
% where $\tilde{\Phi} = i \sigma_2 \Phi^*$. This yields Dirac masses $m^\nu_{ij} = Y^\nu_{ij} v/\sqrt{2}$. However, the tiny observed neutrino masses ($m_\nu < 1$ eV) would require $Y^\nu_{ij} < 10^{-12}$. The huge discrepancy between the neutrino masses and the other fermions imply the existence of some underlying mechanism which suppresses the neutrino masses. In the absence of such explanation, the light neutrino masses bring up a naturalness problem. Many neutrino mass models have been proposed that produce light neutrino masses via a more natural mechanism. 

If right-handed neutrino fields $\nu_R$ exist, neutrinos may acquire mass through a Dirac mass term analogous to those of the charged leptons. In this case, a Yukawa interaction can be written as:
\begin{equation}
    -\mathcal{L}_{\text{Dirac}} =
    Y_{ij}^\nu \, \overline{L}_{Li} \, \tilde{\Phi} \, \nu_{Rj}
    + \text{h.c.},
\end{equation}
where $L_{Li}$ is the left-handed lepton doublet of generation $i$,
$\tilde{\Phi} = i\sigma_2 \Phi^*$ is the conjugate Higgs doublet, and
$Y^\nu_{ij}$ are the neutrino Yukawa couplings. After electroweak symmetry breaking, when the Higgs field acquires a vacuum expectation value $v$, this interaction generates Dirac neutrino masses:
\begin{equation}
    m^\nu_{ij} = \frac{v}{\sqrt{2}} \, Y^\nu_{ij}.
\end{equation}

This mechanism is entirely analogous to mass generation for the charged leptons and quarks in the Standard Model, where fermion masses arise through Yukawa couplings to the Higgs field following spontaneous symmetry breaking of the $SU(2)_L \times U(1)_Y$ gauge symmetry. In the unitary gauge, the Higgs doublet may be written as:
\begin{equation}
    \Phi = \frac{1}{\sqrt{2}}
    \begin{pmatrix}
        0 \\
        v + h
    \end{pmatrix},
\end{equation}
where $h$ denotes the physical Higgs boson. Coupling this field (or its conjugate) to left- and right-handed fermion fields yields Dirac mass terms proportional to the Higgs vacuum expectation value, as well as Higgs–fermion interaction terms.

Applying this same mechanism to neutrinos, however, presents two conceptual difficulties. First, right-handed neutrino fields are absent from the minimal Standard Model and must be introduced as gauge-singlet states. Such fields do not participate in the weak interaction and are therefore often referred to as \emph{sterile} neutrinos. The inclusion of $\nu_R$ thus constitutes a minimal extension of the Standard Model.

Second, even if right-handed neutrinos exist and neutrinos are purely Dirac fermions, the observed smallness of neutrino masses poses a naturalness problem. Current experimental constraints require neutrino masses to be below the eV scale, implying Yukawa couplings:
\begin{equation}
    Y^\nu_{ij} \lesssim 10^{-12},
\end{equation}
which are many orders of magnitude smaller than those of the charged fermions. This extreme hierarchy is difficult to justify within the Standard Model framework and stands in sharp contrast to the mass spectrum of the other fermions.

The striking disparity between neutrino masses and those of the charged leptons and quarks strongly suggests the presence of an underlying mechanism that suppresses neutrino masses relative to the electroweak scale. This observation has motivated a wide class of neutrino mass models that extend the Standard Model and generate light neutrino masses in a more natural way. Several such mechanisms are discussed in the following sections.


\subsection{Majorana Neutrino Mass}
Since neutrinos have been experimentally shown to possess nonzero masses, the two-component theory of massless neutrinos is no longer sufficient. In 1937, Majorana proposed an alternative formulation of the Dirac equation in which a massive fermion can be described by a single spinor field rather than independent left- and right-handed components. The key assumption of the Majorana construction is that the right-handed field is not independent, but instead related to the left-handed field by charge conjugation:
\begin{equation}
    \psi_R = C \, \overline{\psi_L}^{\,T},
\end{equation}
where $C$ is the charge conjugation matrix.

Using the properties of the charge conjugation operator and the chiral projection operators, one finds that:
\begin{equation}
    P_L \big( C \, \overline{\psi_L}^{\,T} \big) = 0,
\end{equation}
which demonstrates that the charge-conjugated left-handed field transforms as a right-handed field. In other words, charge conjugation converts a left-handed Weyl spinor into a right-handed one.

With this identification, the fermion field may be written entirely in terms of a single chiral component:
\begin{equation}
    \psi = \psi_L + \psi_L^C,
\end{equation}
where $\psi_L^C \equiv C \overline{\psi_L}^{\,T}$. The corresponding equation of motion takes the form:
\begin{equation}
    i \gamma^\mu \partial_\mu \psi = m \psi^C,
\end{equation}
where $\psi^C$ denotes the charge-conjugated field. This leads directly to the Majorana condition:
\begin{equation}
    \psi = \psi^C,
    \label{eq:majorana}
\end{equation}
which implies that the fermion is identical to its antiparticle.

Equation~\eqref{eq:majorana} can only be satisfied by electrically neutral fermions. Among the known elementary fermions, only neutrinos meet this criterion, making them unique candidates for Majorana particles. Since neutrinos interact solely through the weak interaction, the overall charge parity of the neutrino field has no observable consequence and may be chosen arbitrarily.

If neutrinos are Majorana particles, neutrinos and antineutrinos are not distinct states but differ only by their helicities. By convention, negative-helicity states are referred to as neutrinos, while positive-helicity states are referred to as antineutrinos.

In the Standard Model framework, the simplest Majorana mass term that can be constructed using only Standard Model fields and respecting gauge symmetries is a lepton-number-violating dimension-five operator:
\begin{equation}
    \mathcal{L}_5 =
    \frac{Z^\nu_{ij}}{\Lambda}
    \big( \overline{L_L^i} \, \tilde{\Phi} \big)
    \big( \tilde{\Phi}^T L_L^j \big)
    + \text{h.c.},
    \label{eq:eff_lagrangian}
\end{equation}
where $Z^\nu_{ij}$ is a dimensionless $3 \times 3$ coupling matrix, $\tilde{\Phi} = i\sigma_2 \Phi^*$ is the conjugate Higgs doublet, and $\Lambda$ denotes the scale of new physics beyond the Standard Model.

After electroweak symmetry breaking, this effective interaction generates a Majorana mass term for the light neutrinos:
\begin{equation}
    \mathcal{L}_{M_\nu}
    = \frac{1}{2} \, \frac{v^2}{\Lambda}
    Z^\nu_{ij} \,
    \overline{\nu_{L_i}} \, \nu^C_{L_j}
    + \text{h.c.},
\end{equation}
corresponding to the Majorana neutrino mass matrix:
\begin{equation}
    \mathcal{M}_\nu = Z^\nu_{ij} \, \frac{v^2}{\Lambda}.
\end{equation}

Compared to the renormalizable Dirac mass terms of the charged fermions, this effective operator contains two Higgs fields and is therefore of mass dimension five. As a result, it is non-renormalizable and must be interpreted as a low-energy manifestation of new physics at the scale $\Lambda$. Notably, this operator is the only dimension-five operator that can be constructed within the Standard Model field content and gauge symmetries to generate neutrino masses.

The suppression factor $v^2 / \Lambda$ naturally explains the smallness of neutrino masses if $\Lambda$ lies far above the electroweak scale. This structure mirrors the mass suppression obtained in seesaw mechanisms, which provide explicit ultraviolet completions of the effective operator in Eq.~\eqref{eq:eff_lagrangian}. These mechanisms are discussed in the following section.
 

\subsection{Seesaw Mechanism}

We now discuss an extension of the Standard Model that naturally generates light active neutrino masses through the introduction of one or more heavy sterile neutrinos. Adding $m$ sterile neutrino fields,
$\nu_{s i}$ $(i = 1,\dots,m)$, allows for two distinct types of neutrino mass terms. The most general neutrino mass Lagrangian can be written as:
\begin{equation}
    -\mathcal{L}_{M_\nu}
    =
    M_{D_{ij}} \, \overline{\nu}_{s i} \nu_{L j}
    + \frac{1}{2} M_{N_{ij}} \, \overline{\nu}_{s i} \nu^{c}_{s j}
    + \text{h.c.},
    \label{eq:seesaw_lagrangian}
\end{equation}
where $M_D$ is a complex $m \times 3$ Dirac mass matrix and $M_N$ is a complex symmetric $m \times m$ Majorana mass matrix.

The first term in Eq.~\eqref{eq:seesaw_lagrangian} corresponds to a Dirac mass term, generated after electroweak symmetry breaking through Yukawa couplings between the sterile neutrinos, the left-handed lepton doublets, and the Higgs field:
\begin{equation}
    Y^\nu_{ij} \, \overline{\nu}_{s i} \, \tilde{\Phi}^\dagger L_{L j}
    \;\;\Rightarrow\;\;
    M_{D_{ij}} = Y^\nu_{ij} \, \frac{v}{\sqrt{2}}.
\end{equation}
The second term is a Majorana mass term for the sterile neutrinos and violates lepton number by two units.

Equation~\eqref{eq:seesaw_lagrangian} may be rewritten in matrix form as:
\begin{equation}
    -\mathcal{L}_{M_\nu}
    =
    \frac{1}{2}
    \begin{pmatrix}
        \overline{\nu_L} & \overline{\nu_s}
    \end{pmatrix}
    \begin{pmatrix}
        0 & M_D^T \\
        M_D & M_N
    \end{pmatrix}
    \begin{pmatrix}
        \nu_L^c \\
        \nu_s^c
    \end{pmatrix}
    + \text{h.c.},
    \label{eq:seesaw_matrix}
\end{equation}
where $\vec{\nu} = (\vec{\nu}_L , \vec{\nu}_s^{\,c})^T$ is a $(3+m)$-dimensional vector. The full neutrino mass matrix $M_\nu$ is complex and symmetric, and may be diagonalized by a unitary transformation:
\begin{equation}
    (V^\nu)^T M_\nu V^\nu
    =
    \mathrm{diag}(m_1, m_2, \dots, m_{3+m}),
\end{equation}
with the corresponding mass eigenstates given by:
\begin{equation}
    \vec{\nu}_{\text{mass}} = (V^\nu)^\dagger \vec{\nu}.
\end{equation}

\noindent In terms of the mass eigenstates, the neutrino mass Lagrangian becomes:
\begin{align}
    -\mathcal{L}_{M_\nu}
    &=
    \frac{1}{2} \sum_{k=1}^{3+m} m_k
    \left(
        \overline{\nu}^c_{\text{mass},k} \nu_{\text{mass},k}
        +
        \overline{\nu}_{\text{mass},k} \nu^c_{\text{mass},k}
    \right) \\
    &=
    \frac{1}{2} \sum_{k=1}^{3+m} m_k \,
    \overline{\nu}_{M_k} \nu_{M_k},
\end{align}
where $\nu_{M_k} = \nu_{\text{mass},k} + \nu_{\text{mass},k}^c$ satisfies the Majorana condition
$\nu_{M_k} = \nu_{M_k}^c$.

In this mass basis, the original weak-interaction neutrino fields are related to the Majorana mass eigenstates by:
\begin{equation}
    \nu_{L i}
    =
    P_L \sum_{j=1}^{3+m}
    V^\nu_{i j} \, \nu_{M_j},
    \qquad i = 1,2,3.
\end{equation}

\noindent In the phenomenologically relevant limit where the eigenvalues of $M_N$ are much larger than the electroweak scale, $M_N \gg v$, the diagonalization of $M_\nu$ yields three light neutrino states $\nu_l$ and $m$ heavy neutrino states $N$:
\begin{equation}
    -\mathcal{L}_{M_\nu}
    =
    \frac{1}{2} \, \overline{\nu}_l M^l \nu_l
    +
    \frac{1}{2} \, \overline{N} M^h N,
\end{equation}
with approximate mass matrices:
\begin{align}
    M^l &\simeq - V_l^T \, M_D^T \, M_N^{-1} \, M_D \, V_l, \\
    M^h &\simeq V_h^T \, M_N \, V_h,
\end{align}
and mixing matrix:
\begin{equation}
    V^\nu \simeq
    \begin{pmatrix}
        \left( 1 - \frac{1}{2} M_D^\dagger M_N^{*-1} M_N^{-1} M_D \right) V_l
        &
        M_D^\dagger M_N^{*-1} V_h
        \\
        - M_N^{-1} M_D V_l
        &
        \left( 1 - \frac{1}{2} M_N^{-1} M_D M_D^\dagger M_N^{*-1} \right) V_h
    \end{pmatrix}.
\end{equation}

\noindent Here, $V_l$ and $V_h$ are $3\times3$ and $m\times m$ unitary matrices that describe mixing among the light and heavy neutrino sectors, respectively. The matrix $V_l$ may be identified with the Pontecorvo–Maki–Nakagawa–Sakata (PMNS) matrix, which is discussed in detail in a later section.

The structure of the mass eigenvalues illustrates the origin of the term \emph{seesaw}: the heavy neutrino masses scale with $M_N$, while the light neutrino masses are suppressed by $M_N^{-1}$. This mechanism is known as the Type-I seesaw, characterized by the introduction of heavy sterile neutrinos. It naturally produces light, predominantly left-handed neutrinos and heavy, predominantly right-handed neutrinos.

The Type-I seesaw mechanism therefore provides a compelling extension of the Standard Model that explains the smallness of neutrino masses without invoking extremely small Yukawa couplings. Furthermore, the heavy sterile neutrinos introduced in this framework may have important implications for physics beyond the Standard Model, including potential connections to dark matter.




\subsection{Lepton Number Violation and Leptogenesis}

A key consequence of Majorana neutrinos is the violation of lepton number. In the Standard Model, lepton number is an accidental global symmetry.  This means that it's not imposed by construction, but instead emerges because no renormalizable operators that violate lepton number are allowed by the gauge symmetries and field content. Many extensions of the Standard Model, however, naturally incorporate lepton number violation.

Lepton number violation also plays a central role in leptogenesis, a proposed explanation for one of the most fundamental questions in particle physics: why the observable Universe contains more matter than antimatter.  A comprehensive review of the phenomenology of matter–antimatter asymmetry is beyond the scope of this chapter. Instead, a brief summary of the key observational and theoretical ingredients relevant to leptogenesis is presented here.

The baryon asymmetry of the Universe is inferred from two independent classes of observations. The first arises from measurements of the primordial abundances of light elements, such as $D$, $^3\mathrm{He}$, $^4\mathrm{He}$, and $^7\mathrm{Li}$, produced during Big Bang nucleosynthesis (BBN). These abundances depend on the baryon-to-photon asymmetry parameter $\eta$, which is measured to be~\cite{Davidson_2008}:
\begin{equation}
    \eta^{\mathrm{BBN}}
    \equiv
    \left. \frac{n_B - n_{\bar{B}}}{n_\gamma} \right|_0
    =
    (4.7\text{--}6.5) \times 10^{-10}
\end{equation}

\noindent The second constraint comes from observations of anisotropies in the cosmic microwave background (CMB)~\cite{Hu_2002}. A key CMB observable is the speed of sound $c_s$ in the photon–baryon fluid. Measurements of temperature fluctuations in the CMB determine the baryon energy density $\rho_B$, commonly expressed in terms of the baryon density parameter:
\begin{equation}
    \Omega_B = \frac{\rho_B}{\rho_{\mathrm{crit}}}
\end{equation}
The corresponding baryon asymmetry inferred from CMB observations is:
\begin{equation}
    \eta^{\mathrm{CMB}}
    =
    2.74 \times 10^{-8} \, \Omega_B h^2
    =
    6.1^{+0.3}_{-0.2} \times 10^{-10}
\end{equation}
\noindent where $h = H_0 / (100\,\mathrm{km\,s^{-1}\,Mpc^{-1}}) = 0.682 \pm 0.0028$ is the dimensionless Hubble parameter~\cite{DESI}. The remarkable agreement between the BBN and CMB determinations of the baryon asymmetry represents a major success of hot Big Bang cosmology.

Any dynamical mechanism responsible for generating the observed baryon asymmetry must satisfy three necessary conditions, first identified by Sakharov and now known as the Sakharov conditions:
\begin{enumerate}[itemsep=-2pt]
    \item Baryon number violation
    \item C and CP violation
    \item Departure from thermal equilibrium
\end{enumerate}
Although the Standard Model contains all three ingredients in principle, it fails to produce a baryon asymmetry of sufficient magnitude.

Leptogenesis provides a compelling beyond-the-Standard-Model framework in which these conditions can be satisfied. In this scenario, the heavy sterile neutrinos introduced in the Type-I seesaw mechanism undergo CP-violating, out-of-equilibrium decays that generate a net lepton asymmetry. Electroweak sphaleron processes subsequently convert a fraction of this lepton asymmetry into a baryon asymmetry, linking the origin of neutrino mass to the matter–antimatter asymmetry of the Universe.


% \subsection{Neutrino Mass Hierarchy}
% We've discussed open questions relating to neutrino nature; whether they are dirac or majorana, what is the absolute neutrino mass scale, and their potential role in resolving matter-antimatter asymmetry in the Universe. Another open question has to do with the relative ordering of the light neutrino masses.

% In the simplified 2-flavor case, the probability of a neutrino "surviving" or being observed in the flavor state in which it was produced, is:
% \begin{equation}
%     P(\nu_e\rightarrow \nu_e)=1-\frac{1}{2}\sin^22\theta_{12}\sin^2\left(\frac{\delta m^2_{21}}{4E}L\right)
% \end{equation}
% Note that while the survival probability or one minus the oscillation probability is is dependent on the absolute value of the squared-mass difference, $\delta m_{ij}^2=m_i^2-m_j^2$, it is insensitvie to its' sign. Solar neutrino oscillation experiments have measured the following squared- mass difference:
% \begin{equation}
% 	\delta m_{21}^2=\delta m^2_{sol}\approx 7.39^{+0.21}_{-0.20} \times 10^{-5} eV^2
% \end{equation}
% while observation of atmospheric oscillation data results in:
% \begin{equation}
% 	|\delta m_{13}^2| = |\delta m_{23}^2| = \delta m_{atm}^2\approx 2.449^{+0.032}_{-0.030} \times 10^{-3} eV^2
% \end{equation}
% As mentioned previously, oscillation experiments demonstrated that neutrinos have mass, however these experiments only measure the mass-squared differences.

% While observation of neutrino oscilation enhancement in matter determined the sign of $\delta m_{21}^2$, the signs of $\delta m_{13}^2$ and $|\delta m_{23}^2|$ are unknown. This leaves two possible orderings of the three neutrino mass eigenstates.
% \begin{itemize}
% 	\item \textbf{Inverted Ordering} with negative $\delta m_{13}^2$ and $m_3 < m_1 < m_2$
% 	\item \textbf{Normal Ordering} with positive $\delta m_{13}^2$ and $m_1<m_2<m_3$
% \end{itemize}
% Where $m_1$ is the neutrino mass eigenstate with the highest $\nu_e$ flavor content. According to combined T2K, Super-K atmospheric analysis, the normal ordering is slightly favored \cite{wester_2004}. 

\section{Double Beta Decay}
\label{sec:double_beta_decay}

As discussed in the previous sections, the observation of neutrino oscillations establishes that neutrinos are massive and motivates extensions of the Standard Model that may violate lepton number. Double beta decay arises naturally in this context, providing an experimental probe that is simultaneously sensitive to neutrino properties and to the nuclear many-body dynamics governing rare weak processes. In particular, $0\nu\beta\beta$ offers a direct test of lepton number violation and the Majorana nature of neutrinos. However, the interpretation of $0\nu\beta\beta$ searches is fundamentally limited by uncertainties in the associated nuclear matrix elements (NMEs).

The $2\nu\beta\beta$ decay process, which is allowed within the Standard Model and has been experimentally observed in multiple nuclei, plays a critical complementary role. Measurements of this process provide an essential experimental benchmark for nuclear structure calculations. Of particular interest are two-neutrino double beta decay transitions to excited states of the daughter nucleus, which offer an underexplored but highly informative probe of nuclear dynamics. Measurements or limits on these transitions directly test the same nuclear operators that appear in $0\nu\beta\beta$ decay and are therefore especially relevant for reducing theoretical uncertainties in $0\nu\beta\beta$ decay searches. In the following sections, the theoretical framework of double beta decay is reviewed, with emphasis on two-neutrino double beta decay to excited states and their connection to nuclear matrix elements relevant for $0\nu\beta\beta$.

\section{Two-Neutrino Double Beta Decay}
\label{sec:2vbb}

Most unstable nuclei decay through first-order weak processes such as single $\beta^-$ decay or electron capture, converting a neutron into a proton or vice versa while conserving lepton number. In some even--even nuclei, however, single beta decay is energetically forbidden or strongly suppressed, while a second-order weak process involving the simultaneous conversion of two neutrons into two protons becomes allowed. This $2\nu\beta\beta$ process was first proposed by Goeppert-Mayer in 1935 and is permitted within the Standard Model~\cite{PhysRev.48.512}.

The canonical $2\nu\beta\beta$ process proceeds as
\begin{equation}
(A,Z) \rightarrow (A,Z+2) + 2e^- + 2\bar{\nu}_e ,
\end{equation}
conserving total lepton number. The decay changes the nuclear charge by two units while leaving the mass number unchanged and occurs only if the resulting daughter nucleus is more tightly bound than the parent. The available energy, or $Q$ value, is given by

\begin{equation}
Q_{\beta\beta} = m_N(^{A}_{Z}X) - m_{N-2}(^{A}_{Z+2}X') - 2m_e 
\end{equation}
which sets the scale for the phase space available to the emitted leptons. Here, $m_e$ is the electron mass in the rest frame, $m_N\left( ^{A}_{Z}X \right)$ is the mass of the mother nucleus in the rest frame, and $m_{N-2}\left( ^{A}_{Z+2}X' \right)$ represents the mass of the daughter nucleus in the rest frame.

Not all nuclei are capable of undergoing double beta decay. Whether a given isotope can decay via the $2\nu\beta\beta$ process is determined by nuclear binding energies and the relative stability of neighboring isobars. Insight into this behavior can be obtained from the semi-empirical description of nuclear masses.

The mass of an atomic nucleus with mass number $A$ and atomic number $Z$ may be written as:
\begin{equation}
    M = Z m_p + (A - Z) m_n - E(A,Z)
\end{equation}
where $m_p$ and $m_n$ are the proton and neutron rest masses, respectively, and $E(A,Z)$ is the nuclear binding energy. The binding energy is well approximated by the semi-empirical mass formula:
\begin{equation}
    E = a_v A - a_s A^{2/3}
    - a_C \frac{Z(Z - 1)}{A^{1/3}}
    - a_A \frac{(N - Z)^2}{A}
    + \delta(N,Z)
\end{equation}
which captures the dominant contributions to nuclear stability. The volume and surface terms describe the short-range attractive strong force, the Coulomb term accounts for proton–proton repulsion, and the asymmetry term reflects the energy cost of unequal numbers of protons and neutrons imposed by the Pauli exclusion principle.

The final contribution, $\delta(N,Z)$, represents the pairing energy and depends on whether the nucleus contains even or odd numbers of protons and neutrons:
\begin{equation}
    \delta(N,Z) =
    \begin{cases}
        -a_p A^{-1/2} & \text{even–even nuclei}, \\
        0 & \text{even–odd or odd–even nuclei}, \\
        +a_p A^{-1/2} & \text{odd–odd nuclei}
    \end{cases}
\end{equation}
This term favors nuclei with paired nucleons, making even–even nuclei systematically more tightly bound than their odd–odd neighbors.

As a result, nuclei with odd mass number $A$ typically possess a single stable isobar and undergo ordinary beta decay if energetically allowed. In contrast, nuclei with even mass number exhibit two possible pairing configurations, leading to the familiar parabolic dependence of nuclear mass on $Z$ for fixed $A$ as shown in Figure~\ref{fig:mass_parabola}. In some even–even nuclei, single $\beta^-$ decay is energetically forbidden or strongly suppressed due to angular momentum and parity constraints, while the nucleus two units away in $Z$ is more tightly bound. In such cases, the nucleus may decay directly to this lower-energy configuration through the simultaneous emission of two electrons and two antineutrinos.

\begin{figure}[b!]
	\centering
	\includegraphics[scale=0.5]{2_Chapter_Theory/Figures/mass_parabola.png}
	\caption{Atomic masses of $A = 136$ isotopes. Masses are given as differences with respect to the most bound isotope, $^{136}$Ba. The red (green) levels indicate odd-odd (even-even) nuclei. Figure taken from~\cite{Gomez-Cadenas:2011oep}.}
	\label{fig:mass_parabola}
\end{figure}

This structure explains why $2\nu\beta\beta$ decay is observed only in a limited set of even–even nuclei and highlights the close connection between double beta decay and nuclear pairing effects. These same nuclear structure considerations also influence transitions to excited states of the daughter nucleus and play an important role in shaping the nuclear matrix elements relevant for both $2\nu\beta\beta$ and $0\nu\beta\beta$ decay.

The rate of $2\nu\beta\beta$ decay can be calculated using Fermi’s golden rule, treating the process as a second-order weak interaction involving the emission of four leptons in the final state. The inverse half-life for $2\nu\beta\beta$ decay may be written in the standard factorized form:
\begin{equation}
    \Gamma^{2\nu}
    \equiv
    \left(T^{2\nu}_{1/2}\right)^{-1}
    =
    G^{2\nu}(Q_{\beta\beta}, Z)
    \left|
        \mathcal{M}^{2\nu}_{GT}
        +
        \frac{g_V^2}{g_A^2}
        \mathcal{M}^{2\nu}_{F}
    \right|^2 
\end{equation}
where $G^{2\nu}(Q_{\beta\beta}, Z)$ is the phase-space factor, $g_V$ and $g_A$ are the vector and axial-vector weak coupling constants, and $\mathcal{M}^{2\nu}_{F}$ and $\mathcal{M}^{2\nu}_{GT}$ are the Fermi and Gamow--Teller nuclear matrix elements, respectively.

The phase-space factor $G^{2\nu}$ accounts for the integration over the energies and angles of the two emitted electrons and two antineutrinos and depends strongly on the available decay energy $Q_{\beta\beta}$ and the nuclear charge $Z$. These phase-space factors can be calculated with relatively high precision using relativistic electron wave functions and Coulomb corrections and are among the best-controlled theoretical inputs to the $2\nu\beta\beta$ decay rate~\cite{Kotija2012,Horoi2018}.

In contrast, the nuclear matrix elements encode the details of nuclear structure and many-body correlations. The Fermi matrix element $\mathcal{M}^{2\nu}_{F}$ corresponds to transitions with no change in nuclear spin and arises from the vector component of the weak interaction, while the Gamow--Teller matrix element $\mathcal{M}^{2\nu}_{GT}$ corresponds to spin-changing transitions mediated by the axial-vector current. For two-neutrino double beta decay, Fermi transitions are strongly suppressed due to isospin conservation, as the dominant contribution proceeds through intermediate $1^+$ states, corresponding to $0^+ \rightarrow 1^+ \rightarrow 0^+$ nuclear transitions. As a result, the decay rate is overwhelmingly dominated by the Gamow--Teller contribution.

Because $2\nu\beta\beta$ decay is a second-order weak process, its half-life is extremely long, with experimentally measured values typically in the range of $10^{19}$ to $10^{22}$ years~\cite{PRITYCHENKO2025101694}. While calculations of $G^{2\nu}$ are robust, theoretical predictions of $\mathcal{M}^{2\nu}_{GT}$ depend sensitively on the nuclear model employed and often fail to reproduce experimental half-lives without additional modifications. To account for this discrepancy, it is common to introduce an effective axial-vector coupling constant, $g_A^{\mathrm{eff}}$, which is quenched relative to the free-nucleon value $g_A$. This is typically implemented through a rescaling of the Gamow--Teller matrix element,
\begin{equation}
    \mathcal{M}^{2\nu,\mathrm{eff}}_{GT}
    =
    \left( \frac{g_A^{\mathrm{eff}}}{g_A} \right)^2
    \mathcal{M}^{2\nu}_{GT}.
\end{equation}
The magnitude and physical origin of this quenching remain subjects of active investigation and are thought to arise from a combination of missing many-body correlations, non-nucleonic degrees of freedom, and limitations of model spaces used in nuclear calculations~\cite{10.3389/fphy.2019.00029}. Importantly, uncertainties associated with $g_A^{\mathrm{eff}}$ directly impact predictions for both $2\nu\beta\beta$ and $0\nu\beta\beta$ decay rates, further motivating experimental studies of $2\nu\beta\beta$ decay, including transitions to excited nuclear states.




% Because $2\nu\beta\beta$ is a second-order weak process, its decay rate is highly suppressed, with measured half-lives ranging from $10^{19}$ to $10^{22}$ years. Nevertheless, it has now been observed in a number of isotopes and represents the rarest nuclear decay process measured to date. Importantly, $2\nu\beta\beta$ provides the only experimentally accessible double beta decay mode and therefore serves as a benchmark for theoretical models of nuclear structure.

\section{Neutrinoless Double Beta Decay}
\label{sec:0vbb}

Building on Goeppert-Mayer’s work on $2\nu\beta\beta$ decay, Wendell Furry proposed the neutrinoless mode, now known as $0\nu\beta\beta$ decay \cite{Furry1939a}:
\begin{equation}
    (A,Z) \rightarrow (A,Z+2) + 2e^- 
\end{equation}
Unlike the two-neutrino mode, this process violates total lepton number by two units and is therefore forbidden within the Standard Model. Observation of $0\nu\beta\beta$ would constitute direct evidence of lepton number violation and establish that neutrinos are Majorana particles.

From an experimental standpoint, $0\nu\beta\beta$ and $2\nu\beta\beta$ share many similarities. Both are second-order weak processes and occur in the same set of even–even nuclei for which single beta decay is energetically forbidden. In both cases, nuclear recoil is negligible, and the decay energy is carried almost entirely by the emitted electrons. The key experimental distinction is that while $2\nu\beta\beta$ produces a continuous electron energy spectrum extending up to the endpoint energy $Q_{\beta\beta}$, $0\nu\beta\beta$ would manifest as a monoenergetic peak at the endpoint, smeared only by detector energy resolution.

Although phase-space considerations alone would favor the neutrinoless mode, the requirement of lepton number violation renders the decay rate extremely small, leading to half-lives exceeding $10^{26}$ years for experimentally accessible isotopes. As a result, $0\nu\beta\beta$ is among the rarest processes sought in modern experimental physics.

In general, several mechanisms beyond the Standard Model could mediate $0\nu\beta\beta$ decay, including heavy particle exchange or other lepton-number-violating interactions. However, a model-independent result known as the Schechter–Valle (or ``Black Box'') theorem demonstrates that the observation of $0\nu\beta\beta$, regardless of the underlying mechanism, necessarily implies that neutrinos possess a Majorana mass~\cite{blackbox}. The theorem does not specify which mechanism dominates the decay rate, nor does it require a direct connection to neutrino oscillation phenomenology. Specifically, it proposes the following:
\begin{itemize}
	\item Should \0nbb be observed, its Feynman diagram must feature two electrons, two up-quark fields, and two down-quark fields. The process connecting these fields is arbitrary and is referred to as the "black box process". The theorem argues that this "black box process" effectively establishes the dimension-9 operator.
	\item The up and down quarks are contracted by the $W$ boson.
	\item On the other end of the $W$ boson propagators, electron fields are converted into neutrino fields.
	\item The entire diagram can be rotated to turn into a process that converts anti-neutrinos to neutrinos as shown in Figure \ref{fig:blackbox}.
	\item Finally, the possible cancelation of this process by other diagrams is dismissed by naturalness arguments.
\end{itemize}

\begin{figure}[b!]
	\centering
	\includegraphics[scale=0.4]{blackbox.png}
	\caption{Depiction of the \0nbb black box theorem, the black box represents an arbitrary \0nbb process, which can be used to convert antineutrinos into neutrinos. Figure taken from Reference~\cite{merle_blackbox}.}
	\label{fig:blackbox}
\end{figure}

The key conclusion of the black-box theorem is that should \0nbb be observed, even if the observed mechanism is not light Majorana neutrino exchange, the neutrino is a Majorana particle. It should be noted that since the theorem's original proposal, counterexamples have been found allowing \0nbb without Majorana neutrinos~\cite{GRAF2024139111}, but the theorem still indicates potential links between \0nbb, Majorana mass, and lepton number violation more broadly.

The most widely studied and experimentally motivated scenario is the light Majorana neutrino exchange mechanism. In this case, the decay proceeds through the exchange of a virtual neutrino between two Standard Model weak vertices as illustrated in comparison to $2\nu\beta\beta$ decay in Figure~\ref{fig:feynman_double_beta}. The amplitude is nonzero only if neutrinos are massive and Majorana, as the process requires a helicity flip proportional to the neutrino mass. This mechanism provides a direct link between $0\nu\beta\beta$ decay and the absolute neutrino mass scale.

\begin{figure}[t!]
  \centering
    \begin{minipage}{0.45\textwidth}   
  \begin{tikzpicture}
    \begin{feynman}
      \vertex (n1) at (0, 1.5) {$n$};
      \vertex (n2) at (0,-1.5) {$n$};
      \vertex (v1) at (2, 1.5);
      \vertex (v2) at (2,-1.5);
      \vertex (w1) at (4, 0.75);
      \vertex (w2) at (4,-0.75);
      \vertex (p1) at (6, 2.0) {$p$};
      \vertex (p2) at (6,-2.0) {$p$};
      \vertex (e1) at (6, 0.75) {$e^-$};
      \vertex (e2) at (6,-0.75) {$e^-$};

      % antineutrino external legs (place them a bit to the left of the W vertices)
      \vertex (nu1) at (6, 1.35) {$\bar{\nu}_e$};
      \vertex (nu2) at (6,-1.35) {$\bar{\nu}_e$};

      \diagram* {
        (n1) -- [fermion] (v1) -- [fermion] (p1),
        (n2) -- [fermion] (v2) -- [fermion] (p2),
        (v1) -- [boson, edge label=$W^-$] (w1),
        (v2) -- [boson, edge label'=$W^-$] (w2),

        (w1) -- [fermion] (e1),
        (w2) -- [fermion] (e2),

        % outgoing antineutrinos
        (w1) -- [fermion] (nu1),
        (w2) -- [fermion] (nu2),
      };
    \end{feynman}
  \end{tikzpicture}
  \end{minipage}
  \hfill
  \begin{minipage}{0.45\textwidth}
  \begin{tikzpicture}
    \begin{feynman}
      \vertex (n1) at (0,1.5) {$n$};
      \vertex (n2) at (0,-1.5) {$n$};
      \vertex (v1) at (2,1.5);
      \vertex (v2) at (2,-1.5);
      \vertex (w1) at (4,0.75);
      \vertex (w2) at (4,-0.75);
      \vertex (p1) at (6,2) {$p$};
      \vertex (p2) at (6,-2) {$p$};
      \vertex (e1) at (6,0.75) {$e^-$};
      \vertex (e2) at (6,-0.75) {$e^-$};
      
      \diagram* {
        (n1) -- [fermion] (v1) -- [fermion] (p1),
        (n2) -- [fermion] (v2) -- [fermion] (p2),
        (v1) -- [boson, edge label=$W$] (w1) -- [fermion] (e1),
        (v2) -- [boson, edge label'=$W$] (w2) -- [fermion] (e2),
        (w1) -- [fermion, edge label=$\nu$] (w2),
      };
    \end{feynman}
  \end{tikzpicture}
  \end{minipage}
  \caption{Feynman diagrams for $2\nu\beta\beta$ decay (left) and $0\nu\beta\beta$ decay (right).}
  \label{fig:feynman_double_beta}
\end{figure}

Under the assumption that light Majorana neutrino exchange dominates, the inverse half-life for $0\nu\beta\beta$ decay can be written in factorized form as:
\begin{equation}
    \left(T^{0\nu}_{1/2}\right)^{-1}
    =
    G^{0\nu}(Q_{\beta\beta}, Z)
    \left| \mathcal{M}^{0\nu} \right|^2
    \left( \frac{m_{\beta\beta}}{m_e} \right)^2 
\end{equation}
where $G^{0\nu}$ is the phase-space factor, $\mathcal{M}^{0\nu}$ is the nuclear matrix element, $m_e$ is the electron mass, and $m_{\beta\beta}$ is the effective Majorana neutrino mass:

\begin{equation}
    m_{\beta\beta}
    =
    \left|
        m_1 U_{e1}^2
        +
        m_2 U_{e2}^2 e^{2 i \xi_1}
        +
        m_3 U_{e3}^2 e^{2 i \xi_2}
    \right| 
\end{equation}


\noindent Here, $m_i$ are the light neutrino mass eigenvalues, $U_{ei}$ are elements of the PMNS mixing matrix, and $\xi_i$ are the Majorana CP-violating phases. Unlike neutrino oscillation experiments, which are insensitive to the absolute mass scale and Majorana phases, $0\nu\beta\beta$ decay probes both. The unknown values of the Majorana phases lead to allowed bands for $m_{\beta\beta}$ as a function of the lightest neutrino mass, with distinct regions corresponding to normal and inverted mass orderings. As a result, experimental limits on the $0\nu\beta\beta$ half-life translate into ranges of allowed $m_{\beta\beta}$ values rather than a single constraint which is illustrated in Figure~\ref{fig:lobster}.

\begin{figure}[tt!]
    \centering
    \includegraphics[width = \textwidth]{2_Chapter_Theory/Figures/lobster_full.jpeg}
    \caption[Possible Majorana masses for normal (magenta) and inverted (blue) mass orderings, calculated with mixing angles and mass differences from the PMNS matrix.]{Possible Majorana masses for normal (magenta) and inverted (blue) mass orderings, calculated with mixing angles and mass differences from the PMNS matrix. The error bands come from uncertainties in the mixing parameters~\cite{Esteban_2024}. The KamLAND-Zen experimental limit on \(m_{\beta\beta}\) is shown in gray.  Recent limits~\cite{PhysRevLett.125.252502,PhysRevLett.126.181802,PhysRevLett.129.222501} for other key isotopes are shown in the panel on the right. Figure taken from Reference~\cite{DGooding2025}.}\label{fig:lobster}
\end{figure}

While the phase-space factor $G^{0\nu}$ can be calculated with high precision, the NME $\mathcal{M}^{0\nu}$ remains the largest source of theoretical uncertainty in interpreting $0\nu\beta\beta$ searches. The calculation of $\mathcal{M}^{0\nu}$ requires detailed knowledge of nuclear wave functions and many-body correlations and depends on the treatment of short-range physics, nuclear deformation, and the effective axial-vector coupling.  For the standard light Majorana neutrino exchange mechanism, the NME can be written schematically as a sum of long- and short-range contributions:
\begin{equation}
    \mathcal{M}^{0\nu}_{\mathrm{light}}
    =
    g_A^4
    \left(
        \mathcal{M}^{0\nu}_{\mathrm{long}} + \mathcal{M}^{0\nu}_{\mathrm{short}}
    \right)
\end{equation}
The axial-vector coupling constant $g_A$ governs the long-range weak interaction of nucleons and is factored out explicitly, serving as an important parameter in nuclear many-body calculations. The short-range contribution depends additionally on a two-nucleon coupling, $g^{\mathrm{NN}}$, which is not written explicitly here. As in the case of $2\nu\beta\beta$ decay, the phase-space factors relevant for $0\nu\beta\beta$ are known with high precision for all experimentally relevant isotopes~\cite{Kotija2012,Horoi2018}. In contrast, the NMEs themselves, along with some associated hadronic couplings, remain a dominant source of theoretical uncertainty despite significant recent progress.

The nuclear matrix elements (NMEs) encode the influence of nuclear structure on the rate of neutrinoless double beta decay and are obtained by combining nuclear wave functions for the initial and final states with the appropriate transition operators. In practice, NMEs are evaluated using nuclear many-body methods that attempt to capture correlations among nucleons across a wide range of length and energy scales. When limits on the $0\nu\beta\beta$ half-life are translated into constraints on the effective Majorana mass $m_{\beta\beta}$, the resulting uncertainty is presently dominated by the spread in available NME calculations rather than by experimental systematics.  A detailed theoretical treatment of the construction of the transition operators, the treatment of short-range correlations, and the associated theoretical uncertainties lies beyond the scope of this thesis. Comprehensive reviews of modern NME calculations can be found in Reference~\cite{Engel_2017}.

In general, NME calculations proceed by first specifying an effective nuclear Hamiltonian that includes nucleon--nucleon interactions relevant to the decay process. A second step then introduces a framework for describing collective nuclear structure and many-body correlations beyond the mean-field level. The most widely used approaches are summarized below.

\begin{itemize}

\item The \textbf{Nuclear Shell Model (NSM)} has long served as a foundational tool for describing nuclear structure. In this framework, nucleons are treated as moving independently within a mean-field potential, augmented by a strong spin--orbit interaction. This potential, often modeled using harmonic oscillator or Woods--Saxon forms, represents the averaged interaction of a nucleon with the rest of the nucleus. The resulting single-particle states organize into energy shells, with particularly stable configurations occurring at so-called magic numbers. For practical calculations, the nucleus is typically separated into an inert core of filled shells and a smaller set of active valence nucleons. While the shell model provides a highly detailed description of nuclear correlations within the chosen model space, computational limitations restrict its applicability to relatively small valence spaces.

\item The \textbf{Quasiparticle Random Phase Approximation (QRPA)} extends mean-field approaches by incorporating collective excitations and pairing correlations across a large set of nuclear orbitals. It is particularly well suited for medium and heavy nuclei, where shell-model calculations become computationally prohibitive. QRPA calculations rely on effective proton--neutron interactions, commonly parameterized by the coupling strength $g_{pp}$, which governs proton--neutron pairing. This parameter is often constrained by requiring agreement with experimentally measured $2\nu\beta\beta$ decay rates and subsequently applied to predictions of $0\nu\beta\beta$ decay. As a result, QRPA provides an explicit link between two-neutrino and neutrinoless decay calculations.

\item \textbf{Energy Density Functional (EDF)} methods describe nuclei using energy functionals that depend on local densities and currents, extending the concept of mean-field theory in a self-consistent manner. These methods allow for the inclusion of important nuclear effects such as deformation, pairing, configuration mixing, and collective motion. EDF calculations have proven effective in describing medium and heavy nuclei, which are of primary interest for double beta decay searches. However, EDF-based NMEs are often among the largest reported values, in part because certain proton--neutron correlations are not treated explicitly.

\item The \textbf{Interacting Boson Model (IBM)} provides a simplified, phenomenological description of nuclear structure by mapping pairs of valence nucleons onto bosonic degrees of freedom. In the context of double beta decay, the IBM has been extended to distinguish between proton and neutron bosons (IBM-2), enabling calculations of NMEs for even--even nuclei. While the model sacrifices microscopic detail, it offers a computationally efficient framework for exploring systematic trends across isotopic chains.

\item \textbf{\emph{Ab initio}} methods aim to describe nuclei starting from fundamental interactions derived from quantum chromodynamics via chiral effective field theory. These methods treat all nucleons explicitly and employ nuclear Hamiltonians with minimal phenomenological input. A key advantage of ab initio calculations is their systematic improvability and the ability to assess convergence. While these methods successfully reproduce properties of light and some medium-mass nuclei, extending them to the heavy nuclei relevant for $0\nu\beta\beta$ decay remains an active area of research~\cite{PhysRevLett.126.042502}.

\end{itemize}

\noindent To illustrate the level of agreement among different approaches, Figure~\ref{fig:NMEs} summarizes recent NME calculations for several candidate $0\nu\beta\beta$ isotopes. For a given nucleus, the predicted NMEs typically vary by factors of two to three across models. This spread constitutes one of the primary theoretical limitations in extracting neutrino mass information from $0\nu\beta\beta$ decay searches and provides strong motivation for experimental probes, such as two-neutrino double beta decay to excited states, that can help benchmark and constrain nuclear structure calculations.

\begin{figure}[t!]
	\centering
	\includegraphics[scale=0.6]{NME_calculations.png}
	\caption{Results from various NME calculation of $M_{0\nu}$ on particular \0nbb decaying isotopes versus atomic mass. Figure taken from Reference~\cite{Engel_2017}.}
	\label{fig:NMEs}
\end{figure}



Current nuclear structure methods yield values of $\mathcal{M}^{0\nu}$ that differ by factors of two to three for the same isotope. These discrepancies directly propagate into uncertainties on the extracted limits or measurements of $m_{\beta\beta}$. As a result, the physics reach of $0\nu\beta\beta$ experiments is no longer limited solely by exposure or background reduction, but increasingly by the reliability of nuclear matrix element calculations. Reducing these uncertainties has therefore become a central goal in the field.

\section{Double Beta Decay to Excited States}

An additional class of processes can provide valuable experimental input into NME calculations and aid in the interpretation of $0\nu\beta\beta$ decay searches: two-neutrino double beta decay to excited states of the daughter nucleus, denoted $2\nu\beta\beta^*$. In these Standard Model–allowed transitions, the parent nucleus undergoes double beta decay but populates an excited state of the daughter rather than its ground state. The subsequent de-excitation of the daughter nucleus produces a characteristic gamma-ray cascade.

Decays to excited states are suppressed by several orders of magnitude relative to ground-state transitions due to the reduced available phase space arising from the smaller effective $Q$ value~\cite{exstate_stats}. As a consequence, theoretical predictions for the corresponding half-lives span multiple orders of magnitude, reflecting the same nuclear-structure uncertainties that dominate predictions for $0\nu\beta\beta$ decay. Figure~\ref{fig:ex_halflife} illustrates representative predictions for $T^{2\nu^*}_{1/2}$ obtained using different nuclear models. An experimental observation of $2\nu\beta\beta^*$ in xenon would therefore provide a powerful constraint on nuclear matrix element calculations and help reduce theoretical uncertainties relevant to $0\nu\beta\beta$ searches.

\begin{figure}[t!]
	\centering
	\includegraphics[scale=1.3]{halflife_pred_ex.jpg}
	\caption{Predictions of $T_{1/2}^{2\nu^*}$ using various NME calculation methods. Figure taken from \cite{exhalflife_calc}.}
	\label{fig:ex_halflife}
\end{figure}

The discussion below follows closely the formalism presented in Ref.~\cite{excited_nme}, which explores the relationship between $2\nu\beta\beta$ and $0\nu\beta\beta$ NMEs within the NSM and the proton–neutron quasiparticle random-phase approximation (pnQRPA). The experimental signature of $2\nu\beta\beta^*$ consists of a standard $2\nu\beta\beta$ decay, followed (after a delay of order picoseconds) by the emission of one or more gamma rays as the daughter nucleus relaxes to its ground state. Experimentally, identifying this process requires either particle identification techniques capable of tagging the de-excitation gamma rays or the observation of distortions in the electron energy spectrum relative to the dominant ground-state $2\nu\beta\beta$ background. While challenging, these signatures provide additional handles for background discrimination compared to ground-state decays.

As mentioned in Section~\ref{sec:2vbb} the half-life for $2\nu\beta\beta$ decay can be reasonably approximated with a single nuclear matrix element~\cite{excited_nme}:
\begin{equation}
    \mathcal{M}^{2\nu}
    =
    -\sum_k
    \frac{
        \bigl( 0_f^+ \bigl\| \sum_a \tau_a^- \sigma_a \bigr\| 1_k^+ \bigr)
        \bigl( 1_k^+ \bigl\| \sum_b \tau_b^- \sigma_b \bigr\| 0_i^+ \bigr)
    }{
        \left[ E_k - (E_i + E_f)/2 \right] / m_e
    }
\end{equation}
where the indices $a$ and $b$ run over all nucleons, $\tau^-$ is the isospin-lowering operator converting neutrons into protons, and $\sigma$ is the spin operator. The sum extends over all intermediate $1^+$ states of the odd–odd nucleus, with the energy denominator involving the excitation energy of each intermediate state relative to the average of the initial and final nuclear energies.

In contrast, the $0\nu\beta\beta$ nuclear matrix element (assuming the standard light Majorana neutrino exchange mechanism) is conventionally decomposed into three spin–isospin components:
\begin{equation}
    \mathcal{M}^{0\nu}_L
    =
    \mathcal{M}^{0\nu}_{GT}
    -
    \mathcal{M}^{0\nu}_{F}
    +
    \mathcal{M}^{0\nu}_{T},
\end{equation}
corresponding to Gamow--Teller ($GT$), Fermi ($F$), and tensor ($T$) contributions. These components are defined in terms of two-body operators:
\begin{equation}
    \mathcal{M}^{0\nu}_K
    =
    \sum_{k,ab}
    \bigl(
        0_f^+
        \bigl\|
        \mathcal{O}^K_{ab}
        \tau^-_a \tau^-_b
        H_K(r_{ab})
        f^2_{\mathrm{SRC}}(r_{ab})
        \bigr\|
        0_i^+
    \bigr)
\end{equation}
where $\mathcal{O}^F_{ab}=\mathds{1}$, $\mathcal{O}^{GT}_{ab}=\sigma_a\cdot\sigma_b$, and
\[
\mathcal{O}^T_{ab}
=
3(\sigma_a\cdot\hat{r}_{ab})(\sigma_b\cdot\hat{r}_{ab})
-
\sigma_a\cdot\sigma_b 
\]
The quantity $r_{ab}$ denotes the distance between nucleons $a$ and $b$, and $f_{\mathrm{SRC}}(r)$ accounts for short-range correlations.

The neutrino potential $H_K(r_{ab})$ encodes the momentum dependence of the virtual neutrino exchange and is given by:
\begin{equation}
    H_K(r_{ab})
    =
    \frac{2R}{\pi g_A^2}
    \int_0^\infty
    \frac{
        h_K \, j_\lambda(p r_{ab}) \, p^2 \, dp
    }{
        \epsilon_K
    },
\end{equation}
where $\epsilon_K = p \left[ p + E_k - (E_i + E_f)/2 \right]$, $g_A = 1.27$, and $R = 1.2 A^{1/3}$~fm. The structure of this potential introduces a characteristic radial dependence into the NME, allowing the matrix elements to be expressed in terms of radial distributions:
\begin{equation}
    M^{0\nu}_L(1b)
    =
    \int_0^\infty C^{0\nu}(r)\,dr,
    \qquad
    M^{2\nu}
    =
    \int_0^\infty C^{2\nu}(r)\,dr.
\end{equation}
For $0\nu\beta\beta$ decay, the radial distribution may be decomposed as:
\begin{equation}
    C^{0\nu}(r)
    =
    C^{0\nu}_{GT}(r)
    -
    C^{0\nu}_{F}(r)
    +
    C^{0\nu}_{T}(r),
\end{equation}
with:
\begin{equation}
    C^{0\nu}_K(r)
    =
    \sum_{k,ab}
    \bigl(
        0_f^+
        \bigl\|
        \mathcal{O}^K_{ab}
        \tau^-_a \tau^-_b
        H_K(r_{ab})
        f^2_{\mathrm{SRC}}(r_{ab})
        \delta(r - r_{ab})
        \bigr\|
        0_i^+
    \bigr).
\end{equation}

\noindent Reference~\cite{Jokiniemi:2022ayc} demonstrated that the radial distributions $C^{2\nu}(r)$ and $C^{0\nu}(r)$ exhibit striking qualitative similarities across a wide range of nuclei and model assumptions. Figure~\ref{fig:nme_radial} shows representative radial distributions for $^{76}$Ge calculated within the pnQRPA framework. The similarity in the spatial structure of the two matrix elements suggests that both processes probe related nuclear correlations, despite their different momentum transfers.

\begin{figure}[b!]
	\centering
	\includegraphics[scale=0.35]{nme_radial.png}
	\caption{Radial distributions of \0nbb (top) and \2nbb (bottom) NMEs of $^{76}$Ge obtained via pnQRPA. Figure taken from Reference~\cite{excited_nme}.}
	\label{fig:nme_radial}
\end{figure}

While neither $2\nu\beta\beta$ nor $2\nu\beta\beta^*$ decay rates are accurately predicted \emph{a priori} by nuclear models, strong correlations between $2\nu\beta\beta$ and $0\nu\beta\beta$ NMEs have been observed. In practice, calculated $2\nu\beta\beta$ NMEs are often renormalized using an effective axial coupling to reproduce measured half-lives. Despite this adjustment, the relative trends among isotopes remain robust.

Figure~\ref{fig:corr_2nu_0nu} illustrates the correlation between $2\nu\beta\beta$ and $0\nu\beta\beta$ NMEs obtained using both NSM and pnQRPA calculations. The presence of a clear correlation across different nuclear models supports the idea that improved experimental constraints on $2\nu\beta\beta$, and especially on the more selective $2\nu\beta\beta^*$ transitions, can provide meaningful benchmarks for nuclear structure calculations relevant to $0\nu\beta\beta$ decay.

\begin{figure}[t!]
	\centering
	\includegraphics[scale=0.33]{corr_2nu_0nu.png}
	\caption{Correlation of \2nbb and \0nbb NME as calculated by NSM (Nuclear Shell Model) and pnQRPA (proton-neutron quasiparticle random-phase approximation) methods. Figure taken from Reference~\cite{exstate_corr}.}
	\label{fig:corr_2nu_0nu}
\end{figure}

These theoretical insights provide strong motivation for experimental searches for $2\nu\beta\beta$ to excited states. Such measurements offer a unique opportunity to test nuclear many-body methods, constrain NME calculations, and ultimately improve the reliability of neutrino mass limits extracted from $0\nu\beta\beta$ decay experiments.



% \section{Double Beta Decay to Excited States}
% A related physics process can provide further experimental input into the NME calculations, and aid in the interpretation of \0nbb experiment results, \2nbb to excited states, or \2nbb$^*$. \2nbb$^*$ is a SM process in which the parent isotope undergoes double-beta decay but transitions to an excited state of the daughter nucleus instead of the ground state. These decays are suppressed by orders of magnitude compared to decays to ground states because of the reduced phase space from smaller $Q$ values\cite{exstate_stats}. Mirroring the status of NME calculations, predictions for the decay rates into excited states vary by multiple orders of magnitude, see Figure \ref{fig:ex_halflife}. An observation of this decay in Xenon could inform NME calculations and reduce the theoretical uncertainties. The following discussion of the relationship between \2nbb and \0nbb NMEs follows \cite{excited_nme} closely. This discussin describes how the two calculations are related in the nuclear shell model (NSM) and proton-neutron quasiparticle random-phase approximation (pnQRPA) frameworks.

% \begin{figure}[b!]
% 	\centering
% 	\includegraphics[scale=1.3]{halflife_pred_ex.jpg}
% 	\caption{Predictions of $T_{1/2}^{2\nu^*}$ using various NME calculation methods. Figure taken from \cite{exhalflife_calc}.}
% 	\label{fig:ex_halflife}
% \end{figure}

% The physical signature of \2nbb$^*$ is of a \2nbb decay immediately, few pico-second delay, followed by a gamma cascade as the daughter nucleus relaxes into its ground state. The task of detecting this rare process becomes one of either particle identification or resolving a distortion in the much larger decay to ground state background.

% The \2nbb decay half-life, to a very good approximation, depends on a single NME \cite{excited_nme}.
% \begin{equation}
%     M^{2\nu}=-\sum_k\frac{(0^+_f||\sum_a\tau^-_a\sigma_a||1^k_+)(1^k_+||\sum_b\tau^-_b\sigma_b||0^+_i)}{[E_k-(E_i+E_f)/2]/m_e}
% \end{equation}
% where $a$ and $b$ indices run over all nucleons, the isospin operator $\tau^-$ turns neutrons into protons, $\sigma$ is the spin operator, and the denominator involves the energies of the initial, final, and each $kth$ intermediate $1^+$ state. For \0nbb, considering the best motivated light neutrino exchange mechanism, the NME is usually written in terms of three spin structures:
% \begin{equation}
%     M^{0\nu}_L=M^{0\nu}_{GT}-M^{0\nu}_F+M^{0\nu}_T
% \end{equation}
% The three spin structures are referred to Gamow-Teller ($M^{0\nu}_{GT}$), Fermi ($M^{0\nu}_{F}$), and tensor ($M^{0\nu}_{T}$). They correspond to the operators $\mathcal{O}^F_{ab}=\mathds{I}$, $\mathcal{O}^{GT}_{ab}=\sigma_a\cdot\sigma_b$, $\mathcal{O}^T_{ab}=3(\sigma_a\cdot\hat{r}_{ab})(\sigma_b\cdot\hat{r}_{ab})-\sigma_a\cdot\sigma_b$ used in their definitions:
% \begin{equation}
%     M^{0\nu}_K=\sum_{k, ab} (0^+_f||\mathcal{O}^K_{ab}\tau^-_a\tau^-_bH_K(r_{ab})f^2_{SRC}(r_{ab})||0^+_i)
% \end{equation}
% A factor to note here is $r_{ab}$ the distance between nucleon $a$ and nucleon $b$. Used in the neutrino potential:
% \begin{equation}
%     H_K(r_{ab})=\frac{2R}{\pi g^2_A}\int_0^\infty\frac{h_Kj_\lambda(pr_{ab})p^2dp}{\epsilon_K}
% \end{equation}
% with $\epsilon_K=p(p+E_k-(E_i+E_f)/2)$, $g_A=1.27$, and $R=1.2A^{1/3}$fm
% with nucleon number, $A$. It is in these radial distributions, that give the neutrino potentials structure, that the relationship between \0nbb and \2nbb can be explored.

% We note that the NME radial distributions are qualitatively similar and that they satisfy:
% \begin{equation}
%     M^{0\nu}_L(1b)=\int^\infty_0 C^{0\nu}(r)dr
% \end{equation}
% satisfy:
% \begin{equation}
%     M^{2\nu}=\int^\infty_0 C^{2\nu}(r)dr
% \end{equation}
% Once again we can split the integral into three terms.
% \begin{equation}
%     C^{0\nu}(r)=C^{0\nu}_{GT}(r)-C^{0\nu}_F(r)+C^{0\nu}_T(r)
% \end{equation}
% \begin{equation}
%     C^{0\nu}_K=\sum_{k,ab}(0^+_f||\mathcal{O}^K_{ab}\tau^-_a\tau^-_bH_K(r_{ab})f^2_{SRC}(r_{ab})\delta(r-r_{ab})||0^+_i)
% \end{equation}
% Jokineimi et.al. found a strong correlation in the radial distributions of \2nbb and \0nbb nuclear matrix elements.

% \begin{figure}[b!]
% 	\centering
% 	\includegraphics[scale=0.4]{nme_radial.png}
% 	\caption{Radial distributions of \0nbb (top) and \2nbb (bottom) NMEs of $^{76}$Ge obtained via pnQRPA. Figure taken from \cite{excited_nme}.}
% 	\label{fig:nme_radial}
% \end{figure}

% While neither \2nbb or \2nbb$^*$ has had their decay rates successfully predicted. In fact, the simplest model NMEs are usually adjusted with a "quenching" factor to match the observations of \2nbb half-lives. Without these quenching factors, the models consistently overpredict the half-lives. Close correlation between the NMEs of \2nbb and \0nbb have been demonstrated. Figure \ref{fig:corr_2nu_0nu} shows the correlations between \2nbb and \0nbb NME as calculated by two different methods.

% \begin{figure}[h]
% 	\centering
% 	\includegraphics[scale=0.3]{corr_2nu_0nu.png}
% 	\caption{Correlation of \2nbb and \0nbb NME as calculated by NSM (Nuclear Shell Model) and pnQRPA (proton-neutron quasiparticle random-phase approximation) methods. Figure taken from \cite{exstate_corr}.}
% 	\label{fig:corr_2nu_0nu}
% \end{figure}

% These theoretical findings motivates a better understanding of \2nbb and \2nbb$^*$ to aid in theoretical calculations of $M^{0\nu}$, and determination of Majorana neutrino mass.


% \section{Double Beta Decay Experiments}
% \0nbb if it exists has an incredibly long half-life greater than $10^{26}$ years. Should it be discovered, \0nbb would be the slowest natural process observed experimentally. To successfully carry out this rare event search, \0nbb experiments must be designed with crucial criteria in mind:
% \begin{itemize}
% 	\item \textbf{High Energy Resolution:} Distinguishing the \0nbb peak at the \2nbb endpoint requires precise energy resolution to reduce ROI (region of interest) contamination from the \2nbb decay spectrum. Modern experiments on the forefront of this design criterion are achieving $\frac{\sigma_E}{E}\approxeq 0.1\%$ resolution, and completely suppressing the \2nbb background. 
% 	\item \textbf{High Isotope Loading:} Maximizing the amount of double-beta decaying isotoped in the sensitive regions of the experiment is crucial for achieving sensitivity to long \0nbb half-lives in a reasonable timeframe, usually a decade of data-taking. Modern experiments are instrumenting about a metric ton of double-beta decaying isotope. The limiting factor to this criterion is often budgetary or enrichment capability.
% 	\item \textbf{Low Unrelated Backgrounds:} Experiments need to be prepared in environments with low incidental radioactivity. This necessarily means deep underground facilities, with the experiments themselves being constructed with the highest standard of radiopure materials.
% \end{itemize}
% The current world leading limit on Majorana neutrino mass has been set by KamLAND-ZEN at $m_{\beta\beta} < (28-122) $ meV, at 90\% C.L. The limits on majorana neutrino mass based on multiple NME calculations can be seen in Figure \ref{fig:klz_0nu_limit}. It is readily apparent that the exact values of the NMEs drastically impact the significance of the latest experimental results. For the highest NMEs, the limits placed by KamLAND-ZEN enter the non-degnerate regions of inverted ordering phase space. While for the lower NMEs, the limits remain in the degenerate region. KamLAND-ZEN is the only \0nbb experiment to place limits in the inverted ordering region for any NME.

\section{Double Beta Decay Experiments}

If neutrinoless double beta decay ($0\nu\beta\beta$) exists, it is an extraordinarily rare process, with an expected half-life exceeding $10^{26}$~years for the most favorable isotopes. A confirmed observation would represent the slowest natural radioactive decay ever measured and would provide direct evidence for lepton number violation and the Majorana nature of neutrinos. The extreme rarity of this process imposes stringent requirements on experimental design. Successful $0\nu\beta\beta$ searches must simultaneously maximize signal efficiency while suppressing backgrounds to unprecedented levels. Several key criteria therefore define the performance of modern double beta decay experiments:

\begin{itemize}
	\item \textbf{Excellent Energy Resolution:}  
	The experimental signature of $0\nu\beta\beta$ is a monoenergetic peak at the decay $Q$ value, coincident with the endpoint of the continuous $2\nu\beta\beta$ spectrum. Precise energy resolution is essential to minimize contamination from the $2\nu\beta\beta$ tail within the region of interest (ROI). State-of-the-art experiments now achieve fractional energy resolutions of order $\sigma_E/E \simeq 0.1\%$, effectively eliminating $2\nu\beta\beta$ as a limiting background.

	\item \textbf{Large Isotope Mass:}  
	Sensitivity to extremely long half-lives requires large exposures, motivating the deployment of tonne-scale quantities of double beta decaying isotopes. Achieving such isotope masses typically demands both isotopic enrichment and scalable detector technologies. In practice, enrichment cost and isotope availability often set the ultimate experimental scale.

	\item \textbf{Ultra-Low Background Environment:}  
	Backgrounds from natural radioactivity and cosmic-ray interactions must be suppressed to levels below one count per tonne-year in the ROI. This necessitates operation in deep underground laboratories, stringent material screening, and detector designs that enable powerful background discrimination.
\end{itemize}

\noindent Among current-generation experiments, KamLAND-Zen has demonstrated world-leading sensitivity to $0\nu\beta\beta$ decay in $^{136}$Xe. The most recent KamLAND-Zen result places a 90\% confidence level limit on the effective Majorana neutrino mass of $m_{\beta\beta} < (28\text{--}122)\ \text{meV},$ where the range reflects uncertainties associated with NME calculations. Figure~\ref{fig:klz_0nu_limit} shows the corresponding constraints in the $m_{\beta\beta}$–$m_{\text{lightest}}$ plane for multiple NME models. The figure highlights the critical role of nuclear theory: for larger NMEs, the KamLAND-Zen limit begins to probe the non-degenerate region of the inverted neutrino mass ordering, whereas for smaller NMEs the constraint remains within the quasi-degenerate regime. KamLAND-Zen is currently the only $0\nu\beta\beta$ experiment whose sensitivity reaches the inverted-ordering parameter space for any NME calculation.


\begin{figure}[t!!]
	\centering
	\includegraphics[scale=0.35]{klz_0nu_limit.png}
	\caption{Effective Majorana neutrino mass as a function of the lightest neutrino mass state $m_{lightest}$. The shaded regions are based on best-fit values of neutrino oscilation parameters for (a) the normal ordering (NO) and (b) the inverted ordering (IO), the lighter shaded regions indicate the $3\sigma$ ranges based on oscillation parameter uncertainties. The horizontal lines indicate 90\% C.L. limits on $m_{\beta\beta}$ considering multiple NME calculations. Figure taken from Reference~\cite{full_klz_0nu}.}
	\label{fig:klz_0nu_limit}
\end{figure}


Two-neutrino double beta decay to excited states ($2\nu\beta\beta^*$) has been experimentally observed in only a small subset of known double beta decay isotopes. To date, positive detections have been reported for just two transitions:
\begin{itemize}
    \item $^{100}$Mo $\rightarrow\,^{100}$Ru$(0^+_1)$, with
    $T_{1/2} = 5.9^{+0.9}_{-0.6}\times10^{20}$~years,
    \item $^{150}$Nd $\rightarrow\,^{150}$Sm$(0^+_1)$, with
    $T_{1/2} = 1.33^{+0.45}_{-0.26}\times10^{20}$~years.
\end{itemize}

\noindent The scarcity of observed $2\nu\beta\beta^*$ transitions reflects the substantial experimental challenges associated with these decays. Relative to ground-state $2\nu\beta\beta$, decays to excited states suffer from significantly reduced phase space and therefore much longer half-lives. However, their distinctive experimental signature, characterized by the coincident emission of de-excitation gamma rays, offers additional handles for background suppression and provides a powerful probe of nuclear structure.

To date, $2\nu\beta\beta^*$ decay has not been observed in $^{136}$Xe, the isotope used by KamLAND-Zen and the primary focus of this thesis. The current most stringent limit on this process:
\[
T^{2\nu}_{1/2}(0^+ \rightarrow 0^+_1) > 1.4\times10^{24}\ \text{years} \quad (90\%~\text{C.L.})
\]
was established by the EXO-200 experiment~\cite{exo200}. Figure~\ref{fig:exo200_fit} shows the EXO-200 spectral fits used to extract this limit. No statistically significant excess consistent with an excited-state signal was observed.

The analysis presented in this dissertation reports the latest search for $2\nu\beta\beta^*$ using KamLAND-Zen~800 data. Leveraging KamLAND-Zen’s large $^{136}$Xe mass, low background environment, and excellent energy resolution, this work achieves sensitivity beyond the existing EXO-200 limit. An improved constraint on $2\nu\beta\beta^*$ decay in $^{136}$Xe would provide an important experimental benchmark for NME calculations and directly inform the interpretation of current and future $0\nu\beta\beta$ searches.


% \subsection{Observations and Current Limits of $2\nu\beta\beta^*$}
% Double beta decay to excited states has been observed in only two isotopes, a minority of the isotopes known to undergo double-beta decay.
% \begin{itemize}
%     \item $^{100}Mo-^{100}Ru(0^+_1)$ : $T_{1/2}=5.9^{+0.9}_{-0.6}\times 10^{20}$ years
%     \item $^{150}Nd-^{150}Sm(0^+_1)$ : $T_{1/2}=1.33^{+0.45}_{-0.26}\times 10^{20}$ years
% \end{itemize}
% \2nbb$^*$ has yet to be observed in $^{136}Xe$ the double beta decaying isotope of the KamLAND-ZEN experiment and the focus of this thesis. The current world-leading limit on this process is $T^{2\nu}_{1/2}(0^+\rightarrow 0^+_1)>1.4\times 10^{24}$ years at 90\% C.L. set by the EXO-200 experiment. This work describes the latest analysis of KamLAND-ZEN 800 data which surpasses this limit.

\begin{figure}[b!]
	\centering
	\includegraphics[scale=0.42]{exo200_fit.png}
	\caption{EXO-200's fit over energy spectrum (left) and particle ID discriminator spectrum (right) in two data-taking phases of the excited state signal and background. The decay to excited states was not found, and a lower limit was placed. Figure taken from \cite{exo200}.}
	\label{fig:exo200_fit}
\end{figure}