% ABSTRACT

Two-neutrino double beta decay to excited nuclear states
($2\nu\beta\beta^\ast$) provides important experimental constraints on nuclear matrix element calculations relevant to neutrinoless double beta decay searches. This dissertation presents a search for
$2\nu\beta\beta^\ast$ decay of $^{136}$Xe using data from the
KamLAND-Zen~800 experiment, which employs 745~kg of xenon enriched to
91\% in $^{136}$Xe dissolved in a liquid scintillator detector. The
analysis uses 4.3~years of data collected between February~2019 and
April~2023, corresponding to an exposure of more than one ton-year of
$^{136}$Xe. A background model based on Monte Carlo simulations, calibrated with in situ sources, is used to perform a binned maximum likelihood fit in the energy range of 2.5--4.8~MeV. Independent studies are used to constrain key low-energy background components, including $^{11}$C from cosmic muon spallation and intrinsic $^{40}$K contamination of the inner balloon film, and these constraints are incorporated into the final fit. No statistically significant excess is observed. A 90\% confidence level lower limit on the $2\nu\beta\beta^\ast$ half-life is derived using the Feldman--Cousins method, yielding $T_{1/2} > 4.25 \times 10^{24}$~yr. This result represents a preliminary but potentially world-leading constraint and demonstrates the sensitivity of KamLAND-Zen~800 to excited-state double beta decay.
