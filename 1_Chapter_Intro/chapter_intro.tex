\chapter{Introduction}
\label{chapter:introduction}
\thispagestyle{myheadings}

One of the most striking features of the universe is that it exists in a form capable of forming stars, planets, and ultimately life. This fact alone points to a deep asymmetry in nature: matter is abundant, while antimatter is almost entirely absent. In the early universe, following the Big Bang, energy was readily converted into particle–antiparticle pairs under extreme temperatures and densities. According to the known laws of physics, these processes should have produced matter and antimatter in equal quantities, leading to their mutual annihilation as the universe cooled. The survival of matter therefore signals that a subtle but fundamental imbalance must have emerged during the universe’s earliest moments, the origin of which remains one of the central open questions in modern physics.

The existence of this imbalance implies that the fundamental symmetries governing particle interactions are not exact. In particular, charge–parity (CP) symmetry determines whether the laws of physics treat matter and antimatter in the same way. Although CP violation has been observed in the quark sector, its measured effects within the Standard Model are far too weak to account for the matter dominance inferred from cosmological observations. This gap between theory and observation suggests that additional sources of CP violation, or entirely new particles and interactions, played a role in shaping the universe we observe today.

Neutrinos offer a compelling window into this missing physics. Unlike other fermions in the Standard Model, neutrinos are exceptionally light, weakly interacting, and exhibit properties that already require physics beyond the Standard Model. Many theoretical frameworks link these unusual features to the origin of the cosmic matter asymmetry through the mechanism of leptogenesis. In such scenarios, CP-violating processes involving heavy neutrino states in the early universe generate an excess of leptons over antileptons, which is later converted into a baryon asymmetry by electroweak interactions. A key ingredient in many of these models is that neutrinos are Majorana particles, identical to their own antiparticles. This possibility can be tested experimentally through the search for neutrinoless double beta decay ($0\nu\beta\beta$), a rare nuclear process whose observation would reveal lepton number violation and provide direct evidence for the Majorana nature of neutrinos and for new physics beyond the Standard Model.

Neutrinoless double beta decay ($0\nu\beta\beta$) is a hypothetical nuclear transition in which two neutrons decay into two protons and two electrons without the emission of neutrinos. If observed, this process would demonstrate the violation of lepton number and provide a direct link between nuclear decay rates and fundamental neutrino properties. While experimental searches for $0\nu\beta\beta$ continue to improve in sensitivity, the interpretation of any observed signal, or increasingly stringent null result, depends critically on the reliability of nuclear matrix element calculations.

At present, theoretical predictions for the nuclear matrix elements governing $0\nu\beta\beta$ differ substantially among nuclear-structure approaches, leading to significant uncertainties in the inferred neutrino mass scale. Reducing these uncertainties is therefore essential for fully realizing the physics potential of $0\nu\beta\beta$ experiments. One promising avenue for constraining nuclear matrix element calculations is provided by measurements of Standard Model two-neutrino double beta decay ($2\nu\beta\beta$), which serve as important benchmarks for nuclear theory. In addition to the well-studied decays to the ground state of the daughter nucleus, double beta decay can also proceed to excited nuclear states. Although such excited-state transitions are strongly suppressed by reduced phase space, they probe complementary aspects of nuclear structure and provide additional experimental constraints on the models used to calculate $0\nu\beta\beta$ nuclear matrix elements.

In particular, $2\nu\beta\beta$ to excited states of the daughter nucleus ($2\nu\beta\beta^\ast$) offers a unique opportunity to test nuclear-structure calculations beyond the single ground-state transition. These decays involve different combinations of nuclear wave-function components and intermediate-state contributions, and are accompanied by characteristic gamma-ray cascades as the daughter nucleus de-excites. As a result, excited-state decays provide sensitivity to modeling assumptions that may not be fully constrained by ground-state $2\nu\beta\beta$ data alone. Experimental information on these suppressed channels can therefore help discriminate among competing nuclear models and reduce the spread of predicted $0\nu\beta\beta$ nuclear matrix element calculations.

This thesis focuses on a search for double beta decay of $^{136}$Xe to excited states of $^{136}$Ba using data from the KamLAND-Zen 800 experiment. Owing to the extreme rarity of these processes and the presence of substantial radioactive and instrumental backgrounds, such a search is inherently challenging. The analysis is sensitive primarily to the most dominant excited-state decay modes and is largely agnostic to the specific excited state involved. Rather than targeting a particular transition, the search is designed to address a more fundamental question: whether any statistically significant indication of excited-state double beta decay can be observed in the available dataset. Establishing an observation, or setting improved limits in the absence of a signal, provides valuable new experimental input for nuclear matrix element calculations and strengthens the interpretation of $^{136}$Xe-based searches for $0\nu\beta\beta$.

The remainder of this dissertation is organized as follows. Chapter~\ref{chapter:theory} reviews the theoretical framework of neutrinos, neutrino mass, and double beta decay, with emphasis on the relationship between two-neutrino and neutrinoless modes and their associated nuclear matrix elements. Subsequent Chapters~\ref{chapter:klz-detector} -- \ref{chapter:Analysis} describe the KamLAND-Zen detector and dataset, the modeling of signal and background processes, the analysis techniques used to search for excited-state decays, and the resulting constraints and their implications for nuclear theory and future $0\nu\beta\beta$ sensitivity.
