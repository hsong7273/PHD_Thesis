\chapter{Introduction}
\label{chapter:introduction}
\thispagestyle{myheadings}

\graphicspath{{1_Chapter_Intro/Figures/}}

\section{Neutrinos in the Standard Model}
Neutrinos remain the least understood component of the Standard Model of Particle Physics, henceforth referred to as the Standard Model or SM. In 1914, James Chadwick used magnetic spectrometry to resolve the continuous energy spectrum of beta decay. The revelation that the continuous electron energies, in contrast to the constant nuclear recoil energy, emitted from a bound state seemed to violate energy conservation.

In an apologetic attempt to resolve this contradiction, Wolfgang Pauli postulated that a yet undiscovered neutral particle was carrying away the missing energy. In 1930, Pauli addressed his now famous letter to the, "Dear Radioactive Ladies and Gentlemen"\cite{Pauli:1930pc}, a gathering of nuclear scientists at the Gauverin meeting in Tübingen. Enrico Fermi would go on to name Pauli's emitted "neutron" the neutrino or little neutral one. It would take until 1956 for the neutrino to finally be detected in the Cowan-Reines experiment\cite{cowan1956}. Over the next 70 years, the Standard Model has been extended to include the known interactions of massive neutrinos. This section gives a brief introduction to the relevant theoretical foundation of neutrinos in the Standard Model.
\subsection{Neutrino Interactions}
\subsection{Neutrino Masses and Mixing}
\section{Neutrino Mass}
\subsection{Neutrino Mass Heirarchy}
We know the 12 mass splitting sign from the observation of MSW resonance from solar neutrinos.\cite{kamei_phd}
\subsection{Majorana Neutrino Mass}
\subsection{Lepton Number Violation and Leptogenesis}
\subsection{Seesaw Mechanism}

\section{Neutrinoless Double Beta Decay}
\subsection{Decay Process}
\subsection{Black Box Theorem for \0nbb Decay}
\subsection{Detection Experiments}
\subsection{Current Limits}
\subsection{Nuclear Matrix Elements}
\section{Double Beta Decay to Excited States}
\subsection{Impact on Nuclear Matrix Elements}
\subsection{Observations and Current Limits}



