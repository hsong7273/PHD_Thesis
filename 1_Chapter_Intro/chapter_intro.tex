\chapter{Theory of \0nbb, Neutrinoless Double-Beta Decay}
\label{chapter:introduction}
\thispagestyle{myheadings}

\graphicspath{{1_Chapter_Intro/Figures/}}

\section{Neutrinos in the Standard Model}
Neutrinos remain the least understood component of the Standard Model (SM) of Particle Physics\cite{lamport1985:latex}. Their elusive nature and extremely weak interactions make them challenging to study, yet they play a crucial role in both particle physics and cosmology. The story of the neutrino began in 1914, when James Chadwick used magnetic spectrometry to resolve the continuous energy spectrum of beta decay, revealing an apparent violation of energy conservation.

To resolve this, Wolfgang Pauli postulated in 1930 the existence of a neutral, light particle that carried away the missing energy\cite{Pauli:1930pc}. He announced this in his famous letter to the "Radioactive Ladies and Gentlemen." Enrico Fermi later named this particle the "neutrino" ("little neutral one") and incorporated it into his theory of beta decay. The neutrino was finally detected in 1956 by Cowan and Reines\cite{cowan1956}, confirming its existence. Since then, the SM has been extended to include three flavors of neutrinos, each associated with a charged lepton.

The Standard Model is a gauge theory based on the symmetry group $SU(3)_C \times SU(2)_L \times U(1)_Y$\cite{lamport1985:latex}. Neutrinos participate only in the weak interaction, mediated by the $W^\pm$ and $Z^0$ bosons, and are electrically neutral. Their extremely small cross-sections make them difficult to detect, but also allow them to traverse vast distances unimpeded, making them unique cosmic messengers.
\subsection{Neutrino Interactions}

The SM unifies the strong, weak, and electromagnetic interactions under the $SU(3)_C \times SU(2)_L \times U(1)_Y$ gauge symmetry. The $SU(2)_L \times U(1)_Y$ sector describes the electroweak interaction, with the $W^\pm$ and $Z^0$ bosons mediating the weak force. Neutrinos are part of the left-handed lepton doublets, transforming as weak isospin doublets under $SU(2)_L$:


\begin{equation}
    L_\ell = \begin{pmatrix} \nu_{\ell L} \\ \ell_L \end{pmatrix}, \quad \ell = e, \mu, \tau
\end{equation}
where $\nu_{\ell L}$ and $\ell_L$ are the left-handed neutrino and charged lepton fields, respectively. The left-handed projection operator is $P_L = \frac{1-\gamma_5}{2}$, with $\gamma_5 = i\gamma^0\gamma^1\gamma^2\gamma^3$ built from the Dirac matrices.

The electroweak quantum numbers are \textbf{weak isospin} $I$ and \textbf{weak hypercharge} $Y$. The electric charge operator is:
\begin{equation}
    Q = I_3 + \frac{Y}{2}
    \label{eq:charge}
\end{equation}
where $I_3$ is the third component of isospin. For the lepton doublet, $I=1/2$, $Y=-1$, so $Q(\nu_{\ell L})=0$ and $Q(\ell_L)=-1$.

% eigen values table
\begin{table}[h]
    \centering
    \begin{tabular}{lclcccc}
        \hline
        & & & $I$ & $I_3$ & $Y$ & $Q$ \\
        \hline
        lepton doublet & $L_L \equiv$ & $\begin{pmatrix} \nu_{eL} \\ e_L \end{pmatrix}$ & $1/2$ & $1/2$ & $-1$ & $0$ \\
        & & & $1/2$ & $-1/2$ & $-1$ & $-1$ \\
        lepton singlet & $e_R$ & & $0$ & $0$ & $-2$ & $-1$ \\
        quark doublet & $Q_L \equiv$ & $\begin{pmatrix} u_L \\ d_L \end{pmatrix}$ & $1/2$ & $1/2$ & $1/3$ & $2/3$ \\
        & & & $1/2$ & $-1/2$ & $1/3$ & $-1/3$ \\
        quark singlets & $u_R$ & & $0$ & $0$ & $4/3$ & $2/3$ \\
        & $d_R$ & & $0$ & $0$ & $-2/3$ & $-1/3$ \\
        \hline
    \end{tabular}
	\caption{Eigenvalues of the weak isospin $I$, of its third component $I_3$, of the hypercharge $Y$, and of the charge $Q = I_3 + Y/2$ of the fermion doublets and singlets.}
	\label{tab:weakisospin}
\end{table}

The neutrino components of lepton doublets are called "active" neutrinos, in contrast to hypothetical sterile neutrinos. Right-handed fermions are singlets under $SU(2)_L$ and do not participate in weak interactions. The SM contains one active neutrino per charged lepton ($e, \mu, \tau$).

$SU(2)_L$ gauge invariance dictates the form of the weak charged current (CC) and neutral current (NC) interactions:
\begin{align}
    -\mathcal{L}_{\mathrm{CC}} &= \frac{g}{\sqrt{2}} \sum_{\ell} \bar{\nu}_{\ell L} \gamma^\mu \ell_L W_\mu^+ + \text{h.c.} \\
    -\mathcal{L}_{\mathrm{NC}} &= \frac{g}{2\cos\theta_W} \sum_{\ell} \bar{\nu}_{\ell L} \gamma^\mu \nu_{\ell L} Z_\mu^0
    \label{eq:NC}
\end{align}
where $g$ is the weak coupling constant and $\theta_W$ is the Weinberg angle. The $Z^0$ decay width into neutrinos constrains the number of light, active neutrinos to $N_\nu = 2.984 \pm 0.008$\cite{lamport1985:latex}, consistent with three generations.

\subsection{Fermion Masses in the Standard Model}
In the SM, fermion masses arise from Yukawa couplings to the Higgs doublet $\Phi$:
\begin{equation}
    -\mathcal{L}_{\text{Yukawa,lep}} = Y_{ij}^\ell \overline{L}_{Li} \Phi E_{Rj} + \text{h.c.}
	\label{eq:fermion_mass}
\end{equation}
After spontaneous symmetry breaking ($\langle \Phi \rangle = (0, v/\sqrt{2})^T$), this yields charged lepton masses:
\begin{equation}
    m^\ell_{ij} = Y^\ell_{ij} \frac{v}{\sqrt{2}}
\end{equation}
where $v \approx 246$ GeV is the Higgs vacuum expectation value. In the absence of right-handed neutrinos, no analogous Yukawa term exists for neutrinos, so they are massless at tree level in the SM.

\section{Neutrino Mass}
The discovery of neutrino oscillations\cite{Nufit} demonstrates that neutrinos have nonzero masses and that flavor eigenstates are mixtures of mass eigenstates. The exact mechanism by which neutrinos acquire mass is unknown. Several extensions to the SM have been proposed, as discussed below.

Here we follow the derivation in \cite{giunti2007}. Landau, Lee and Yang, and Salam showed that a massless fermion can be described by a chiral field via their two-component theory of massless neutrinos. Let us begin that derivation with the Dirac Equation:
\begin{equation}
    (i\gamma^\mu \partial_\mu - m)\psi = 0
\end{equation}
given a fermion field, $\psi = \psi_L+\psi_R$, the Dirac equation is equivalent to the system of equations:
\begin{equation}
	i\gamma^\mu \partial_\mu\psi_L=m\psi_R
	\label{eq:left_weyl}
\end{equation}
\begin{equation}
	i\gamma^\mu \partial_\mu\psi_R=m\psi_L
	\label{eq:right_weyl}
\end{equation}
for the chiral fields, $\psi_L$ and $\psi_R$, whose space-time evolutions are coupled by the mass $m$.

If the fermion is massless, the two equations, \ref{eq:left_weyl} and \ref{eq:right_weyl}, are decoupled:
\begin{equation}
	i\gamma^\mu \partial_\mu\psi_L=0 
\end{equation}
\begin{equation}
	i\gamma^\mu \partial_\mu\psi_R=0
\end{equation}
Thus, a massless fermion can be completely described by a single chiral field (either left-handed or right-handed) which has only two independent components. The equations, \ref{eq:left_weyl} and \ref{eq:right_weyl} are known as the Weyl equations and the spinors $\psi_L$ and $\psi_R$ are the Weyl spinors.

The simplest form of the SM incorporates what is known as the two-component theory of massless neutrinos. Whereby the neutrino is entirely described by the left-handed Weyl spinor which participates in the weak interaction, $\nu_L$, and there are no $\nu_R$ fields.

\subsection{Dirac Masses}
If right-handed neutrino fields $\nu_{R}$ exist, a Dirac mass term, just like the one for the charged leptons, can be written:
\begin{equation}
    -\mathcal{L}_{\text{Dirac}} = Y_{ij}^\nu \overline{L}_{Li} \tilde{\Phi} \nu_{Rj} + \text{h.c.}
\end{equation}
where $\tilde{\Phi} = i \sigma_2 \Phi^*$. This yields Dirac masses $m^\nu_{ij} = Y^\nu_{ij} v/\sqrt{2}$. However, the tiny observed neutrino masses ($m_\nu < 1$ eV) would require $Y^\nu_{ij} < 10^{-12}$. The huge discrepancy between the neutrino masses and the other fermions imply the existence of some underlying mechanism which suppresses the neutrino masses. In the abscence of such explanation, the light neutrino masses bring up a naturalness problem. Many neutrino mass models have been proposed that produce light neutrino masses via a more natural mechanism. 

\subsection{Majorana Neutrino Mass}
Since neutrinos have indeed been shown to have mass, the two-component theory is insufficient. In 1937, Ettore Majorana proposed a new solution to the Dirac equation. His insight was that a massive fermion could be described with a single spinor instead of the two, $\psi=\psi_L+\psi_R$. Majorana made the assumption that the two spinors are not independent, but rather:
\begin{equation}
	\psi_R=C\bar{\psi_L}^T
\end{equation} 
where $C$ is the charge conjugation operator. By observing that $C$ and the left-handed projection operator $P_L$ have the following relationship,
\begin{equation}
	P_L(C\psi_L^T)=0
\end{equation}
one can clearly see that $C\psi_L^T$ is a right-handed field. Charge conjugating a left-hadned Weyl spinor converts the spinor to its right-handed form. Modifying the Dirac equation, we now obtain the Majorana equation for the chiral field:
\begin{equation}
	i\gamma^\mu\partial_\mu\psi=m\psi^C
\end{equation}
$\psi^C$ represents the charge conjugated Majorana field. This implies that, $\psi=\psi_L+\psi_L^C$, which finally leads to the Majorana relation:
\begin{equation}
	\psi=\psi^C
	\label{eq:majorana}
\end{equation}

Equation \ref{eq:majorana} implies that the particle $\psi$ is its own antiparticle. Since neutrinos interact only through weak interactions, and are electrically neutral, the charge parity of the neutrino field has no physical meaning and can be chosen arbitrarily. Among the elementary fermions, only the neutrinos are neutral and have the potential to be Majorana particles. 

Should the neutrino be Majorana, the neutrino and antineutrino would only be distinguishable by their helicities. It is customary to refer to negative helicity neutrinos as "neutrinos" and positive helicity neutrinos as "antineutrinos".

With majorana neutrinos, the simplest mass term one could construct with SM fields and respecting SM symmetries is the lepton number violating term:
\begin{equation}
	\mathcal{L}_5=\frac{Z_{ij}^\nu}{\Lambda}(\bar{L_L^i} \bar{\Phi})(\bar{\Phi}^T L_L^j)+h.c.
	\label{eq:eff_lagrangian}
\end{equation}
Here, $Z_{ij}^\nu$ is a 3x3 matrix which controls the mixing between the neutrino masses for each flavor combination, along with the Yukawa coupling strengths. Finally, $\Lambda$ is the high energy scale at which we should expect new physics, which suppresses the neutrino masses. This lagrangian extension generates the Majorana neutrino mass term:
\begin{equation}
	\mathcal{L}_{M_\nu}=\frac{Z_{ij}^\nu}{2}\frac{v^2}{\Lambda}\bar{\nu}_{L_i}\nu^C_{L_j}+h.c.
\end{equation}
With the majorana neutrino mass matrix:
\begin{equation}
	\mathcal{M}_\nu=Z_{ij}^\nu\frac{v^2}{\Lambda}
\end{equation}

Comparing the effective lagrangian, equation \ref{eq:eff_lagrangian}, to the charged lepton mass terms, equation \ref{eq:fermion_mass}, our new effective Lagrangian term has two Higgs fields to couple to the extra $L_L$ field. This raises the dimension of this operator to 5, rendering the term non-renormalizable while following the SM gauge transformations. This term hints at new physics beyond the Standard Model at the mass scale $\Lambda$. Also, note that this term is the only possible dimension-5 Lagrangian term to generate neutrino mass. The new physics mass scale, $\Lambda$, suppresses the neutrino masses by $\frac{v^2}{\Lambda}$. This suppression of the neutrino masses has the same structure as the masses produced by the seesaw mechanism that will be discussed in the next section. 

\subsection{Seesaw Mechanism}
We will now discuss one interesting extension of the Standard Model which produces light active neutrinos, the addition of one or more heavy sterile neutrinos. Adding $m$ new sterile neutrinos, $\nu_{si}\ (i=1,...m)$, leads to two types of mass terms.
\begin{equation}
	-\mathcal{L}_{M_\nu}=M_{D_{ij}}\bar{\nu}_{si}\nu_{L_j}+\frac{1}{2}M_{N_{ij}}\bar{\nu}_{si}\nu_{sj}^c+h.c.
	\label{eq:seesaw_lagrangian}
\end{equation}
Here, $M_D$ is a complex matrix of dimension $m\times 3$ and $M_N$ is a symmetric $m\times m$ matrix.

The first term is often referred to as a Dirac mass term, after spontaneous symmetry breaking, the neutrinos acquire a mass through Yukawa couplings, similar to the charged fermions and as discussed in an earlier section. 
\begin{equation}
    Y_{ij}^\nu \bar{\nu}_{si} \tilde{\phi}^\dagger L_{Lj} \;\;\Rightarrow\;\; M_{D_{ij}} = Y_{ij}^\nu \frac{v}{\sqrt{2}},
\end{equation}
The second term is a lepton number violating Majorana mass term.

Eq. \ref{eq:seesaw_lagrangian} can be rewritten as:
\begin{equation}
    -\mathcal{L}_{M_\nu} = \frac{1}{2}
    \begin{pmatrix}
        \overline{\nu_L} & \overline{\nu_s}
    \end{pmatrix}
    \begin{pmatrix}
        0 & M_D^T \\
        M_D & M_N
    \end{pmatrix}
    \begin{pmatrix}
        \nu_L^c \\
        \nu_s^c
    \end{pmatrix}
    + \text{h.c.}
    \equiv \overline{\nu} M_\nu \nu + \text{h.c.}
	\label{eq:seesaw_matrix}
\end{equation}
where $\vec{\nu}=\left(\vec{\nu}_L,\vec{\nu^c}_s\right)^T$ is a $(3+m)$-dimensional vector. The matrix $M_\nu$ is complex and symmetric, and can be diagonalized producing mass eigenstates:
\begin{equation}
	(V^\nu)^TM_\nu V^\nu=diag(m_1,m_2,...,M_{3+m})
\end{equation}
\begin{equation}
	\vec{\nu}_{mass} = (V^\nu)^\dagger\vec{\nu}
\end{equation}
Eq. \ref{eq:seesaw_matrix} can be rewritten in terms of the mass eigenstates:
\begin{align}
    -\mathcal{L}_{M_\nu} &= \frac{1}{2} \sum_{k=1}^{3+m} m_k \left( \bar{\nu}_{\text{mass},k}^c \nu_{\text{mass},k} + \bar{\nu}_{\text{mass},k} \nu_{\text{mass},k}^c \right) \\
    &= \frac{1}{2} \sum_{k=1}^{3+m} m_k \overline{\nu}_{Mk} \nu_{Mk},
\end{align}
where $\nu_{Mk} = \nu_{\text{mass},k} + \nu_{\text{mass},k}^c$. Thus, the Majorana condition is satisfied, $\nu_M=\nu_M^c$, and are referred to as Majorana neutrinos. 

In this new neutrino mass basis, the original weak-interacting neutrino fields are:
\begin{equation}
	\nu_{Li} = P_L\sum_{j=1}^{3+m}V_{ij}^\nu\nu_{M_j},\ i=1,2,3
\end{equation}

In the specific case, where the $M_N$ mass eigenvalues are much higher than the scale of electroweak symmetry breaking, $v$, the diagonalization of $M_\nu$ leads to three light neutrinos, $\nu_l$, and $m$ heavy neutrinos $N$:
\begin{equation}
	-\mathcal{L}_{M_\nu}=\frac{1}{2}\bar{\nu}_lM^l\nu_l+\frac{1}{2}\bar{N}M^hN
\end{equation}
where
\begin{align}
    M^l &\simeq -V_l^T M_D^T M_N^{-1} M_D V_l, \\
    M^h &\simeq V_h^T M_N V_h,
\end{align}
\begin{equation}
    V^\nu \simeq
    \begin{bmatrix}
        \left(1 - \frac{1}{2} M_D^\dagger M_N^{*-1} M_N^{-1} M_D \right) V_l & M_D^\dagger M_N^{*-1} V_h \\
        -M_N^{-1} M_D V_l & \left(1 - \frac{1}{2} M_N^{-1} M_D M_D^\dagger M_N^{*-1} \right) V_h
    \end{bmatrix},
\end{equation}
where $V_l$ and $V_h$ are $3\times 3$ and $m\times m$ matrices respectively that parameterize neutrino mixing between the light nad heavy states. $V_l$ could be thought of as the PMNS, Pontecorve-Maki-Nakagawa-Sakata, matrix which we will discuss in a later section in more detail. We can see that the heavier mass states are proportional to $M_N$, while the lighter mass states are proportional to $M_N^{-1}$. This behavior is where the seesaw mechanism gets its name. This seesaw mechanism is of Type I specifically, where new sterile neutrinos are added. The seesaw mechanism also produces heavy states that are mostly right-handed, while the light states are mostly left-handed.

Thus, the Type I seesaw mechanism is a promising extension to the SM that produces light, weakly interacting, left-handed neutrinos and heavy, sterile, right-handed neutrinos, while providing a more natural suppression of the active neutrino masses. Not to mention that these heavy, right-handed steriles could also be candidates for dark matter.
\subsection{Lepton Number Violation and Leptogenesis}
A key consequence of Majorana neutrinos is lepton number violation. Lepton number violation is what is known as an accidental global symmetry in the Standard Model. Where the symmetry is not enforced when constructing the model, but comes about simply because no terms that violate it were included. Many beyond the Standard Model theories include lepton number violation.

Lepton number violation also plays a key role in leptogenesis. Leptogenesis is a proposed solution to one of the most fundamental questions in elementary physics: Why is there more matter than antimatter?

A thorough review of the phenemology of matter-antimatter asymmetry is beyond the scope of this chapter. In lieu of that, here is a highly condensed description of the matter-antimatter asymmetry of our universe.

The value of the baryon asymmetry of the Universe is inferred from two independent observations. The first is of the abundances of light elements, $D$, $^3He$, $^4He$, and $^7Li$, after big bang nucleosynthesis. These abundances depend on the assymmetry parameter, $\eta$, measured to be \cite{Davidson_2008}:
\begin{equation}
	\eta^{BBN} \equiv \frac{n_B-n_{\bar{B}}}{n_\gamma}|_0=4.7-6.5\times 10^{-10}
\end{equation}
The second observation is of the cosmic microwave background (CMB) anisotropies \cite{Hu_2002}. A key CMB observable is $c_s$, the speed of sound in the photon-baryon fluid. Measuring the temperature fluctuations in the CMB constrain the baryon energy density, $\rho_B$, which is related to $\eta$ by:
\begin{equation}
	\Omega_B=\frac{\rho_B}{\rho_{crit}}
\end{equation}
\begin{equation}
	\eta^{CMB}=2.74\times 10^{-8}\Omega_B h^2=6.1^{+0.3}_{-0.2}\times 10^{-10}
\end{equation}
where $h=H_0/100$ km $s^{-1}$ Mpc$^{-1}=0.682\pm 0.0028$ is the present Hubble parameter\cite{DESI}. The impressive consistency between the nucleosynthesis and CMB constraints on the baryon density of the Universe is a triumph of hot big-bang cosmology.

Sakharov outlined three conditions for generating a dynamic baryon asymmetry, now referred to as the Sakharov conditions.
\begin{enumerate}
	\item Baryon number violation
	\item C and CP violation
	\item Out of equilibrium dynamics
\end{enumerate}
While the Standard Model features all three ingredients, no SM mechanism generates a large enough baryon asymmetry. Leptogenesis is a beyond the Standard Model theory that introduces the previously discussed singlet neutrinos. Along with new sphaleron processes that can convert lepton number violation into baryon number violation, The Type I seesaw sterile neutrions play a pivotal role in Leptogenesis.

\subsection{Neutrino Mass Hierarchy}
We've discussed open questions relating to neutrino nature; whether they are dirac or majorana, what is the absolute neutrino mass scale, and their potential role in resolving matter-antimatter asymmetry in the Universe. Another open question has to do with the relative ordering of the light neutrino masses.

In the simplified 2-flavor case, the probability of a neutrino "surviving" or being observed in the flavor state in which it was produced, is:
\begin{equation}
    P(\nu_e\rightarrow \nu_e)=1-\frac{1}{2}\sin^22\theta_{12}\sin^2\left(\frac{\delta m^2_{21}}{4E}L\right)
\end{equation}
Note that while the survival probability or one minus the oscillation probability is is dependent on the absolute value of the squared-mass difference, $\delta m_{ij}^2=m_i^2-m_j^2$, it is insensitvie to its' sign. Solar neutrino oscillation experiments have measured the following squared- mass difference:
\begin{equation}
	\delta m_{21}^2=\delta m^2_{sol}\approx 7.39^{+0.21}_{-0.20} \times 10^{-5} eV^2
\end{equation}
while observation of atmospheric oscillation data results in:
\begin{equation}
	|\delta m_{13}^2| = |\delta m_{23}^2| = \delta m_{atm}^2\approx 2.449^{+0.032}_{-0.030} \times 10^{-3} eV^2
\end{equation}
As mentioned previously, oscillation experiments demonstrated that neutrinos have mass, however these experiments only measure the mass-squared differences.

While observation of neutrino oscilation enhancement in matter determined the sign of $\delta m_{21}^2$, the signs of $\delta m_{13}^2$ and $|\delta m_{23}^2|$ are unknown. This leaves two possible orderings of the three neutrino mass eigenstates.
\begin{itemize}
	\item \textbf{Inverted Ordering} with negative $\delta m_{13}^2$ and $m_3 < m_1 < m_2$
	\item \textbf{Normal Ordering} with positive $\delta m_{13}^2$ and $m_1<m_2<m_3$
\end{itemize}
Where $m_1$ is the neutrino mass eigenstate with the highest $\nu_e$ flavor content. According to combined T2K, Super-K atmospheric analysis, the normal ordering is slightly favored \cite{wester_2004}. 
\section{Neutrinoless Double Beta Decay}

Neutrinoless Double Beta Decay, \0nbb, is a hypothesized process that has implications for all the open questions about neutrino nature discussed so far. It is the most sensitive probe into whether neutrinos are Dirac or Majorana. 
\begin{equation}
	(A,Z)\rightarrow (A, Z+2)+2e^-
\end{equation}
This lepton number violating process is shown in Fig \ref{fig:0nbb}.

Double Beta Decay, \2nbb, is a rare nuclear process by which two neutrons simultaneously undergo beta decay, emitting a neutrino and electron. First studied by Maria Goeppert-Mayer in 1935, \2nbb is a Standard Model process that has now been observed in multiple isobaric nuclei, nuclei with even numbers of protons and neutrons.
\begin{equation}
	(A,Z)\rightarrow (A,Z+2)+2e^-+2\bar{\nu}_e
\end{equation}
In some of these isotopes, single beta decay is energetically forbidden. Because the isotope a single-beta decay step away is at a higher energy level than the isotope two steps away. Figure \ref{fig:nuclear_mass} shows how isobaric nuclei can produce this arrangement of nuclear states. As a second-order weak interaction, \2nbb, is the slowest proess in the universe that has been experimentally observed. 

\begin{figure}[h]
  \centering
  \begin{tikzpicture}
    \begin{feynman}
      \vertex (n1) at (0,1.5) {$n$};
      \vertex (n2) at (0,-1.5) {$n$};
      \vertex (v1) at (2,1.5);
      \vertex (v2) at (2,-1.5);
      \vertex (w1) at (4,0.75);
      \vertex (w2) at (4,-0.75);
      \vertex (p1) at (6,2) {$p$};
      \vertex (p2) at (6,-2) {$p$};
      \vertex (e1) at (6,0.75) {$e^-$};
      \vertex (e2) at (6,-0.75) {$e^-$};
      
      \diagram* {
        (n1) -- [fermion] (v1) -- [fermion] (p1),
        (n2) -- [fermion] (v2) -- [fermion] (p2),
        (v1) -- [boson, edge label=$W$] (w1) -- [fermion] (e1),
        (v2) -- [boson, edge label'=$W$] (w2) -- [fermion] (e2),
        (w1) -- [fermion, edge label=$\nu$] (w2),
      };
    \end{feynman}
  \end{tikzpicture}
  \caption{Feynman diagram for neutrinoless double beta decay ($0\nu\beta\beta$).}
  \label{fig:0nbb}
\end{figure}

In \0nbb, the final state electrons carry away almost the entire decay energy, while the energy is split between electrons and neutrinos in \2nbb. When the easier to observe electron kinetic energies are accumulated, the distribution is as shown in Figure \ref{fig:2nbb_energy}. The sharp peak at the endpoint of the \2nbb decay spectrum is the experimental signature of \0nbb.

The key observable in a \0nbb decay experiment is the half-life of \0nbb. A predicition for the half-life is given by:
\begin{equation}
	|T^{0\nu}_{1/2}|^{-1}=G_{0\nu}|M_{0\nu}|^2\left(\frac{m_{\beta\beta}}{m_e}\right)^2
	\label{eq:half-life}
\end{equation}
where $G_{0\nu}$ is the phase space integral accounting for the final states of the elctrons and daughter atom, $|M_{0\nu}|$ is the nuclear matrix element of the transition, and $m_{\beta\beta}$ is known as the effective Majorana mass of $\nu_e$,
\begin{equation}
m_{\beta\beta} = \left| \sum_i m_i U_{ei}^2 \right| = 
\left\{
\begin{array}{ll}
m_0 c_{12}^2 c_{13}^2 + \sqrt{\Delta m_{21}^2 + m_0^2} s_{12}^2 c_{13}^2 e^{2i(\eta - \eta_1)} \\
\quad + \sqrt{\Delta m_{32}^2 + \Delta m_{21}^2 + m_0^2} s_{13}^2 e^{-2i(\delta_{\text{CP}} + \eta_1)} & \text{in NO,} \\[2ex]
m_0 s_{13}^2 + \sqrt{m_0^2 - \Delta m_{32}^2} s_{12}^2 c_{13}^2 e^{2i(\eta_2 + \delta_{\text{CP}})} \\
\quad + \sqrt{m_0^2 - \Delta m_{32}^2 - \Delta m_{21}^2} c_{12}^2 c_{13}^2 e^{2i(\eta_1 + \delta_{\text{CP}})} & \text{in IO,}
\end{array}
\right.
\end{equation}
Thus, the effective majorana mass is sensitive to the PMNS parameters which govern neutrino oscillation, such as the neutrino masses and mixing angles. It is also sensitive to the majorana CP violating phases, $\eta_1$ and $\eta_2$, which neutrino oscillations are independent of.

As the Majorana phases are unknown, there is a range of allowed $m_{\beta\beta}$ bounded by the oscillation parameters. Figure \ref{fig:lobster} shows how the allowed regions vary over $m_{lightest}$.

\subsection{Black Box Theorem for $0\nu\beta\beta$ Decay}
The half-life definition of Eq. \ref{eq:half-life} assumes the light neutrino exchange mechanism for \0nbb shown in Figure \ref{fig:0nbb}. Under this assumption the interaction strength is governed by the effective Majorana mass, $m_{\beta\beta}$. While there are other possible mechanisms that could facility \0nbb, a key insight was given by Schecther and Valle in \cite{blackbox}. In what is now known as the famous Black Box Theorem, they propose that\cite{merle_blackbox}:
\begin{itemize}
	\item Should \0nbb be observed, its Feynman diagram must feature two electrons, two up-quark fields, and two down-quark fields. The process connecting these fields is arbitrary and is referred to as the "black box process". The theorem argues that this "black box process" effectively establishes the dimension-9 operator
	\item The up and down quarks are contracted by the W boson
	\item On the other end of the W boson propagators, electron fields are converted into neutrino fields
	\item The entire diagram can be rotated to turn into a process that converts anti-neutrinos to neutrinos as shown in Figure \ref{fig:blackbox}
	\item Finally, the possible cancelation of this process by other diagrams is dismissed by naturalness arguments.
\end{itemize}

\begin{figure}[h]
	\centering
	\includegraphics[scale=0.2]{blackbox.png}
	\caption{Depiction of the \0nbb black box theorem, the black box represents an arbitrary \0nbb process, which can be used to convert antineutrinos into neutrinos. Figure taken from \cite{merle_blackbox}}
	\label{fig:blackbox}
\end{figure}

The key conclusion of the black-box theorem is that should \0nbb be observed, even if the observed mechanism is not light majorana neutrino exchange, the neutrino is a majorana particle. It should be noted that since the theorem's original proposal, counterexamples have been found allowing \0nbb without majorana neutrinos, but the theorem still indicates potential links between \0nbb, Majorana mass, and lepton number violation more broadly.

\subsection{Double Beta Decay Experiments}
\0nbb if it exists has an incredibly long half-life greater than $10^{26}$ years. Should it be discovered, \0nbb would be the slowest natural process observed experimentally. To successfully carry out this rare event search, \0nbb experiments must be designed with crucial criteria in mind:
\begin{itemize}
	\item \textbf{High Energy Resolution:} Distinguishing the \0nbb peak at the \2nbb endpoint requires precise energy resolution to reduce ROI (region of interest) contamination from the \2nbb decay spectrum. Modern experiments on the forefront of this design criterion are achieving $\frac{\sigma_E}{E}\approxeq 0.1\%$ resolution, and completely suppressing the \2nbb background. 
	\item \textbf{High Isotope Loading:}
	\item \textbf{Low Unrelated Backgrounds:}
\end{itemize}
\subsection{Nuclear Matrix Elements}

\section{Double Beta Decay to Excited States}

\subsection{Impact on Nuclear Matrix Elements}

\subsection{Observations and Current Limits of $2\nu\beta\beta^*$}

