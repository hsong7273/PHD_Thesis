\chapter{KamLAND-ZEN Simulation and Reconstruction}
\label{chapter:details}
\thispagestyle{myheadings}

% set this to the location of the figures for this chapter. it may
% also want to be ../Figures/2_Body/ or something. make sure that
% it has a trailing directory separator (i.e., '/')!
\graphicspath{{3_Chapter_Details/Figures/}}

KamLAND-ZEN uses detailed simulations defined in KLG4Sim, a GEANT4-based Monte Carlo (MC) simulation software. The MC simulated events are tuned with real calibration events to carefully match real detector response. Simulated and physical events produce detector responses that are reconstructed to extract higher-level information such as energy and position. The reconstructed event information is used for data-selection and spectrum fitting. This chapter discusses the MC simulation and event reconstruction procedures used in KamLAND-ZEN 800.

\section*{Analysis Framework}
\subsection*{Data Flow}
Figure outlines the data flow in KamLAND-ZEN. PMT signals are digitized in either KamFEE or MoGURA, the two DAQ systems discussed in the previous chapter, the digitized signals are stored in Kinoko Data Format (KDF). KDF files contain trigger information and timestamped, digitized PMT waveforms. KDF files also store run condition information in the header.

For some students adding the following two lines in ``thesis.tex'' preamble has worked:\\
%
{\tt
$\backslash$usepackage[T1]\{fontenc\}\\
$\backslash$usepackage{pslatex}
   } 


The easiest way to check if fonts are embedded well and of what type, is to use Adobe Acrobat's Preflight -- it shows exactly where the Type 3 fonts are in the thesis. You can learn more here: \url{https://community.adobe.com/t5/acrobat/figure-out-where-a-specific-font-is-used-in-a-pdf/m-p/10880057?page=1#M238035}

If you don't have Adobe Acrobat (BU students get it for free), you can quickly check which fonts have which type by looking into Files $>>$ Properties $>>$ Fonts, but it doesn't tell where the text with a specific font type is.

{\bf Linux/Unix}: If you are using LaTeX or Unix, the problem is that, by default, LaTeX uses Type 3 fonts. Since most users have a tendency to use the default settings, then Type 3 fonts will be used by default. You can try to change the first line in the preamble in ``thesis.tex'' to:\\
%
{\tt $\backslash$documentstyle[12pt,times,letterpaper]\{report\}}

\noindent
since then Times fonts will be used (which are not Type 3). If there are mathematical formulas in the text, it is better to use:\\
%
{\tt $\backslash$documentstyle[12pt,times,mathptm,letterpaper]\{report\}}


\section{Font embedding}

All fonts must be embedded into the final PDF file. If they are not, sometimes equations may look strange or may not show up at all for several pages. This is often due to unembedded font problem. Should you have a font-embedding issue, this page may prove useful:\\
%
\url{https://www.karlrupp.net/2016/01/embed-all-fonts-in-pdfs-latex-pdflatex}

For those using Overleaf, this page might help:
\url{https://www.overleaf.com/learn/latex/Questions/My_submission_was_rejected_by_the_journal_because_%22Font_XYZ_is_not_embedded%22._What_can_I_do%3F}
